\documentclass[a4paper, 11pt]{article}
\usepackage{comment} % enables the use of multi-line comments (\ifx \fi)
\usepackage{lipsum} %This package just generates Lorem Ipsum filler text.
\usepackage{fullpage} % changes the margin
\usepackage[a4paper, total={7in, 10in}]{geometry}
\usepackage[fleqn]{amsmath}
\usepackage{amssymb,amsthm}  % assumes amsmath package installed
\newtheorem{theorem}{Theorem}
\newtheorem{corollary}{Corollary}
\usepackage{graphicx}
\usepackage{tikz}
\usetikzlibrary{arrows}
\usepackage{verbatim}
\usepackage[numbered]{mcode}
\usepackage{float}
\usepackage{tikz}
    \usetikzlibrary{shapes,arrows}
    \usetikzlibrary{arrows,calc,positioning}

    \tikzset{
        block/.style = {draw, rectangle,
            minimum height=1cm,
            minimum width=1.5cm},
        input/.style = {coordinate,node distance=1cm},
        output/.style = {coordinate,node distance=4cm},
        arrow/.style={draw, -latex,node distance=2cm},
        pinstyle/.style = {pin edge={latex-, black,node distance=2cm}},
        sum/.style = {draw, circle, node distance=1cm},
    }
\usepackage{xcolor}
\usepackage{mdframed}
\usepackage[shortlabels]{enumitem}
\usepackage{indentfirst}
\usepackage{hyperref}
\usepackage{bm}


\renewcommand{\thesubsection}{\thesection.\alph{subsection}}

\newenvironment{problem}[2][Problem]
    { \begin{mdframed}[backgroundcolor=gray!20] \textbf{#1 #2} \\}
    {  \end{mdframed}}

% Define solution environment
\newenvironment{solution}
    {\textit{Solution:}}
    {}

\renewcommand{\qed}{\quad\qedsymbol}
%%%%%%%%%%%%%%%%%%%%%%%%%%%%%%%%%%%%%%%%%%%%%%%%%%%%%%%%%%%%%%%%%%%%%%%%%%%%%%%%%%%%%%%%%%%%%%%%%%%%%%%%%%%%%%%%%%%%%%%%%%%%%%%%%%%%%%%%
\begin{document}
%Header-Make sure you update this information!!!!
\noindent
%%%%%%%%%%%%%%%%%%%%%%%%%%%%%%%%%%%%%%%%%%%%%%%%%%%%%%%%%%%%%%%%%%%%%%%%%%%%%%%%%%%%%%%%%%%%%%%%%%%%%%%%%%%%%%%%%%%%%%%%%%%%%%%%%%%%%%%%
\large\textbf{Yonghua Su} \hfill \textbf{Homework04}   \\
Email: 2019000154@ruc.edu.cn \hfill ID: 2019000154 \\
\normalsize Course: Stochastic Processes \hfill Term: Spring 2020\\
% Instructor: Dr. Sriram \hfill Due Date: $22^{nd}$ November, 2019 \\
\noindent\rule{7in}{2.8pt}
%%%%%%%%%%%%%%%%%%%%%%%%%%%%%%%%%%%%%%%%%%%%%%%%%%%%%%%%%%%%%%%%%%%%%%%%%%%%%%%%%%%%%%%%%%%%%%%%%%%%%%%%%%%%%%%%%%%%%%%%%%%%%%%%%%%%%%%%
% Problem 1
%%%%%%%%%%%%%%%%%%%%%%%%%%%%%%%%%%%%%%%%%%%%%%%%%%%%%%%%%%%%%%%%%%%%%%%%%%%%%%%%%%%%%%%%%%%%%%%%%%%%%%%%%%%%%%%%%%%%%%%%%%%%%%%%%%%%%%%%
\begin{problem}{1.55}
  \begin{equation*}
  \begin{array}{l}\text { The Markov chain associated with a manufacturing process may be de- } \\ \text { scribed as follows: A part to be manufactured will begin the process by entering } \\ \text { step 1. After step 1, 20\% of the parts must be reworked, i.e., returned to step } \\ 1,10 \% \text { of the parts are thrown away, and 70\% proceed to step 2. After step 2, } \\ 5 \% \text { of the parts must be returned to the step 1, 10\% to step 2, 5\% are scrapped, } \\ \text { and 80\% emerge to be sold for a profit. (a) Formulate a four-state Markov chain } \\ \text { with states 1, 2, 3, and 4 where 3 = a part that was scrapped and 4 = a part } \\ \text { that was sold for a profit. (b) Compute the probability a part is scrapped in } \\ \text { the production process. }\end{array}
  \end{equation*}
\end{problem}

\begin{solution}

\end{solution}

\noindent\rule{7in}{2.8pt}
%%%%%%%%%%%%%%%%%%%%%%%%%%%%%%%%%%%%%%%%%%%%%%%%%%%%%%%%%%%%%%%%%%%%%%%%%%%%%%%%%%%%%%%%%%%%%%%%%%%%%%%%%%%%%%%%%%%%%%%%%%%%%%%%%%%%%%%%
% Problem 2
%%%%%%%%%%%%%%%%%%%%%%%%%%%%%%%%%%%%%%%%%%%%%%%%%%%%%%%%%%%%%%%%%%%%%%%%%%%%%%%%%%%%%%%%%%%%%%%%%%%%%%%%%%%%%%%%%%%%%%%%%%%%%%%%%%%%%%%%
\begin{problem}{1.59}
  \begin{equation*}
  \begin{array}{l}\text { Use the second solution in Example } 1.48 \text { to compute the expected waiting } \\ \text { times for the patterns } H H H, H H T, H T T, \text { and } H T H . \text { Which pattern has the } \\ \text { longest waiting time? Which ones achieve the minimum value of } 8 ?\end{array}
  \end{equation*}
\end{problem}

\begin{solution}
  
\end{solution}

\noindent\rule{7in}{2.8pt}
%%%%%%%%%%%%%%%%%%%%%%%%%%%%%%%%%%%%%%%%%%%%%%%%%%%%%%%%%%%%%%%%%%%%%%%%%%%%%%%%%%%%%%%%%%%%%%%%%%%%%%%%%%%%%%%%%%%%%%%%%%%%%%%%%%%%%%
% Problem 3
%%%%%%%%%%%%%%%%%%%%%%%%%%%%%%%%%%%%%%%%%%%%%%%%%%%%%%%%%%%%%%%%%%%%%%%%%%%%%%%%%%%%%%%%%%%%%%%%%%%%%%%%%%%%%%%%%%%%%%%%%%%%%%%%%%%%%%%%
\begin{problem}{1.60}
  \begin{equation*}
  \begin{array}{l}\text {Sucker bet. Consider the following gambling game. Player 1 picks a three } \\ \text { coin pattern (for example } H T H \text { and player 2 picks another (say } T H H \text { ). A coin } \\ \text { is flipped repeatedly and outcomes are recorded until one of the two patterns } \\ \text { appears. Somewhat surprisingly player 2 has a considerable advantage in this } \\ \text { game. No matter what player 1 picks, player 2 can win with probability } \geq 2 / 3 \text { . } \\ \text { Suppose without loss of generality that player 1 picks a pattern that begins } \\ \text { with H: }\end{array}
  \end{equation*}
  \begin{equation*}
  \begin{array}{cccc}\hline \text { case } & \text { Player 1 } & \text { Player 2 } & \text { Prob. 2 wins } \\ \hline 1 & \mathrm{HHH} & \mathrm{THH} & 7 / 8 \\ 2& \mathrm{HHT} & \mathrm{THH} & 3 / 4 \\ 3 & \mathrm{HTH} & \mathrm{HHT} & 2 / 3 \\ 4 & \mathrm{HTT} & \mathrm{HHT} & 2 / 3\end{array}
  \end{equation*}
  Verify the results in the table. You can do this by solving six equations in six unknowns but this is not the easiest way.
\end{problem}

\begin{solution}

\end{solution}

\noindent\rule{7in}{2.8pt}
%%%%%%%%%%%%%%%%%%%%%%%%%%%%%%%%%%%%%%%%%%%%%%%%%%%%%%%%%%%%%%%%%%%%%%%%%%%%%%%%%%%%%%%%%%%%%%%%%%%%%%%%%%%%%%%%%%%%%%%%%%%%%%%%%%%%%%%%
% Problem 4
%%%%%%%%%%%%%%%%%%%%%%%%%%%%%%%%%%%%%%%%%%%%%%%%%%%%%%%%%%%%%%%%%%%%%%%%%%%%%%%%%%%%%%%%%%%%%%%%%%%%%%%%%%%%%%%%%%%%%%%%%%%%%%%%%%%%%%%%
\begin{problem}{1.69}
  \begin{equation*}
  \begin{array}{l}\text { Algorthmic efficiency. The simplex method minimizes linear functions by } \\ \text { moving between extreme points of a polyhedral region so that each transition } \\ \text { decreases the objective function. Suppose there are } n \text { extreme points and they } \\ \text { are numbered in increasing order of their values. Consider the Markov chain in } \\ \text { which } p(1,1)=1 \text { and } p(i, j)=1 / i-1 \text { for } j<i . \text { In words, when we leave } j \text { we } \\ \text { are equally likely to go to any of the extreme points with better value. (a) Use } \\ (1.25) \text { to show that for } i>1 \end{array}
  \end{equation*}
  \begin{equation*}
  \begin{array}{l}\qquad E_{i} T_{1}=1+1 / 2+\cdots+1 /(i-1) \\ \text { (b) Let } I_{j}=1 \text { if the chain visits } j \text { on the way from } n \text { to } 1 . \text { Show that for } j<n \\ \qquad P\left(I_{j}=1 | I_{j+1}, \ldots I_{n}\right)=1 / j \\ \text { to get another proof of the result and conclude that } I_{1}, \ldots I_{n-1} \text { are independent. }\end{array}
  \end{equation*}
\end{problem}

\begin{solution}

\end{solution}
\noindent\rule{7in}{2.8pt}
%%%%%%%%%%%%%%%%%%%%%%%%%%%%%%%%%%%%%%%%%%%%%%%%%%%%%%%%%%%%%%%%%%%%%%%%%%%%%%%%%%%%%%%%%%%%%%%%%%%%%%%%%%%%%%%%%%%%%%%%%%%%%%%%%%%%%%
% Problem 5
%%%%%%%%%%%%%%%%%%%%%%%%%%%%%%%%%%%%%%%%%%%%%%%%%%%%%%%%%%%%%%%%%%%%%%%%%%%%%%%%%%%%%%%%%%%%%%%%%%%%%%%%%%%%%%%%%%%%%%%%%%%%%%%%%%%%%%
\begin{problem}{1.71}
  To see what the conditions in the last problem say we will now consider
  some concrete examples. Let \(p_{x}=1 / 2, q_{x}=e^{-c x^{-\alpha}} / 2, r_{x}=1 / 2-q_{x}\) for
  \(x \geq 1\) and \(p_{0}=1 .\) For large \(x, q_{x} \approx\left(1-c x^{-\alpha}\right) / 2,\) but the exponential
  formulation keeps the probabilities nonnegative and makes the problem easier
  to solve. Show that the chain is recurrent if \(\alpha>1\) or if \(\alpha=1\) and \(c \leq 1\) but is
  transient otherwise.
\end{problem}

\begin{solution}

\end{solution}
\noindent\rule{7in}{2.8pt}
%%%%%%%%%%%%%%%%%%%%%%%%%%%%%%%%%%%%%%%%%%%%%%%%%%%%%%%%%%%%%%%%%%%%%%%%%%%%%%%%%%%%%%%%%%%%%%%%%%%%%%%%%%%%%%%%%%%%%%%%%%%%%%%%%%%%%%%%
% Problem 6
%%%%%%%%%%%%%%%%%%%%%%%%%%%%%%%%%%%%%%%%%%%%%%%%%%%%%%%%%%%%%%%%%%%%%%%%%%%%%%%%%%%%%%%%%%%%%%%%%%%%%%%%%%%%%%%%%%%%%%%%%%%%%%%%%%%%%%%%
\begin{problem}{1.72}
  Consider the Markov chain with state space \(\{0,1,2, \ldots\}\) and transition
  probability
  $$
  \begin{aligned} p(m, m+1)=\frac{1}{2}\left(1-\frac{1}{m+2}\right) & \text { for } m \geq 0 \\ p(m, m-1)=\frac{1}{2}\left(1+\frac{1}{m+2}\right) & \text { for } m \geq 1 \end{aligned}
  $$
  and \(p(0,0)=1-p(0,1)=3 / 4 .\) Find the stationary distribution \(\pi\)
\end{problem}

\begin{solution}

\end{solution}

\noindent\rule{7in}{2.8pt}
%%%%%%%%%%%%%%%%%%%%%%%%%%%%%%%%%%%%%%%%%%%%%%%%%%%%%%%%%%%%%%%%%%%%%%%%%
\end{document}
