\documentclass[a4paper, 11pt]{article}
\usepackage{comment} % enables the use of multi-line comments (\ifx \fi) 
\usepackage{lipsum} %This package just generates Lorem Ipsum filler text. 
\usepackage{fullpage} % changes the margin
\usepackage[a4paper, total={7in, 10in}]{geometry}
\usepackage[fleqn]{amsmath}
\usepackage{amssymb,amsthm}  % assumes amsmath package installed
\newtheorem{theorem}{Theorem}
\newtheorem{corollary}{Corollary}
\usepackage{graphicx}
\usepackage{tikz}
\usetikzlibrary{arrows}
\usepackage{verbatim}
\usepackage[numbered]{mcode}
\usepackage{float}
\usepackage{tikz}
    \usetikzlibrary{shapes,arrows}
    \usetikzlibrary{arrows,calc,positioning}

    \tikzset{
        block/.style = {draw, rectangle,
            minimum height=1cm,
            minimum width=1.5cm},
        input/.style = {coordinate,node distance=1cm},
        output/.style = {coordinate,node distance=4cm},
        arrow/.style={draw, -latex,node distance=2cm},
        pinstyle/.style = {pin edge={latex-, black,node distance=2cm}},
        sum/.style = {draw, circle, node distance=1cm},
    }


\usepackage{xcolor}
\usepackage{mdframed}
\usepackage[shortlabels]{enumitem}
\usepackage{indentfirst}
\usepackage{hyperref}
    
\renewcommand{\thesubsection}{\thesection.\alph{subsection}}

\newenvironment{problem}[2][Problem]
    { \begin{mdframed}[backgroundcolor=gray!20] \textbf{#1 #2} \\}
    {  \end{mdframed}}

% Define solution environment
\newenvironment{solution}
    {\textit{Solution:}}
    {}

\renewcommand{\qed}{\quad\qedsymbol}
%%%%%%%%%%%%%%%%%%%%%%%%%%%%%%%%%%%%%%%%%%%%%%%%%%%%%%%%%%%%%%%%%%%%%%%%%%%%%%%%%%%%%%%%%%%%%%%%%%%%%%%%%%%%%%%%%%%%%%%%%%%%%%%%%%%%%%%%
\begin{document}
%Header-Make sure you update this information!!!!
\noindent
%%%%%%%%%%%%%%%%%%%%%%%%%%%%%%%%%%%%%%%%%%%%%%%%%%%%%%%%%%%%%%%%%%%%%%%%%%%%%%%%%%%%%%%%%%%%%%%%%%%%%%%%%%%%%%%%%%%%%%%%%%%%%%%%%%%%%%%%
\large\textbf{*** } \hfill \textbf{Homework - 7}   \\
Email: ***@ruc.edu.cn \hfill ID: 20190001** \\
\normalsize Course:Stochastic Processes \hfill Term: Spring 2020\\
%Instructor: Dr. He \hfill Due Date: $8^{th}$ Mar., 2020 \\
\noindent\rule{7in}{2.8pt}
%%%%%%%%%%%%%%%%%%%%%%%%%%%%%%%%%%%%%%%%%%%%%%%%%%%%%%%%%%%%%%%%%%%%%%%%%%%%%%%%%%%%%%%%%%%%%%%%%%%%%%%%%%%%%%%%%%%%%%%%%%%%%%%%%%%%%%%%
% Problem 1
%%%%%%%%%%%%%%%%%%%%%%%%%%%%%%%%%%%%%%%%%%%%%%%%%%%%%%%%%%%%%%%%%%%%%%%%%%%%%%%%%%%%%%%%%%%%%%%%%%%%%%%%%%%%%%%%%%%%%%%%%%%%%%%%%%%%%%%%
\begin{problem}{2.7}
	Let $S$ and $T$ be exponentially distributed with rates $\lambda$ and $\mu$. Let $U =
	\min\{S, T\}$ and $V = \max\{S, T\}$. Find 
	
	(a) $\mathbb{E}U$. 
	
	(b) $\mathbb{E}(V-U)$, 
	
	(c) $\mathbb{E}V$,
	
	(d) Use the identity $V = S +T -U$ to get a different looking formula for $\mathbb{E}V$ and verify
	the two are equal.
\end{problem}
\begin{solution}


\end{solution} 
\noindent\rule{7in}{2.8pt}

%%%%%%%%%%%%%%%%%%%%%%%%%%%%%%%%%%%%%%%%%%%%%%%%%%%%%%%%%%%%%%%%%%%%%%%%%
% Problem 2
%%%%%%%%%%%%%%%%%%%%%%%%%%%%%%%%%%%%%%%%%%%%%%%%%%%%%%%%%%%%%%%%%%%%%%%%%%%%%%%%%%%%%%%%%%%%%%%%%%%%%%%%%%%%%%%%%%%%%%%%%%%%%%%%%%%%%%%%

\begin{problem}{2.12}
	A flashlight needs two batteries to be operational. You start with four
	batteries numbered $1$ to $4$. Whenever a battery fails it is replaced by the lowestnumbered working battery. Suppose that battery life is exponential with mean
	$100$ hours. Let $T$ be the time at which there is one working battery left and $N$
	be the number of the one battery that is still good. 
	
	(a) Find $\mathbb{E}T$. 
	
	(b) Find the distribution of $N$. 
	
	(c) Solve (a) and (b) for a general number of batteries.
\end{problem}
\begin{solution}

	
\end{solution} 

\noindent\rule{7in}{2.8pt}
%%%%%%%%%%%%%%%%%%%%%%%%%%%%%%%%%%%%%%%%%%%%%%%%%%%%%%%%%%%%%%%%%%%%%%%%%
% Problem 3
%%%%%%%%%%%%%%%%%%%%%%%%%%%%%%%%%%%%%%%%%%%%%%%%%%%%%%%%%%%%%%%%%%%%%%%%%%%%%%%%%%%%%%%%%%%%%%%%%%%%%%%%%%%%%%%%%%%%%%%%%%%%%%%%%%%%%%%%

\begin{problem}{2.13}
A machine has two critically important parts and is subject to three
different types of shocks. Shocks of type $i$ occur at times of a Poisson process
with rate $\lambda_i$. Shocks of types $1$ break part $1$, those of type $2$ break part $2$,
while those of type $3$ break both parts. Let $U$ and $V$ be the failure times of the
two parts. 

(a) Find $\Pr(U > s, V > t)$. 

(b) Find the distribution of $U$ and the
distribution of $V$. 

(c) Are $U$ and $V$ independent?
\end{problem}

\begin{solution}
	
\end{solution} 

\noindent\rule{7in}{2.8pt}
%%%%%%%%%%%%%%%%%%%%%%%%%%%%%%%%%%%%%%%%%%%%%%%%%%%%%%%%%%%%%%%%%%%%%%%%%
% Problem 4
%%%%%%%%%%%%%%%%%%%%%%%%%%%%%%%%%%%%%%%%%%%%%%%%%%%%%%%%%%%%%%%%%%%%%%%%%%%%%%%%%%%%%%%%%%%%%%%%%%%%%%%%%%%%%%%%%%%%%%%%%%%%%%%%%%%%%%%%

\begin{problem}{2.17}
Let $T_i, i = 1, 2, 3$ be independent exponentials with rate $\lambda_i$. 

(a) Show
that for any numbers $t_1, t_2, t_3$
$$\max\{t_1, t_2, t_3\} = t_1 + t_2 + t_3 - \min\{t_1, t_2\} - \min\{t_1, t_3\}
- \min\{t_2, t_3\} + \min\{t_1, t_2, t_3\}$$

(b) Use (a) to find $\mathbb{E} \max\{T_1, T_2, T_3\}$. 

(c) Use the formula to give a simple
solution of part (c) of Exercise 2.16.

\end{problem}
\begin{solution}
	
\end{solution} 

\noindent\rule{7in}{2.8pt}
%%%%%%%%%%%%%%%%%%%%%%%%%%%%%%%%%%%%%%%%%%%%%%%%%%%%%%%%%%%%%%%%%%%%%%%%%
% Problem 5
%%%%%%%%%%%%%%%%%%%%%%%%%%%%%%%%%%%%%%%%%%%%%%%%%%%%%%%%%%%%%%%%%%%%%%%%%%%%%%%%%%%%%%%%%%%%%%%%%%%%%%%%%%%%%%%%%%%%%%%%%%%%%%%%%%%%%%%%

\begin{problem}{2.22}
	Suppose $N(t)$ is a Poisson process with rate $3$. Let $T_n$ denote the time of
	the $n$-th arrival. Find 
	
	(a) $\mathbb{E}(T_{12})$, 
	
	(b) $\mathbb{E}(T_{12}|N(2) = 5)$, 
	
	(c) $\mathbb{E}(N(5)|N(2) = 5)$
\end{problem}
\begin{solution}
	
\end{solution} 

\noindent\rule{7in}{2.8pt}
%%%%%%%%%%%%%%%%%%%%%%%%%%%%%%%%%%%%%%%%%%%%%%%%%%%%%%%%%%%%%%%%%%%%%%%%%
% Problem 6
%%%%%%%%%%%%%%%%%%%%%%%%%%%%%%%%%%%%%%%%%%%%%%%%%%%%%%%%%%%%%%%%%%%%%%%%%%%%%%%%%%%%%%%%%%%%%%%%%%%%%%%%%%%%%%%%%%%%%%%%%%%%%%%%%%%%%%%%

\begin{problem}{2.29}
	Consider a Poisson process with rate $\lambda$ and let $L$ be the time of the last arrival in the interval $[0, t]$, with $L = 0$ if there was no arrival. 
	
	(a) Compute
	$\mathbb{E}(t - L)$ 
	
	(b) What happens when we let $t\rightarrow\infty$ in the answer to (a)?
\end{problem}
\begin{solution}
\end{solution} 

\noindent\rule{7in}{2.8pt}
%%%%%%%%%%%%%%%%%%%%%%%%%%%%%%%%%%%%%%%%%%%%%%%%%%%%%%%%%%%%%%%%%%%%%%%%%

%%%%%%%%%%%%%%%%%%%%%%%%%%%%%%%%%%%%%%%%%%%%%%%%%%%%%%%%%%%%%%%%%%%%%%%%%
%%%%%%%%%%%%%%%%%%%%%%%%%%%%%%%%%%%%%%%%%%%%%%%%%%%%%%%%%%%%%%%%%%%%%%%%%
% Problem 7
%%%%%%%%%%%%%%%%%%%%%%%%%%%%%%%%%%%%%%%%%%%%%%%%%%%%%%%%%%%%%%%%%%%%%%%%%%%%%%%%%%%%%%%%%%%%%%%%%%%%%%%%%%%%%%%%%%%%%%%%%%%%%%%%%%%%%%%%

\begin{problem}{2.31}
	Customers arrive at a sporting goods store at rate 10 per hour. $60\%$ of
	the customers are men and $40\%$ are women. Women spend an amount of time
	shopping that is uniformly distributed on $[0, 30]$ minutes, while men spend an
	exponentially distributed amount of time with mean $30$ minutes. Let $M$ and
	$N$ be the number of men and women in the store. What is the distribution of
	$(M, N)$ in equilibrium.
\end{problem}
\begin{solution}
\end{solution} 

\noindent\rule{7in}{2.8pt}
%%%%%%%%%%%%%%%%%%%%%%%%%%%%%%%%%%%%%%%%%%%%%%%%%%%%%%%%%%%%%%%%%%%%%%%%%

%%%%%%%%%%%%%%%%%%%%%%%%%%%%%%%%%%%%%%%%%%%%%%%%%%%%%%%%%%%%%%%%%%%%%%%%%
\end{document}
 