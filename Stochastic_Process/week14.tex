\documentclass[a4paper, 11pt]{article}
\usepackage{comment} % enables the use of multi-line comments (\ifx \fi) 
\usepackage{lipsum} %This package just generates Lorem Ipsum filler text. 
\usepackage{fullpage} % changes the margin
\usepackage[a4paper, total={7in, 10in}]{geometry}
\usepackage[fleqn]{amsmath}
\usepackage{amssymb,amsthm}  % assumes amsmath package installed
\newtheorem{theorem}{Theorem}
\newtheorem{corollary}{Corollary}
\usepackage{graphicx}
\usepackage{tikz}
\usetikzlibrary{arrows}
\usepackage{verbatim}
\usepackage[numbered]{mcode}
\usepackage{float}
\usepackage{tikz}
    \usetikzlibrary{shapes,arrows}
    \usetikzlibrary{arrows,calc,positioning}

    \tikzset{
        block/.style = {draw, rectangle,
            minimum height=1cm,
            minimum width=1.5cm},
        input/.style = {coordinate,node distance=1cm},
        output/.style = {coordinate,node distance=4cm},
        arrow/.style={draw, -latex,node distance=2cm},
        pinstyle/.style = {pin edge={latex-, black,node distance=2cm}},
        sum/.style = {draw, circle, node distance=1cm},
    }
\usepackage{xcolor}
\usepackage{mdframed}
\usepackage[shortlabels]{enumitem}
\usepackage{indentfirst}
\usepackage{hyperref}
    
\renewcommand{\thesubsection}{\thesection.\alph{subsection}}

\newenvironment{problem}[2][Problem]
    { \begin{mdframed}[backgroundcolor=gray!20] \textbf{#1 #2} \\}
    {  \end{mdframed}}

% Define solution environment
\newenvironment{solution}
    {\textit{Solution:}}
    {}

\renewcommand{\qed}{\quad\qedsymbol}
%%%%%%%%%%%%%%%%%%%%%%%%%%%%%%%%%%%%%%%%%%%%%%%%%%%%%%%%%%%%%%%%%%%%%%%%%%%%%%%%%%%%%%%%%%%%%%%%%%%%%%%%%%%%%%%%%%%%%%%%%%%%%%%%%%%%%%%%
\begin{document}
%Header-Make sure you update this information!!!!
\noindent
%%%%%%%%%%%%%%%%%%%%%%%%%%%%%%%%%%%%%%%%%%%%%%%%%%%%%%%%%%%%%%%%%%%%%%%%%%%%%%%%%%%%%%%%%%%%%%%%%%%%%%%%%%%%%%%%%%%%%%%%%%%%%%%%%%%%%%%%
\large\textbf{} \hfill \textbf{Homework - 14}   \\
Email: @ruc.edu.cn \hfill ID: 201900015\\
\normalsize Course: Stochastic Process \hfill Term: Spring 2020\\
Instructor: Dr. Zhang \& Dr. Dai \hfill Due Date: $3^{rd}$ June, 2020 \\
\noindent\rule{7in}{2.8pt}
%%%%%%%%%%%%%%%%%%%%%%%%%%%%%%%%%%%%%%%%%%%%%%%%%%%%%%%%%%%%%%%%%%%%%%%%%%%%%%%%%%%%%%%%%%%%%%%%%%%%%%%%%%%%%%%%%%%%%%%%%%%%%%%%%%%%%%%%
% Problem 1
%%%%%%%%%%%%%%%%%%%%%%%%%%%%%%%%%%%%%%%%%%%%%%%%%%%%%%%%%%%%%%%%%%%%%%%%%%%%%%%%%%%%%%%%%%%%%%%%%%%%%%%%%%%%%%%%%%%%%%%%%%%%%%%%%%%%%%%%
\begin{problem}{3.10}
Derive the spectral density for $u(t)=2 x(t) x^{\prime}(t)$ if $x(t)$ is a differentiable stationary Gaussian process with spectral density $f_{x}(\omega)$

\end{problem}
\begin{solution}



\end{solution} 
\noindent\rule{7in}{2.8pt}

%%%%%%%%%%%%%%%%%%%%%%%%%%%%%%%%%%%%%%%%%%%%%%%%%%%%%%%%%%%%%%%%%%%%%%%%%
% Problem 2
%%%%%%%%%%%%%%%%%%%%%%%%%%%%%%%%%%%%%%%%%%%%%%%%%%%%%%%%%%%%%%%%%%%%%%%%%%%%%%%%%%%%%%%%%%%%%%%%%%%%%%%%%%%%%%%%%%%%%%%%%%%%%%%%%%%%%%%%

\begin{problem}{3.11}
Let $e_{t}, t=0,\pm 1,\pm 2, \ldots,$ be independent standard normal variables and define, for $|\theta|<1$, the stationary processes
\[
x_{t}=\theta x_{t-1}+e_{t}=\sum_{n=-\infty}^{t} \theta^{t-n} e_{n}, \quad y_{t}=e_{t}+\psi e_{t-1}
\]
a) Find the covariances and spectral densities of $x_{t}$ and $y_{t}$\\
b) Express the spectral processes $Z_{x}(\omega)$ and $Z_{y}(\omega)$ in terms of the spectral process $Z_{e}(\omega),$ and derive the cross-spectrum between $x_{t}$ and $y_{t}$

\end{problem}
\begin{solution}


\end{solution} 
%\lstinputlisting{HW6Q2.m}
\noindent\rule{7in}{2.8pt}
%%%%%%%%%%%%%%%%%%%%%%%%%%%%%%%%%%%%%%%%%%%%%%%%%%%%%%%%%%%%%%%%%%%%%%%%%
% Problem 3
%%%%%%%%%%%%%%%%%%%%%%%%%%%%%%%%%%%%%%%%%%%%%%%%%%%%%%%%%%%%%%%%%%%%%%%%%%%%%%%%%%%%%%%%%%%%%%%%%%%%%%%%%%%%%%%%%%%%%%%%%%%%%%%%%%%%%%%%

\begin{problem}{3.12}
Let $u_{n}$ and $v_{n}$ be two sequences of independent, identically distributed variables with zero mean and let the stationary sequences $x_{n}$ and $y_{n}$ be defined by
\[
y_{n}=a_{1}+b_{1} x_{n-1}+u_{n}, \quad x_{n}=a_{2}-b_{2} y_{n}+v_{n}
\]
Express the spectral processes $\mathrm{d} Z_{x}$ and $\mathrm{d} Z_{y}$ as functions of $u_{n}$ and $v_{n},$ and derive the spectral densities for $x_{n}$ and $y_{n}$ and their cross-spectrum.

\end{problem}
\begin{solution}

\end{solution} 
%\lstinputlisting{HW6Q2.m}
\noindent\rule{7in}{2.8pt}
%%%%%%%%%%%%%%%%%%%%%%%%%%%%%%%%%%%%%%%%%%%%%%%%%%%%%%%%%%%%%%%%%%%%%%%%%
% Problem 4
%%%%%%%%%%%%%%%%%%%%%%%%%%%%%%%%%%%%%%%%%%%%%%%%%%%%%%%%%%%%%%%%%%%%%%%%%%%%%%%%%%%%%%%%%%%%%%%%%%%%%%%%%%%%%%%%%%%%%%%%%%%%%%%%%%%%%%%%

\begin{problem}{3.13}
Show the following form of the law of large numbers: If $\left\{x_{n}, n \in \mathbb{N}\right\}$ is a stationary sequence with spectral process $Z(\omega),$ then
\[
\frac{1}{n} \sum_{k=1}^{n} x_{n} \stackrel{q \cdot m}{\rightarrow} z(0+)-Z(0-)
\]
cf. the limit (3.28) in Theorem 3.8.

\end{problem}
\begin{solution}



\end{solution} 
%\lstinputlisting{HW6Q2.m}
\noindent\rule{7in}{2.8pt}
%%%%%%%%%%%%%%%%%%%%%%%%%%%%%%%%%%%%%%%%%%%%%%%%%%%%%%%%%%%%%%%%%%%%%%%%%
% Problem 5
%%%%%%%%%%%%%%%%%%%%%%%%%%%%%%%%%%%%%%%%%%%%%%%%%%%%%%%%%%%%%%%%%%%%%%%%%%%%%%%%%%%%%%%%%%%%%%%%%%%%%%%%%%%%%%%%%%%%%%%%%%%%%%%%%%%%%%%%

\begin{problem}{3.15}
Take a stationary process $\{x(t), t \in \mathbb{R}\}$ with $E(x(t))=m$ and spectral density $f(\omega),$ and use it to amplitude modulate a pure randomly shifted cosine function with constant frequency $\Gamma$. Find the covariance function, spectral density, and spectral representation of the resulting process
\[
y(t)=x(t) \cos (\Gamma t+\phi)
\]
when $\phi$ is uniformly distributed in $(0,2 \pi)$

\end{problem}
\begin{solution}



	
\end{solution} 
%\lstinputlisting{HW6Q2.m}
\noindent\rule{7in}{2.8pt}
%%%%%%%%%%%%%%%%%%%%%%%%%%%%%%%%%%%%%%%%%%%%%%%%%%%%%%%%%%%%%%%%%%%%%%%%%




%%%%%%%%%%%%%%%%%%%%%%%%%%%%%%%%%%%%%%%%%%%
%%%%%%%%%%%%%%%%%%%%%%%%%%%%%%%%%%%%%%%%%%%
\end{document}