\documentclass[a4paper, 11pt]{article}
\usepackage{comment} % enables the use of multi-line comments (\ifx \fi) 
\usepackage{lipsum} %This package just generates Lorem Ipsum filler text. 
\usepackage{fullpage} % changes the margin
\usepackage[a4paper, total={7in, 10in}]{geometry}
\usepackage[fleqn]{amsmath}
\usepackage{amssymb,amsthm}  % assumes amsmath package installed
\newtheorem{theorem}{Theorem}
\newtheorem{corollary}{Corollary}
\usepackage{graphicx}
\usepackage{tikz}
\usetikzlibrary{arrows}
\usepackage{verbatim}
\usepackage[numbered]{mcode}
\usepackage{float}
\usepackage{diagbox} 
\newcommand{\normal}{\mathcal{N}}
\newcommand{\E}{\mathbb{E}}
\renewcommand{\P}{\mathbb{P}}
\newcommand{\Var}{\mathrm{Var}}
\newcommand{\Cov}{\mathrm{Cov}}
\usepackage{tikz}
    \usetikzlibrary{shapes,arrows}
    \usetikzlibrary{arrows,calc,positioning}

    \tikzset{
        block/.style = {draw, rectangle,
            minimum height=1cm,
            minimum width=1.5cm},
        input/.style = {coordinate,node distance=1cm},
        output/.style = {coordinate,node distance=4cm},
        arrow/.style={draw, -latex,node distance=2cm},
        pinstyle/.style = {pin edge={latex-, black,node distance=2cm}},
        sum/.style = {draw, circle, node distance=1cm},
    }
\usepackage{xcolor}
\usepackage{mdframed}
\usepackage[shortlabels]{enumitem}
\usepackage{indentfirst}
\usepackage{hyperref}
    
\renewcommand{\thesubsection}{\thesection.\alph{subsection}}

\newenvironment{problem}[2][Problem]
    { \begin{mdframed}[backgroundcolor=gray!20] \textbf{#1 #2} \\}
    {  \end{mdframed}}

% Define solution environment
\newenvironment{solution}
    {\textit{Solution:}}
    {}

\renewcommand{\qed}{\quad\qedsymbol}
%%%%%%%%%%%%%%%%%%%%%%%%%%%%%%%%%%%%%%%%%%%%%%%%%%%%%%%%%%%%%%%%%%%%%%%%%%%%%%%%%%%%%%%%%%%%%%%%%%%%%%%%%%%%%%%%%%%%%%%%%%%%%%%%%%%%%%%%
\begin{document}
%%%%%%%%%%%%%%%%%%%%%%%%%%%%%%%%%%%%%%%%%%%%%%%%%%%%%%%%%%%%%%%%%%%%%%%%%%%%%%%%%%%%%%%%%%%%%%%%%%%%%%%%%%%%%%%%%%%%%%%%%%%%%%%%%%%%%%%%
\noindent
%%%%% Information!!!
\large\textbf{xxx} \hfill \textbf{Homework - 16}   \\
Email: xxx \hfill ID: xxx \\
\normalsize Course: Stochastic Processes \hfill Term: Spring 2020\\
%Instructor: Zheng Zhang, Wenlin Dai \hfill Due Date: $17^{th}$ June, 2020 \\
\noindent\rule{7in}{2.8pt}
%%%%%%%%%%%%%%%%%%%%%%%%%%%%%%%%%%%%%%%%%%%%%%%%%%%%%%%%%%%%%%%%%%%%%%%%%%%%%%%
% Problem 1

\begin{problem}{4.6}
	 Consider the covariance function
	\[
	r_{y}(t)=\frac{\sigma^{2}}{4 \alpha \omega_{0}^{2}} e^{-\alpha|t|}\left(\cos \beta t+\frac{\alpha}{\beta} \sin \beta|t|\right)
	\]
	of the linear oscillator in Example 4.5 on page 138 The covariance function contains some non-differentiable $|t| ;$ show that it still fulfills a condition for sample function differentiability, but not for twice differentiability.
	
	Find the relation between the relative damping $\zeta$ and the spectral width parameter $\alpha=\omega_{2} / \sqrt{\omega_{0} \omega_{4}}.$ 
\end{problem}
\begin{solution}

...

\end{solution} 

\noindent\rule{7in}{2.8pt}
%%%%%%%%%%%%%%%%%%%%%%%%%%%%%%%%%%%%%%%%%%%%%%%%%%%%%%%%%%%%%%%%%%%%%%%%%
%%%%%%%%%%%%%%%%%%%%%%%%%%%%%%%%%%%%%%%%%%%%%%%%%%%%%%%%%%%%%%%%%%%%%%%%%%%%%%%


\begin{problem}{4.7}
Define
\[
v(T)=V\left(\frac{1}{T} \int_{0}^{T} x(t) \mathrm{d} t\right)
\]
and assume $\{x(t), t \in \mathbb{R}\}$ has a bounded spectral density $f(\omega)$. Find the relation between $f(0)$ and the asymtotic behavior of $v(T)$ as $T \rightarrow \infty.$
\end{problem}
\begin{solution}

...

\end{solution} 

\noindent\rule{7in}{2.8pt}
%%%%%%%%%%%%%%%%%%%%%%%%%%%%%%%%%%%%%%%%%%%%%%%%%%%%%%%%%%%%%%%%%%%%%%%%%
%%%%%%%%%%%%%%%%%%%%%%%%%%%%%%%%%%%%%%%%%%%%%%%%%%%%%%%%%%%%%%%%%%%%%%%%%%%%%%%


\begin{problem}{4.14}
	Take a spectral distribution $F(\omega)$ with unit mass over $(-\pi, \pi],$ with $F(\pi)-F(-\pi+)=1,$ and let the random variable $v$ be distributed according to $F$. Also, let $u$ be uniformly distributed over $[-\pi, \pi),$ and independent of $v .$ Prove that the random sequence $x_{n}=e^{i(v n+u)}$ is stationary, compute its covariance function, and show that its spectrum is $F$.
\end{problem}
\begin{solution}
	
...
	
\end{solution} 

\noindent\rule{7in}{2.8pt}
%%%%%%%%%%%%%%%%%%%%%%%%%%%%%%%%%%%%%%%%%%%%%%%%%%%%%%%%%%%%%%%%%%%%%%%%%
%%%%%%%%%%%%%%%%%%%%%%%%%%%%%%
%%%%%%%%%%%%%%%%%%%%%%%%%%%%%%%%%%%%%%%%%%%%%%%%%


\begin{problem}{5.4}
Let $z_{k}$ be independent standard normal variables and define
\[
x(t)=\sqrt{2} \sum_{k=1}^{\infty} \frac{z_{k}}{\pi k} \sin (\pi k t), \quad 0 \leq t \leq 1
\]
This is the Karhunen-Loève representation of a Gaussian process with covariance function
\[
r_{x}(s, t)=\sum_{k=1}^{\infty} \frac{2}{\pi^{2} k^{2}} \sin (\pi k s) \sin (\pi k t)
\]
What process is this? Hint: Compare $r_{x}(1-s, 1-t)$ and $r_{x}(s, t).$
\end{problem}
\begin{solution}
	
	...
	
\end{solution} 

\noindent\rule{7in}{2.8pt}
%%%%%%%%%%%%%%%%%%%%%%%%%%%%%%%%%%%%%%%%%%%%%%%%%%%%%%%%%%%%%%%%%%%%%%%%%
%%%%%%%%%%%%%%%%%%%%%%%%%%%%%%%%%%%%%%%%%%%%%%%%%%%%%%%%%%%%%%%%%%%%%%%%%%%%%%%


\begin{problem}{5.5}
	Let $\{w(t), 0 \leq t<\infty\}$ be standard Brownian motion, and let $A$ have $E(A)=0, V(A)=1,$ and be independent of $\{w(t)\} .$ Suppose that $x(s)=$ $A s+w(s)$ is observed for $0 \leq s \leq t$. Find the best estimate of $A$ in mean square sense of the form $\widehat{A}(t)=\int_{0}^{t} h(t, s) \mathrm{d} x(s) ;[123].$
\end{problem}
\begin{solution}
	
	...
	
\end{solution} 

\noindent\rule{7in}{2.8pt}
%%%%%%%%%%%%%%%%%%%%%%%%%%%%%%%%%%%%%%%%%%%%%%%%%%%%%%%%%%%%%%%%%%%%%%%%%
%%%%%%%%%%%%%%%%%%%%%%%%%%%%%%%%%%%%%%%%%%%%%%%%%%%%%%%%%%%%%%%%%%%%%%%%%%%%%%%



\begin{problem}{5.6}
Show that
\[
\int_{a}^{b} E\left(\left|x(t)-\sum_{0}^{n} \sqrt{\lambda_{k}} \phi_{k}(t) z_{k}\right|^{2}\right) \mathrm{d} t=\sum_{n+1}^{\infty} \lambda_{k}
\]
and prove the optimality claim statement in Remark 5.4.
\end{problem}
\begin{solution}
	
	...
	
\end{solution} 

\noindent\rule{7in}{2.8pt}
%%%%%%%%%%%%%%%%%%%%%%%%%%%%%%%%%%%%%%%%%%%%%%%%%%%%%%%%%%%%%%%%%%%%%%%%%
%%%%%%%%%%%%%%%%%%%%%%%%%%%%%%%%%%%%%%%%%%%%%%%%%%%%%%%%%%%%%%%%%%%%%%%%%%%%%%%


\end{document}