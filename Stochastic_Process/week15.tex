\documentclass[a4paper, 11pt]{article}
\usepackage{comment} % enables the use of multi-line comments (\ifx \fi) 
\usepackage{lipsum} %This package just generates Lorem Ipsum filler text. 
\usepackage{fullpage} % changes the margin
\usepackage[a4paper, total={7in, 10in}]{geometry}
\usepackage[fleqn]{amsmath}
\usepackage{amssymb,amsthm}  % assumes amsmath package installed
\newtheorem{theorem}{Theorem}
\newtheorem{corollary}{Corollary}
\usepackage{graphicx}
\usepackage{tikz}
\usetikzlibrary{arrows}
\usepackage{verbatim}
\usepackage[numbered]{mcode}
\usepackage{float}
\usepackage{diagbox} 
\newcommand{\normal}{\mathcal{N}}
\newcommand{\E}{\mathbb{E}}
\renewcommand{\P}{\mathbb{P}}
\newcommand{\Var}{\mathrm{Var}}
\newcommand{\Cov}{\mathrm{Cov}}
\usepackage{tikz}
    \usetikzlibrary{shapes,arrows}
    \usetikzlibrary{arrows,calc,positioning}

    \tikzset{
        block/.style = {draw, rectangle,
            minimum height=1cm,
            minimum width=1.5cm},
        input/.style = {coordinate,node distance=1cm},
        output/.style = {coordinate,node distance=4cm},
        arrow/.style={draw, -latex,node distance=2cm},
        pinstyle/.style = {pin edge={latex-, black,node distance=2cm}},
        sum/.style = {draw, circle, node distance=1cm},
    }
\usepackage{xcolor}
\usepackage{mdframed}
\usepackage[shortlabels]{enumitem}
\usepackage{indentfirst}
\usepackage{hyperref}
    
\renewcommand{\thesubsection}{\thesection.\alph{subsection}}

\newenvironment{problem}[2][Problem]
    { \begin{mdframed}[backgroundcolor=gray!20] \textbf{#1 #2} \\}
    {  \end{mdframed}}

% Define solution environment
\newenvironment{solution}
    {\textit{Solution:}}
    {}

\renewcommand{\qed}{\quad\qedsymbol}
%%%%%%%%%%%%%%%%%%%%%%%%%%%%%%%%%%%%%%%%%%%%%%%%%%%%%%%%%%%%%%%%%%%%%%%%%%%%%%%%%%%%%%%%%%%%%%%%%%%%%%%%%%%%%%%%%%%%%%%%%%%%%%%%%%%%%%%%
\begin{document}
%%%%%%%%%%%%%%%%%%%%%%%%%%%%%%%%%%%%%%%%%%%%%%%%%%%%%%%%%%%%%%%%%%%%%%%%%%%%%%%%%%%%%%%%%%%%%%%%%%%%%%%%%%%%%%%%%%%%%%%%%%%%%%%%%%%%%%%%
\noindent
\large\textbf{xx} \hfill \textbf{Homework15}   \\
Email: 20190001xx@ruc.edu.cn \hfill ID: 20190001xx \\
\normalsize Course: Stochastic Processes \hfill Term: Spring 2020\\
%Instructor: Zheng Zhang, Wenlin Dai \hfill Due Date: $21^{st}$ May, 2020 \\
\noindent\rule{7in}{2.8pt}
%%%%%%%%%%%%%%%%%%%%%%%%%%%%%%%%%%%%%%%%%%%%%%%%%%%%%%%%%%%%%%%%%%%%%%%%%%%%%%%
% Problem 1

\begin{problem}{4.1}
    \begin{enumerate}[(a)]
        \item Use the Taylor expansion of $E(e^{i(t_1x_1+t_2x_2+t_3x_3+t_4x_4)})$ to prove
        $$E(x_1x_2x_3x_4) = E(x_1x_2)E(x_3x_4) +E(x_1x_3)E(x_2x_4)+E(x_1x_4)E(x_2x_3)-2E(x_1)E(x_2)E(x_3)E(x_4)$$
        for Gaussian $x_1,x_2,x_3,x_4$. For normal variables with mean zero, this is a special case of Isserlis’ theorem.
        \item Let $\{x(t), t \in R\}$ be a stationary Gaussian process with mean zero, 
        covariance function $r_x(t)$, and spectral density $f_x(\omega)$, and consider the process
        $$y(t) = x(t) x(t -T)$$
        where $T$ is a constant delay. Derive expressions for the covariance function and spectral density of $y(t)$ 
        in terms of $r_x$ and $f_x$.
    \end{enumerate}
\end{problem}
\begin{solution}

\end{solution} 

\noindent\rule{7in}{2.8pt}
%%%%%%%%%%%%%%%%%%%%%%%%%%%%%%%%%%%%%%%%%%%%%%%%%%%%%%%%%%%%%%%%%%%%%%%%%
%%%%%%%%%%%%%%%%%%%%%%%%%%%%%%%%%%%%%%%%%%%%%%%%%%%%%%%%%%%%%%%%%%%%%%%%%%%%%%%


\begin{problem}{4.5}
    Define the process $y(t) = x(t) +\int_0^\infty e^{-u}x(t -u)du$ as the output from a
    linear filter with $x(t)$ as input. What are the impulse and frequency functions of the filter?
     Derive the covariance function and spectrum of $y(t)$ if
    $x(t)$ is stationary with covariance function $r_x(t) = e^{-2|t|}$.
    
\end{problem}
\begin{solution}
    
\end{solution} 

\noindent\rule{7in}{2.8pt}
%%%%%%%%%%%%%%%%%%%%%%%%%%%%%%%%%%%%%%%%%%%%%%%%%%%%%%%%%%%%%%%%%%%%%%%%%
%%%%%%%%%%%%%%%%%%%%%%%%%%%%%%%%%%%%%%%%%%%%%%%%%%%%%%%%%%%%%%%%%%%%%%%%%%%%%%%





\end{document}