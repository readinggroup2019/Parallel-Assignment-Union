\documentclass[a4paper, 11pt]{article}
\usepackage{comment} % enables the use of multi-line comments (\ifx \fi)
\usepackage{lipsum} %This package just generates Lorem Ipsum filler text.
\usepackage{fullpage} % changes the margin
\usepackage[a4paper, total={7in, 10in}]{geometry}
\usepackage[fleqn]{amsmath}
\usepackage{amssymb,amsthm}  % assumes amsmath package installed
\newtheorem{theorem}{Theorem}
\newtheorem{corollary}{Corollary}
\usepackage{graphicx}
\usepackage{tikz}
\usetikzlibrary{arrows}
\usepackage{verbatim}
\usepackage[numbered]{mcode}
\usepackage{float}
\usepackage{tikz}
    \usetikzlibrary{shapes,arrows}
    \usetikzlibrary{arrows,calc,positioning}

    \tikzset{
        block/.style = {draw, rectangle,
            minimum height=1cm,
            minimum width=1.5cm},
        input/.style = {coordinate,node distance=1cm},
        output/.style = {coordinate,node distance=4cm},
        arrow/.style={draw, -latex,node distance=2cm},
        pinstyle/.style = {pin edge={latex-, black,node distance=2cm}},
        sum/.style = {draw, circle, node distance=1cm},
    }
\usepackage{xcolor}
\usepackage{mdframed}
\usepackage[shortlabels]{enumitem}
\usepackage{indentfirst}
\usepackage{hyperref}
\usepackage{bm}


\renewcommand{\thesubsection}{\thesection.\alph{subsection}}

\newenvironment{problem}[2][Problem]
    { \begin{mdframed}[backgroundcolor=gray!20] \textbf{#1 #2} \\}
    {  \end{mdframed}}

% Define solution environment
\newenvironment{solution}
    {\textit{Solution:}}
    {}

\renewcommand{\qed}{\quad\qedsymbol}
%%%%%%%%%%%%%%%%%%%%%%%%%%%%%%%%%%%%%%%%%%%%%%%%%%%%%%%%%%%%%%%%%%%%%%%%%%%%%%%%%%%%%%%%%%%%%%%%%%%%%%%%%%%%%%%%%%%%%%%%%%%%%%%%%%%%%%%%
\begin{document}
%Header-Make sure you update this information!!!!
\noindent
%%%%%%%%%%%%%%%%%%%%%%%%%%%%%%%%%%%%%%%%%%%%%%%%%%%%%%%%%%%%%%%%%%%%%%%%%%%%%%%%%%%%%%%%%%%%%%%%%%%%%%%%%%%%%%%%%%%%%%%%%%%%%%%%%%%%%%%%
\large\textbf{xx} \hfill \textbf{Homework - 5}   \\
Email: xxx \hfill ID: xxx \\
%\normalsize Course: Stochastic Processes \hfill Term: Spring 2020\\
%Instructor: Zheng Zhang, Wenlin Dai \hfill Due Date: $17^{th}$ March, 2020 \\
\noindent\rule{7in}{2.8pt}
%%%%%%%%%%%%%%%%%%%%%%%%%%%%%%%%%%%%%%%%%%%%%%%%%%%%%%%%%%%%%%%%%%%%%%%%%%%%%%%%%%%%%%%%%%%%%%%%%%%%%%%%%%%%%%%%%%%%%%%%%%%%%%%%%%%%%%%%
% Problem 1
%%%%%%%%%%%%%%%%%%%%%%%%%%%%%%%%%%%%%%%%%%%%%%%%%%%%%%%%%%%%%%%%%%%%%%%%%%%%%%%%%%%%%%%%%%%%%%%%%%%%%%%%%%%%%%%%%%%%%%%%%%%%%%%%%%%%%%%%
\begin{problem}{1.2}
Five white balls and five black balls are distributed in two urns in such a
way that each urn contains five balls. At each step we draw one ball from each
urn and exchange them. Let $X_n$ be the number of white balls in the left urn at
time $n$. Compute the transition probability for $X_n$.
\end{problem}
\begin{solution}
...
\end{solution}

\noindent\rule{7in}{2.8pt}
%%%%%%%%%%%%%%%%%%%%%%%%%%%%%%%%%%%%%%%%%%%%%%%%%%%%%%%%%%%%%%%%%%%%%%%%%%%%%%%%%%%%%%%%%%%%%%%%%%%%%%%%%%%%%%%%%%%%%%%%%%%%%%%%%%%%%%%%
% Problem 2
%%%%%%%%%%%%%%%%%%%%%%%%%%%%%%%%%%%%%%%%%%%%%%%%%%%%%%%%%%%%%%%%%%%%%%%%%%%%%%%%%%%%%%%%%%%%%%%%%%%%%%%%%%%%%%%%%%%%%%%%%%%%%%%%%%%%%%%%
\begin{problem}{1.4}
The 1990 census showed that 36\% of the households in the District of
Columbia were homeowners while the remainder were renters. During the next
decade 6\% of the homeowners became renters and 12\% of the renters became
homeowners. What percentage were homeowners in 2000? in 2010?
\end{problem}
\begin{solution}
...
\end{solution}

\noindent\rule{7in}{2.8pt}
%%%%%%%%%%%%%%%%%%%%%%%%%%%%%%%%%%%%%%%%%%%%%%%%%%%%%%%%%%%%%%%%%%%%%%%%%%%%%%%%%%%%%%%%%%%%%%%%%%%%%%%%%%%%%%%%%%%%%%%%%%%%%%%%%%%%%%
% Problem 3
%%%%%%%%%%%%%%%%%%%%%%%%%%%%%%%%%%%%%%%%%%%%%%%%%%%%%%%%%%%%%%%%%%%%%%%%%%%%%%%%%%%%%%%%%%%%%%%%%%%%%%%%%%%%%%%%%%%%%%%%%%%%%%%%%%%%%%%%
\begin{problem}{1.5}
Consider a gambler’s ruin chain with $N = 4$. That is, if $1\le i \le3$, $p(i, i + 1) = 0.4,\ \text{and } p(i, i - 1) = 0.6$, but the endpoints are absorbing states:
$p(0, 0) = 1$ and $p(4, 4) = 1$. Compute $p^3(1, 4)$ and $p^3(1, 0)$.
\end{problem}
\begin{solution}
...
\end{solution}

\noindent\rule{7in}{2.8pt}
%%%%%%%%%%%%%%%%%%%%%%%%%%%%%%%%%%%%%%%%%%%%%%%%%%%%%%%%%%%%%%%%%%%%%%%%%%%%%%%%%%%%%%%%%%%%%%%%%%%%%%%%%%%%%%%%%%%%%%%%%%%%%%%%%%%%%%%%
% Problem 4
%%%%%%%%%%%%%%%%%%%%%%%%%%%%%%%%%%%%%%%%%%%%%%%%%%%%%%%%%%%%%%%%%%%%%%%%%%%%%%%%%%%%%%%%%%%%%%%%%%%%%%%%%%%%%%%%%%%%%%%%%%%%%%%%%%%%%%%%
\begin{problem}{1.7}
Suppose that the probability it rains today is 0.3 if neither of the last two
days was rainy, but 0.6 if at least one of the last two days was rainy. Let the
weather on day $n$, $W_n$, be $R$ for rain, or $S$ for sun. Wn is not a Markov chain,
but the weather for the last two days $X_n = (W_n-1,W_n)$ is a Markov chain
with four states $\{RR,RS, SR, SS\}$. (a) Compute its transition probability. (b)
Compute the two-step transition probability. (c) What is the probability it will
rain on Wednesday given that it did not rain on Sunday or Monday.
\end{problem}
\begin{solution}
...
\end{solution}

\noindent\rule{7in}{2.8pt}
%%%%%%%%%%%%%%%%%%%%%%%%%%%%%%%%%%%%%%%%%%%%%%%%%%%%%%%%%%%%%%%%%%%%%%%%%%%%%%%%%%%%%%%%%%%%%%%%%%%%%%%%%%%%%%%%%%%%%%%%%%%%%%%%%%%%%%
% Problem 5
%%%%%%%%%%%%%%%%%%%%%%%%%%%%%%%%%%%%%%%%%%%%%%%%%%%%%%%%%%%%%%%%%%%%%%%%%%%%%%%%%%%%%%%%%%%%%%%%%%%%%%%%%%%%%%%%%%%%%%%%%%%%%%%%%%%%%%%%
\begin{problem}{1.9}
	Find the stationary distributions for the Markov chains with transition matrices:
	\[\begin{array}{cccccccccccc}
	(a) & 1 & 2 & 3 & (b) & 1 & 2 & 3 & (c) & 1 & 2 & 3 \\
	1 & .5 & .4 & .1 & 1 & .5 & .4 & .1 & 1 & .6 & .4 & 0 \\
	2 & .2 & .5 & .3 & 2 & .3 & .4 & .3 & 2 & .2 & .4 & .2 \\
	3 & .1 & .3 & .6 & 3 & .2 & .2 & .6 & 3 & 0 & .2 & .8
	\end{array}\]
\end{problem}
\begin{solution}
	...
\end{solution}

\noindent\rule{7in}{2.8pt}
%%%%%%%%%%%%%%%%%%%%%%%%%%%%%%%%%%%%%%%%%%%%%%%%%%%%%%%%%%%%%%%%%%%%%%%%%%%%%%%%%%%%%%%%%%%%%%%%%%%%%%%%%%%%%%%%%%%%%%%%%%%%%%%%%%%%%%
% Problem 6
%%%%%%%%%%%%%%%%%%%%%%%%%%%%%%%%%%%%%%%%%%%%%%%%%%%%%%%%%%%%%%%%%%%%%%%%%%%%%%%%%%%%%%%%%%%%%%%%%%%%%%%%%%%%%%%%%%%%%%%%%%%%%%%%%%%%%%%%
\begin{problem}{1.10}
	1.10. Find the stationary distributions for the Markov chains on $\{1,2,3,4\}$ with transition matrices:
	
	$(a)\left(\begin{array}{cccc}.7 & 0 & .3 & 0 \\ .6 & 0 & .4 & 0 \\ 0 & .5 & 0 & .5 \\ 0 & .4 & 0 & .6\end{array}\right)$
	$(b)\left(\begin{array}{cccc}.7 & .3 & 0 & 0 \\ .2 & .5 & .3 & 0 \\ .0 & .3 & .6 & .1 \\ 0 & 0 & .2 & .8\end{array}\right)$
	$(c)\left(\begin{array}{cccc}.7 & 0 & .3 & 0 \\ .2 & .5 & .3 & 0 \\ .1 & .2 & .4 & .3 \\ 0 & .4 & 0 & .6\end{array}\right)$
	
	(c) The matrix is doubly stochastic so $\pi(i)=1 / 4, i=1,2,3,4$
\end{problem}
\begin{solution}
	...
\end{solution}

\noindent\rule{7in}{2.8pt}
%%%%%%%%%%%%%%%%%%%%%%%%%%%%%%%%%%%%%%%%%%%%%%%%%%%%%%%%%%%%%%%%%%%%%%%%%%%%%
\end{document}
