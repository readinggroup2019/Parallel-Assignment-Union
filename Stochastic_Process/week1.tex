\documentclass[a4paper, 11pt]{article}
\usepackage{comment} % enables the use of multi-line comments (\ifx \fi) 
\usepackage{lipsum} %This package just generates Lorem Ipsum filler text. 
\usepackage{fullpage} % changes the margin
\usepackage[a4paper, total={7in, 10in}]{geometry}
\usepackage[fleqn]{amsmath}
\usepackage{amssymb,amsthm}  % assumes amsmath package installed
\newtheorem{theorem}{Theorem}
\newtheorem{corollary}{Corollary}
\usepackage{graphicx}
\usepackage{tikz}
\usetikzlibrary{arrows}
\usepackage{verbatim}
\usepackage[numbered]{mcode}
\usepackage{float}
\usepackage{tikz}
    \usetikzlibrary{shapes,arrows}
    \usetikzlibrary{arrows,calc,positioning}

    \tikzset{
        block/.style = {draw, rectangle,
            minimum height=1cm,
            minimum width=1.5cm},
        input/.style = {coordinate,node distance=1cm},
        output/.style = {coordinate,node distance=4cm},
        arrow/.style={draw, -latex,node distance=2cm},
        pinstyle/.style = {pin edge={latex-, black,node distance=2cm}},
        sum/.style = {draw, circle, node distance=1cm},
    }
\usepackage{xcolor}
\usepackage{mdframed}
\usepackage[shortlabels]{enumitem}
\usepackage{indentfirst}
\usepackage{hyperref}
\usepackage{bm}
    
\renewcommand{\thesubsection}{\thesection.\alph{subsection}}

\newenvironment{problem}[2][Problem]
    { \begin{mdframed}[backgroundcolor=gray!20] \textbf{#1 #2} \\}
    {  \end{mdframed}}

% Define solution environment
\newenvironment{solution}
    {\textit{Solution:}}
    {}

\renewcommand{\qed}{\quad\qedsymbol}
%%%%%%%%%%%%%%%%%%%%%%%%%%%%%%%%%%%%%%%%%%%%%%%%%%%%%%%%%%%%%%%%%%%%%%%%%%%%%%%%%%%%%%%%%%%%%%%%%%%%%%%%%%%%%%%%%%%%%%%%%%%%%%%%%%%%%%%%
\begin{document}
%Header-Make sure you update this information!!!!
\noindent
%%%%%%%%%%%%%%%%%%%%%%%%%%%%%%%%%%%%%%%%%%%%%%%%%%%%%%%%%%%%%%%%%%%%%%%%%%%%%%%%%%%%%%%%%%%%%%%%%%%%%%%%%%%%%%%%%%%%%%%%%%%%%%%%%%%%%%%%
\large\textbf{Yonghua Su} \hfill \textbf{Homework01}   \\
Email: 2019000154@ruc.edu.cn \hfill ID: 2019000154 \\
\normalsize Course: Stochastic Processes \hfill Term: Spring 2020\\
% Instructor: Dr. Sriram \hfill Due Date: $22^{nd}$ November, 2019 \\
\noindent\rule{7in}{2.8pt}
%%%%%%%%%%%%%%%%%%%%%%%%%%%%%%%%%%%%%%%%%%%%%%%%%%%%%%%%%%%%%%%%%%%%%%%%%%%%%%%%%%%%%%%%%%%%%%%%%%%%%%%%%%%%%%%%%%%%%%%%%%%%%%%%%%%%%%%%
% Problem 1.1
%%%%%%%%%%%%%%%%%%%%%%%%%%%%%%%%%%%%%%%%%%%%%%%%%%%%%%%%%%%%%%%%%%%%%%%%%%%%%%%%%%%%%%%%%%%%%%%%%%%%%%%%%%%%%%%%%%%%%%%%%%%%%%%%%%%%%%%%
\begin{problem}{1.1}
    A fair coin is tossed repeatedly with results $ Y_0, Y_1, Y_2, ...$ that are 0 or 1 with probability 1/2 each. For $n\geq1$ let
    $X_n=Y_n+Y_{n-1}$ be the number of 1's in the $(n-1)$th and $n$th tosses. Is $X_n$ a Markov chain?
\end{problem}
\begin{solution}
    ...
\end{solution}

\noindent\rule{7in}{2.8pt}
%%%%%%%%%%%%%%%%%%%%%%%%%%%%%%%%%%%%%%%%%%%%%%%%%%%%%%%%%%%%%%%%%%%%%%%%%%%%%%%%%%%%%%%%%%%%%%%%%%%%%%%%%%%%%%%%%%%%%%%%%%%%%%%%%%%%%%%%
% Problem 1.3
%%%%%%%%%%%%%%%%%%%%%%%%%%%%%%%%%%%%%%%%%%%%%%%%%%%%%%%%%%%%%%%%%%%%%%%%%%%%%%%%%%%%%%%%%%%%%%%%%%%%%%%%%%%%%%%%%%%%%%%%%%%%%%%%%%%%%%%%
\begin{problem}{1.3}
    We repeated roll two four sided dice with numbers 1, 2, 3, and 4 on them. Let $Y_k$ be the sum on the $k$th roll, $S_n = Y_1+\cdots+Y_n$ be
    the total of the first $n$ rolls, and $X_n = S_n$(mod 6). Find the transition probability for $X_n$.
\end{problem}
\begin{solution}
...
\end{solution} 
\noindent\rule{7in}{2.8pt}
%%%%%%%%%%%%%%%%%%%%%%%%%%%%%%%%%%%%%%%%%%%%%%%%%%%%%%%%%%%%%%%%%%%%%%%%%%%%%%%%%%%%%%%%%%%%%%%%%%%%%%%%%%%%%%%%%%%%%%%%%%%%%%%%%%%%%%%%
% Problem 1.6
%%%%%%%%%%%%%%%%%%%%%%%%%%%%%%%%%%%%%%%%%%%%%%%%%%%%%%%%%%%%%%%%%%%%%%%%%%%%%%%%%%%%%%%%%%%%%%%%%%%%%%%%%%%%%%%%%%%%%%%%%%%%%%%%%%%%%%%%
\begin{problem}{1.6}
    A taxicab driver moves between the airport A and two hotels B and C according to the following rules.
    If he is at the airport, he will be at one of the two hotels next with equal probability. 
    If at a hotel then he returns to the airport with probability 3/4 and goes to the other hotel with probability 1/4.
    (a) Find the transition matrix for the chain. (b) Suppose the driver begins at the airport at time 0. 
    Find the probability for each of his three possible locations at time 2 and the probability he is at hotel B at time 3.
\end{problem}
\begin{solution}
    ...
\end{solution}

\noindent\rule{7in}{2.8pt}

%%%%%%%%%%%%%%%%%%%%%%%%%%%%%%%%%%%%%%%%%%%%%%%%%%%%%%%%%%%%%%%%%%%%%%%%%%%%%%%%%%%%%%%%%%%%%%%%%%%%%%%%%%%%%%%%%%%%%%%%%%%%%%%%%%%%%%%%
% Problem 1.8
%%%%%%%%%%%%%%%%%%%%%%%%%%%%%%%%%%%%%%%%%%%%%%%%%%%%%%%%%%%%%%%%%%%%%%%%%%%%%%%%%%%%%%%%%%%%%%%%%%%%%%%%%%%%%%%%%%%%%%%%%%%%%%%%%%%%%%%%
\begin{problem}{1.8}
    Consider the following transition matrices. Identify the transient and recurrent states, 
    and the irreducible closed sets in the Markov chains. Give reasons for your answers.
	\begin{center}
	$
	\begin{matrix}
	(a) 	& \bm{1} & \bm{2} & \bm{3} & \bm{4} & \bm{5} \\ 
	\bm{1}	& .4	&.3		&.3		&0		&0		\\
	\bm{2}	& 0		&.5		&0		&.5		&0		\\
	\bm{3}	& .5	&0		&.5		&0		&0		\\
	\bm{4}	& 0		&.5		&0		&.5		&0		\\
	\bm{5}	& 0		&.3		&0		&.3		&.4			
	\end{matrix}\qquad \begin{matrix}
	(b) 	& \bm{1} & \bm{2} & \bm{3} & \bm{4} & \bm{5}  & \bm{6}\\ 
	\bm{1}	&.1		&0		&0		&.4		&.5		&0		\\
	\bm{2}	&.1		&.2		&.2		&0		&.5		&0		\\
	\bm{3}	&0		&.1		&.3		&0		&0		&.6		\\
	\bm{4}	&.1		&0		&0		&.9		&0		&0		\\
	\bm{5}	&0		&0		&0		&.4		&0		&.6		\\	
	\bm{6}	&0		&0		&0		&0		&.5		&.5
	\end{matrix}
	$
	\end{center}
	\begin{center}
	$
	\begin{matrix}
	(c) 	& \bm{1} & \bm{2} & \bm{3} & \bm{4} & \bm{5} \\ 
	\bm{1}	&0		&0		&0		&0		&1		\\
	\bm{2}	&0		&.2		&0		&.8		&0		\\
	\bm{3}	&.1		&.2		&.3		&.4		&0		\\
	\bm{4}	&0		&.6		&0		&.4		&0		\\
	\bm{5}	&.3		&0		&0		&0		&.7			
	\end{matrix}\qquad \begin{matrix}
	(d) 	& \bm{1} & \bm{2} & \bm{3} & \bm{4} & \bm{5}  & \bm{6}\\ 
	\bm{1}	&.8		&0		&0		&.2		&0		&0		\\
	\bm{2}	&0		&.5		&0		&0		&.5		&0		\\
	\bm{3}	&0		&0		&.3		&.4		&.3		&0		\\
	\bm{4}	&.1		&0		&0		&.9		&0		&0		\\
	\bm{5}	&0		&.2		&0		&0		&.8		&0		\\	
	\bm{6}	&.7		&0		&0		&.3		&0		&0
	\end{matrix}
	$
	\end{center}

\end{problem}
\begin{solution}
    ...
\end{solution}

\noindent\rule{7in}{2.8pt}

%%%%%%%%%%%%%%%%%%%%%%%%%%%%%%%%%%%%%%%%%%%%%%%%%%%%%%%%%%%%%%%%%%%%%%%%%%%%%%%%%%%%%%%%%%%%%%%%%%%%%%%%%%%%%%%%%%%%%%%%%%%%%%%%%%%%%%%%
% Problem 1.13
%%%%%%%%%%%%%%%%%%%%%%%%%%%%%%%%%%%%%%%%%%%%%%%%%%%%%%%%%%%%%%%%%%%%%%%%%%%%%%%%%%%%%%%%%%%%%%%%%%%%%%%%%%%%%%%%%%%%%%%%%%%%%%%%%%%%%%%%
\begin{problem}{1.13}
    Consider the Markov chain with transition matrix:
    \begin{center}
		$
		\begin{matrix}
		& \bm{1} & \bm{2} & \bm{3} & \bm{4} \\ 
		\bm{1}	&0		&0		&0.1	&0.9		\\
		\bm{2}	&0		&0		&0.6	&0.4		\\
		\bm{3}	&0.8	&0.2	&0		&0			\\
		\bm{4}	&0.4	&0.6	&0		&0					
		\end{matrix}
		$
	\end{center}
   (a) Compute $p^2$ . (b) Find the stationary distributions of $p$ and all of the stationary distributions of
    $p^2$ . (c) Find the limit of $p^{2n} (x, x)$ as $n \rightarrow \infty$.
\end{problem}
\begin{solution}
    ...
\end{solution}

\noindent\rule{7in}{2.8pt}
%%%%%%%%%%%%%%%%%%%%%%%%%%%%%%%%%%%%%%%%%%%%%%%%%%%%%%%%%%%%%%%%%%%%%%%%%
\end{document}
