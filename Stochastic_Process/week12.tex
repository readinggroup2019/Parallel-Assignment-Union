\documentclass[a4paper, 11pt]{article}
\usepackage{comment} % enables the use of multi-line comments (\ifx \fi) 
\usepackage{lipsum} %This package just generates Lorem Ipsum filler text. 
\usepackage{fullpage} % changes the margin
\usepackage[a4paper, total={7in, 10in}]{geometry}
\usepackage[fleqn]{amsmath}
\usepackage{amssymb,amsthm}  % assumes amsmath package installed
\newtheorem{theorem}{Theorem}
\newtheorem{corollary}{Corollary}
\usepackage{graphicx}
\usepackage{tikz}
\usetikzlibrary{arrows}
\usepackage{verbatim}
\usepackage[numbered]{mcode}
\usepackage{float}
\usepackage{diagbox} 
\newcommand{\normal}{\mathcal{N}}
\newcommand{\E}{\mathbb{E}}
\renewcommand{\P}{\mathbb{P}}
\newcommand{\Var}{\mathrm{Var}}
\newcommand{\Cov}{\mathrm{Cov}}
\usepackage{tikz}
    \usetikzlibrary{shapes,arrows}
    \usetikzlibrary{arrows,calc,positioning}

    \tikzset{
        block/.style = {draw, rectangle,
            minimum height=1cm,
            minimum width=1.5cm},
        input/.style = {coordinate,node distance=1cm},
        output/.style = {coordinate,node distance=4cm},
        arrow/.style={draw, -latex,node distance=2cm},
        pinstyle/.style = {pin edge={latex-, black,node distance=2cm}},
        sum/.style = {draw, circle, node distance=1cm},
    }
\usepackage{xcolor}
\usepackage{mdframed}
\usepackage[shortlabels]{enumitem}
\usepackage{indentfirst}
\usepackage{hyperref}
    
\renewcommand{\thesubsection}{\thesection.\alph{subsection}}

\newenvironment{problem}[2][Problem]
    { \begin{mdframed}[backgroundcolor=gray!20] \textbf{#1 #2} \\}
    {  \end{mdframed}}

% Define solution environment
\newenvironment{solution}
    {\textit{Solution:}}
    {}

\renewcommand{\qed}{\quad\qedsymbol}
%%%%%%%%%%%%%%%%%%%%%%%%%%%%%%%%%%%%%%%%%%%%%%%%%%%%%%%%%%%%%%%%%%%%%%%%%%%%%%%%%%%%%%%%%%%%%%%%%%%%%%%%%%%%%%%%%%%%%%%%%%%%%%%%%%%%%%%%
\begin{document}
%%%%%%%%%%%%%%%%%%%%%%%%%%%%%%%%%%%%%%%%%%%%%%%%%%%%%%%%%%%%%%%%%%%%%%%%%%%%%%%%%%%%%%%%%%%%%%%%%%%%%%%%%%%%%%%%%%%%%%%%%%%%%%%%%%%%%%%%
\noindent
\large\textbf{xxx} \hfill \textbf{Homework - 12}   \\
Email: xxx \hfill ID: xxx \\
\normalsize Course: Stochastic Processes \hfill Term: Spring 2020\\
Instructor: Zheng Zhang, Wenlin Dai \hfill Due Date: $21^{st}$ May, 2020 \\
\noindent\rule{7in}{2.8pt}
%%%%%%%%%%%%%%%%%%%%%%%%%%%%%%%%%%%%%%%%%%%%%%%%%%%%%%%%%%%%%%%%%%%%%%%%%%%%%%%
% Problem 1

\begin{problem}{2.4}
Find the values on the constants $a$ and $b$ that make a Gaussian process twice continuously differentiable if its covariance function is $r(t)=$ $e^{-|t|}\left(1+a|t|+b t^{2}\right)$
\end{problem}
\begin{solution}
...
\end{solution} 

\noindent\rule{7in}{2.8pt}
%%%%%%%%%%%%%%%%%%%%%%%%%%%%%%%%%%%%%%%%%%%%%%%%%%%%%%%%%%%%%%%%%%%%%%%%%
%%%%%%%%%%%%%%%%%%%%%%%%%%%%%%%%%%%%%%%%%%%%%%%%%%%%%%%%%%%%%%%%%%%%%%%%%%%%%%%


\begin{problem}{2.5}
Give the argument that a function with $r(t)=r(0)-|t|^{a}+o\left(|t|^{a}\right)$ as $t \rightarrow 0$ cannot be a covariance function if $a>2$
\end{problem}
\begin{solution}
	...
\end{solution} 

\noindent\rule{7in}{2.8pt}
%%%%%%%%%%%%%%%%%%%%%%%%%%%%%%%%%%%%%%%%%%%%%%%%%%%%%%%%%%%%%%%%%%%%%%%%%
%%%%%%%%%%%%%%%%%%%%%%%%%%%%%%%%%%%%%%%%%%%%%%%%%%%%%%%%%%%%%%%%%%%%%%%%%%%%%%%


\begin{problem}{2.6}
Complete the proof of Theorem 2.6 and show that, in the notation of the proof, $\sum_{n} g\left(2^{-n}\right)<\infty,$ and $\sum_{n} 2^{n} q\left(2^{-n}\right)<\infty$
\end{problem}
\begin{solution}
	...
\end{solution} 

\noindent\rule{7in}{2.8pt}
%%%%%%%%%%%%%%%%%%%%%%%%%%%%%%%%%%%%%%%%%%%%%%%%%%%%%%%%%%%%%%%%%%%%%%%%%
%%%%%%%%%%%%%%%%%%%%%%%%%%%%%%%%%%%%%%%%%%%%%%%%%%%%%%%%%%%%%%%%%%%%%%%%%%%%%%%


\begin{problem}{2.7}
Show that the Wiener process has infinite variation, a.s., by showing the stronger statement that if
\[
Y_{n}=\sum_{k=0}^{2^{n}-1}\left|w\left(\frac{k+1}{2^{n}}\right)-w\left(\frac{k}{2^{n}}\right)\right|
\]
then $\sum_{n=1}^{\infty} P\left(Y_{n}<n\right)<\infty$
\end{problem}
\begin{solution}
	...
\end{solution} 

\noindent\rule{7in}{2.8pt}
%%%%%%%%%%%%%%%%%%%%%%%%%%%%%%%%%%%%%%%%%%%%%%%%%%%%%%%%%%%%%%%%%%%%%%%%%
%%%%%%%%%%%%%%%%%%%%%%%%%%%%%%%%%%%%%%%%%%%%%%%%%%%%%%%%%%%%%%%%%%%%%%%%%%%%%%%


\begin{problem}{2.16}
Assume that sufficient conditions on $r(s, t)=E(x(s) \overline{x(t)})$ are satisfied so that the integral
\[
\int_{0}^{T} g(t) x(t) \mathrm{d} t
\]
exists for all $T,$ both as a quadratic mean integral and as a sample function integral. Show that, if
\[
\int_{0}^{\infty}|g(t)| \sqrt{r(t, t)} \mathrm{d} t<\infty
\]
then the generalized integral $\int_{0}^{\infty} g(t) x(t) \mathrm{d} t$ exists as a limit as $T \rightarrow \infty,$ both in quadratic mean and with probability one.
\end{problem}
\begin{solution}
	...
\end{solution} 

\noindent\rule{7in}{2.8pt}
%%%%%%%%%%%%%%%%%%%%%%%%%%%%%%%%%%%%%%%%%%%%%%%%%%%%%%%%%%%%%%%%%%%%%%%%%
%%%%%%%%%%%%%%%%%%%%%%%%%%%%%%%%%%%%%%%%%%%%%%%%%%%%%%%%%%%%%%%%%%%%%%%%%%%%%%%


\begin{problem}{2.19}
Take the trivial process $x(t)=0$ for $0 \leq t \leq 1 .$ Give an example of a sequence of processes $x_{n}(t)$ for which the finite-dimensional distributions tend to those of $x(t)$ but $\max _{[0,1]}\left(x_{n}\right)$ does not tend in distribution to 0
\end{problem}
\begin{solution}
	...
\end{solution} 

\noindent\rule{7in}{2.8pt}
%%%%%%%%%%%%%%%%%%%%%%%%%%%%%%%%%%%%%%%%%%%%%%%%%%%%%%%%%%%%%%%%%%%%%%%%%
%%%%%%%%%%%%%%%%%%%%%%%%%%%%%%%%%%%%%%%%%%%%%%%%%%%%%%%%%%%%%%%%%%%%%%%%%%%%%%%


\end{document}
