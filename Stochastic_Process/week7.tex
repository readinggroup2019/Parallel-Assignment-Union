\documentclass[a4paper, 11pt]{article}
\usepackage{comment} % enables the use of multi-line comments (\ifx \fi) 
\usepackage{lipsum} %This package just generates Lorem Ipsum filler text. 
\usepackage{fullpage} % changes the margin
\usepackage[a4paper, total={7in, 10in}]{geometry}
\usepackage[fleqn]{amsmath}
\usepackage{amssymb,amsthm}  % assumes amsmath package installed
\newtheorem{theorem}{Theorem}
\newtheorem{corollary}{Corollary}
\usepackage{graphicx}
\usepackage{tikz}
\usetikzlibrary{arrows}
\usepackage{verbatim}
\usepackage[numbered]{mcode}
\usepackage{float}
\usepackage{diagbox} 
\newcommand{\normal}{\mathcal{N}}
\newcommand{\E}{\mathbb{E}}
\renewcommand{\P}{\mathbb{P}}
\newcommand{\Var}{\mathrm{Var}}
\newcommand{\Cov}{\mathrm{Cov}}
\usepackage{tikz}
    \usetikzlibrary{shapes,arrows}
    \usetikzlibrary{arrows,calc,positioning}

    \tikzset{
        block/.style = {draw, rectangle,
            minimum height=1cm,
            minimum width=1.5cm},
        input/.style = {coordinate,node distance=1cm},
        output/.style = {coordinate,node distance=4cm},
        arrow/.style={draw, -latex,node distance=2cm},
        pinstyle/.style = {pin edge={latex-, black,node distance=2cm}},
        sum/.style = {draw, circle, node distance=1cm},
    }
\usepackage{xcolor}
\usepackage{mdframed}
\usepackage[shortlabels]{enumitem}
\usepackage{indentfirst}
\usepackage{hyperref}
    
\renewcommand{\thesubsection}{\thesection.\alph{subsection}}

\newenvironment{problem}[2][Problem]
    { \begin{mdframed}[backgroundcolor=gray!20] \textbf{#1 #2} \\}
    {  \end{mdframed}}

% Define solution environment
\newenvironment{solution}
    {\textit{Solution:}}
    {}

\renewcommand{\qed}{\quad\qedsymbol}
%%%%%%%%%%%%%%%%%%%%%%%%%%%%%%%%%%%%%%%%%%%%%%%%%%%%%%%%%%%%%%%%%%%%%%%%%%%%%%%%%%%%%%%%%%%%%%%%%%%%%%%%%%%%%%%%%%%%%%%%%%%%%%%%%%%%%%%%
\begin{document}!
%%%%%%%%%%%%%%%%%%%%%%%%%%%%%%%%%%%%%%%%%%%%%%%%%%%%%%%%%%%%%%%%%%%%%%%%%%%%%%%%%%%%%%%%%%%%%%%%%%%%%%%%%%%%%%%%%%%%%%%%%%%%%%%%%%%%%%%%
\large\textbf{...} \hfill \textbf{Homework - ...\#}   \\
Email: ... \hfill ID: ... \\
\normalsize Course: Stochastic Processes  \hfill Term: Spring 2020\\
Instructor: Dr. Zhang \& Dr. Dai \\%\hfill Due Date:\ 2020-3-17\\
\noindent\rule{7in}{2.8pt}
%%%%%%%%%%%%%%%%%%%%%%%%%%%%%%%%%%%%%%%%%%%%%%%%%%%%%%%%%%%%%%%%%%%%%%%%%%%%%%%%%%%%%%%%%%%%%%%%%%%%%%%%%%%%%%%%%%%%%%%%%%%%%%%%%%%%%%%%%%%%%%%%%%%%%%%%%%%%%%%%%%%%%%%%%%%%%%%%%%%%%%%%%%%%%%%%%%%%%%%%%%%%%%%
% Problem 1
%%%%%%%%%%%%%%%%%%%%%%%%%%%%%%%%%%%%%%%%%%%%%%%%%%%%%%%%%%%%%%%%%%%%%%%%%%%%%%%%%%%%%%%%%%%%%%%%%%%%%%%%%%%%%%%%%%%%%%%%%%%%%%%%%%%%%%%%
\begin{problem}{1.56}
A bank classifies loans as paid in full (F), in good standing (G), in arrears
(A), or as a bad debt (B). Loans move between the categories according to the following transition probability:
$$\begin{array}{lllll} 
& \mathbf{F} & \mathbf{G} & \mathbf{A} & \mathbf{B} \\
\mathbf{F} & 1 & 0 & 0 & 0 \\
\mathbf{G} & .1 & .8 & .1 & 0 \\
\mathbf{A} & .1 & .4 & .4 & .1 \\
\mathbf{B} & 0 & 0 & 0 & 1
\end{array}$$
What fraction of loans in good standing are eventually paid in full? What is the answr for those in arrears?

\end{problem}
\begin{solution}

\end{solution} 
\noindent\rule{7in}{2.8pt}

%%%%%%%%%%%%%%%%%%%%%%%%%%%%%%%%%%%%%%%%%%%%%%%%%%%%%%%%%%%%%%%%%%%%%%%%%
% Problem 2
%%%%%%%%%%%%%%%%%%%%%%%%%%%%%%%%%%%%%%%%%%%%%%%%%%%%%%%%%%%%%%%%%%%%%%%%%%%%%%%%%%%%%%%%%%%%%%%%%%%%%%%%%%%%%%%%%%%%%%%%%%%%%%%%%%%%%%%%

\begin{problem}{1.57}
A warehouse has a capacity to hold four items. If the warehouse is neither full nor empty, the number of items in the warehouse changes whenever a new item is produced or an item is sold. Suppose that (no matter when we look) the probability that the next event is "a new item is produced" is $2 / 3$ and that the new event is a "sale" is $1/3$. If there is currently one item in the warehouse, what is the probability that the warehouse will become full before it becomes empty.

\end{problem}
\begin{solution}
...
\end{solution} 

\noindent\rule{7in}{2.8pt}
%%%%%%%%%%%%%%%%%%%%%%%%%%%%%%%%%%%%%%%%%%%%%%%%%%%%%%%%%%%%%%%%%%%%%%%%%
% Problem 3
%%%%%%%%%%%%%%%%%%%%%%%%%%%%%%%%%%%%%%%%%%%%%%%%%%%%%%%%%%%%%%%%%%%%%%%%%%%%%%%%%%%%%%%%%%%%%%%%%%%%%%%%%%%%%%%%%%%%%%%%%%%%%%%%%%%%%%%%

\begin{problem}{1.58}
Six children (Dick, Helen, Joni, Mark, Sam, and Tony) play catch. If Dick has the ball he is equally likely to throw it to Helen, Mark, Sam, and Tony. If Helen has the ball she is equally likely to throw it to Dick, Joni, Sam, and Tony. If Sam has the ball he is equally likely to throw it to Dick, Helen, Mark, and Tony. If either Joni or Tony gets the ball, they keep throwing it to each other. If Mark gets the ball he runs away with it. (a) Find the transition probability and classify the states of the chain. (b) Suppose Dick has the ball at the beginning of the game. What is the probability Mark will end up with it?

\end{problem}
\begin{solution}


\end{solution} 

\noindent\rule{7in}{2.8pt}
%%%%%%%%%%%%%%%%%%%%%%%%%%%%%%%%%%%%%%%%%%%%%%%%%%%%%%%%%%%%%%%%%%%%%%%%%
% Problem 4
%%%%%%%%%%%%%%%%%%%%%%%%%%%%%%%%%%%%%%%%%%%%%%%%%%%%%%%%%%%%%%%%%%%%%%%%%%%%%%%%%%%%%%%%%%%%%%%%%%%%%%%%%%%%%%%%%%%%%%%%%%%%%%%%%%%%%%%%

\begin{problem}{1.64}
At a manufacturing plant, employees are classified as trainee (R), technician (T) or supervisor (S). Writing Q for an employee who quits we model their progress through the ranks as a Markov chain with transition probability

$$\begin{array}{lllll} 
\ & \mathbf{R} & \mathbf{T} & \mathbf{S} & \mathbf{Q} \\
\mathbf{R} & .2 & .6 & 0 & .2 \\
\mathbf{T} & 0 & .55 & .15 & .3 \\
\mathbf{S} & 0 & 0 & 1 & 0 \\
\mathbf{Q} & 0 & 0 & 0 & 1
\end{array}$$

(a) What fraction of recruits eventually make supervisor? (b) What is the expected time until a trainee auits or becomes supervisor?
\end{problem}
\begin{solution}
...
\end{solution} 

\noindent\rule{7in}{2.8pt}

\end{document}
 