\documentclass[a4paper, 11pt]{article}
\usepackage{comment} % enables the use of multi-line comments (\ifx \fi) 
\usepackage{lipsum} %This package just generates Lorem Ipsum filler text. 
\usepackage{fullpage} % changes the margin
\usepackage[a4paper, total={7in, 10in}]{geometry}
\usepackage[fleqn]{amsmath}
\usepackage{amssymb,amsthm}  % assumes amsmath package installed
\newtheorem{theorem}{Theorem}
\newtheorem{corollary}{Corollary}
\usepackage{graphicx}
\usepackage{tikz}
\usetikzlibrary{arrows}
\usepackage{verbatim}
\usepackage[numbered]{mcode}
\usepackage{float}
\usepackage{tikz}
    \usetikzlibrary{shapes,arrows}
    \usetikzlibrary{arrows,calc,positioning}

    \tikzset{
        block/.style = {draw, rectangle,
            minimum height=1cm,
            minimum width=1.5cm},
        input/.style = {coordinate,node distance=1cm},
        output/.style = {coordinate,node distance=4cm},
        arrow/.style={draw, -latex,node distance=2cm},
        pinstyle/.style = {pin edge={latex-, black,node distance=2cm}},
        sum/.style = {draw, circle, node distance=1cm},
    }
\usepackage{xcolor}
\usepackage{mdframed}
\usepackage[shortlabels]{enumitem}
\usepackage{indentfirst}
\usepackage{hyperref}
    
\renewcommand{\thesubsection}{\thesection.\alph{subsection}}

\newenvironment{problem}[2][Problem]
    { \begin{mdframed}[backgroundcolor=gray!20] \textbf{#1 #2} \\}
    {  \end{mdframed}}

% Define solution environment
\newenvironment{solution}
    {\textit{Solution:}}
    {}

\renewcommand{\qed}{\quad\qedsymbol}
%%%%%%%%%%%%%%%%%%%%%%%%%%%%%%%%%%%%%%%%%%%%%%%%%%%%%%%%%%%%%%%%%%%%%%%%%%%%%%%%%%%%%%%%%%%%%%%%%%%%%%%%%%%%%%%%%%%%%%%%%%%%%%%%%%%%%%%%
\begin{document}
%Header-Make sure you update this information!!!!
\noindent
%%%%%%%%%%%%%%%%%%%%%%%%%%%%%%%%%%%%%%%%%%%%%%%%%%%%%%%%%%%%%%%%%%%%%%%%%%%%%%%%%%%%%%%%%%%%%%%%%%%%%%%%%%%%%%%%%%%%%%%%%%%%%%%%%%%%%%%%
\large\textbf{hhhhhhh} \hfill \textbf{Homework - 6}   \\
Email: ????? \hfill ID:2019000xxx \\
\normalsize Course:Stochastic Processes \hfill Term: Spring 2020\\
%Instructor: Dr. He \hfill Due Date: $8^{th}$ Mar., 2020 \\
\noindent\rule{7in}{2.8pt}
%%%%%%%%%%%%%%%%%%%%%%%%%%%%%%%%%%%%%%%%%%%%%%%%%%%%%%%%%%%%%%%%%%%%%%%%%%%%%%%%%%%%%%%%%%%%%%%%%%%%%%%%%%%%%%%%%%%%%%%%%%%%%%%%%%%%%%%%
% Problem 1
%%%%%%%%%%%%%%%%%%%%%%%%%%%%%%%%%%%%%%%%%%%%%%%%%%%%%%%%%%%%%%%%%%%%%%%%%%%%%%%%%%%%%%%%%%%%%%%%%%%%%%%%%%%%%%%%%%%%%%%%%%%%%%%%%%%%%%%%
\begin{problem}{1.15}
Find $\lim_{n\rightarrow \infty}p^n(i,j)$ for

 $$
\begin{array}{rrrrrr} & \mathbf{1} & \mathbf{2} & \mathbf{3} & \mathbf{4} & \mathbf{5} \\ \mathbf{1} & 1 & 0 & 0 & 0 & 0 \\ \mathbf{2} & 0 & 2 / 3 & 0 & 1 / 3 & 0 \\ p=\mathbf{3} & 1 / 8 & 1 / 4 & 5 / 8 & 0 & 0 \\ \mathbf{4} & 0 & 1 / 6 & 0 & 5 / 6 & 0 \\ \mathbf{5} & 1 / 3 & 0 & 1 / 3 & 0 & 1 / 3\end{array}
$$

You are supposed to do this and the next problem by solving equations. How-
ever you can check your answers by using your calculator to find FRAC($p^{100}$).

\end{problem}
\begin{solution}

\end{solution} 
\noindent\rule{7in}{2.8pt}

%%%%%%%%%%%%%%%%%%%%%%%%%%%%%%%%%%%%%%%%%%%%%%%%%%%%%%%%%%%%%%%%%%%%%%%%%
% Problem 2
%%%%%%%%%%%%%%%%%%%%%%%%%%%%%%%%%%%%%%%%%%%%%%%%%%%%%%%%%%%%%%%%%%%%%%%%%%%%%%%%%%%%%%%%%%%%%%%%%%%%%%%%%%%%%%%%%%%%%%%%%%%%%%%%%%%%%%%%

\begin{problem}{1.16}

If we rearrange the matrix for the seven state chain in Example 1.14 we
get
$$
\begin{array}{c|cc|cc|ccc} & \mathbf{2} & \mathbf{3} & \mathbf{1} & \mathbf{5} & \mathbf{4} & \mathbf{6} & \mathbf{7} \\ \hline \mathbf{2} & .2 & .3 & .1 & 0 & .4 & 0 & 0 \\ \mathbf{3} & 0 & .5 & 0 & .2 & .3 & 0 & 0 \\ \hline \mathbf{1} & 0 & 0 & .7 & .3 & 0 & 0 & 0 \\ \mathbf{5} & 0 & 0 & .6 & .4 & 0 & 0 & 0 \\ \hline \mathbf{4} & 0 & 0 & 0 & 0 & .5 & .5 & 0 \\ \mathbf{6} & 0 & 0 & 0 & 0 & 0 & .2 & .8 \\ \mathbf{7} & 0 & 0 & 0 & 0 & 1 & 0 & 0\end{array}
$$

Find $\lim_{n\rightarrow \infty}p^n(i,j)$ .




\end{problem}
\begin{solution}

\end{solution} 

\noindent\rule{7in}{2.8pt}
%%%%%%%%%%%%%%%%%%%%%%%%%%%%%%%%%%%%%%%%%%%%%%%%%%%%%%%%%%%%%%%%%%%%%%%%%
% Problem 3
%%%%%%%%%%%%%%%%%%%%%%%%%%%%%%%%%%%%%%%%%%%%%%%%%%%%%%%%%%%%%%%%%%%%%%%%%%%%%%%%%%%%%%%%%%%%%%%%%%%%%%%%%%%%%%%%%%%%%%%%%%%%%%%%%%%%%%%%

\begin{problem}{1.18}

A sociology professor postulates that in each decade 8$\%$ of women in the work force leave it and 20$\%$ of the women not in it begin to work. Compare the predictions of his model with the following data on the percentage of women working: 43.3$\%$ in 1970, 51.5$\%$ in 1980, 57.5$\%$ in 1990, and 59.8$\%$ in 2000. In the long run what fraction of women will be working?

\end{problem}
\begin{solution}

\end{solution} 

\noindent\rule{7in}{2.8pt}
%%%%%%%%%%%%%%%%%%%%%%%%%%%%%%%%%%%%%%%%%%%%%%%%%%%%%%%%%%%%%%%%%%%%%%%%%
% Problem 4
%%%%%%%%%%%%%%%%%%%%%%%%%%%%%%%%%%%%%%%%%%%%%%%%%%%%%%%%%%%%%%%%%%%%%%%%%%%%%%%%%%%%%%%%%%%%%%%%%%%%%%%%%%%%%%%%%%%%%%%%%%%%%%%%%%%%%%%%

\begin{problem}{1.19}
A rapid transit system has just started operating. In the first month of
operation, it was found that 25$\%$  of commuters are using the system while 75$\%$ 
are travelling by automobile. Suppose that each month 10$\%$  of transit users go
back to using their cars, while 30$\%$  of automobile users switch to the transit
system. (a) Compute the three step transition probaiblity $p^3$ . (b) What will be the fractions using rapid transit in the fourth month? (c) In the long run?

\end{problem}
\begin{solution}

\end{solution} 

\noindent\rule{7in}{2.8pt}
%%%%%%%%%%%%%%%%%%%%%%%%%%%%%%%%%%%%%%%%%%%%%%%%%%%%%%%%%%%%%%%%%%%%%%%%%
% Problem 5
%%%%%%%%%%%%%%%%%%%%%%%%%%%%%%%%%%%%%%%%%%%%%%%%%%%%%%%%%%%%%%%%%%%%%%%%%%%%%%%%%%%%%%%%%%%%%%%%%%%%%%%%%%%%%%%%%%%%%%%%%%%%%%%%%%%%%%%%

\begin{problem}{1.20}
A regional health study indicates that from one year to the next, 75$\%$ percent of smokers will continue to smoke while 25$\%$ will quit. 8$\%$ of those who stopped smoking will resume smoking while 92$\%$ will not. If 70$\%$ of the population were smokers in 1995, what fraction will be smokers in 1998? in 2005? in the long run?

\end{problem}
\begin{solution}

\end{solution} 

\noindent\rule{7in}{2.8pt}
%%%%%%%%%%%%%%%%%%%%%%%%%%%%%%%%%%%%%%%%%%%%%%%%%%%%%%%%%%%%%%%%%%%%%%%%%
% Problem 6
%%%%%%%%%%%%%%%%%%%%%%%%%%%%%%%%%%%%%%%%%%%%%%%%%%%%%%%%%%%%%%%%%%%%%%%%%%%%%%%%%%%%%%%%%%%%%%%%%%%%%%%%%%%%%%%%%%%%%%%%%%%%%%%%%%%%%%%%

\begin{problem}{1.21}
Three of every four trucks on the road are followed by a car, while only one of every five cars is followed by a truck. What fraction of vehicles on the road are trucks?

\end{problem}
\begin{solution}

\end{solution} 

\noindent\rule{7in}{2.8pt}
%%%%%%%%%%%%%%%%%%%%%%%%%%%%%%%%%%%%%%%%%%%%%%%%%%%%%%%%%%%%%%%%%%%%%%%%%

%%%%%%%%%%%%%%%%%%%%%%%%%%%%%%%%%%%%%%%%%%%%%%%%%%%%%%%%%%%%%%%%%%%%%%%%%
\end{document}
 