\documentclass[a4paper, 11pt]{article}
\usepackage{comment} % enables the use of multi-line comments (\ifx \fi)
\usepackage{lipsum} %This package just generates Lorem Ipsum filler text.
\usepackage{fullpage} % changes the margin
\usepackage[a4paper, total={7in, 10in}]{geometry}
\usepackage[fleqn]{amsmath}
\usepackage{amssymb,amsthm}  % assumes amsmath package installed
\newtheorem{theorem}{Theorem}
\newtheorem{corollary}{Corollary}
\usepackage{graphicx}
\usepackage{tikz}
\usetikzlibrary{arrows}
\usepackage{verbatim}
\usepackage[numbered]{mcode}
\usepackage{float}
\usepackage{tikz}
    \usetikzlibrary{shapes,arrows}
    \usetikzlibrary{arrows,calc,positioning}

    \tikzset{
        block/.style = {draw, rectangle,
            minimum height=1cm,
            minimum width=1.5cm},
        input/.style = {coordinate,node distance=1cm},
        output/.style = {coordinate,node distance=4cm},
        arrow/.style={draw, -latex,node distance=2cm},
        pinstyle/.style = {pin edge={latex-, black,node distance=2cm}},
        sum/.style = {draw, circle, node distance=1cm},
    }
\usepackage{xcolor}
\usepackage{mdframed}
\usepackage[shortlabels]{enumitem}
\usepackage{indentfirst}
\usepackage{hyperref}
\usepackage{bm}

\renewcommand{\thesubsection}{\thesection.\alph{subsection}}

\newenvironment{problem}[2][Problem]
    { \begin{mdframed}[backgroundcolor=gray!20] \textbf{#1 #2} \\}
    {  \end{mdframed}}

% Define solution environment
\newenvironment{solution}
    {\textit{Solution:}}
    {}

\renewcommand{\qed}{\quad\qedsymbol}
%%%%%%%%%%%%%%%%%%%%%%%%%%%%%%%%%%%%%%%%%%%%%%%%%%%%%%%%%%%%%%%%%%%%%%%%%%%%%%%%%%%%%%%%%%%%%%%%%%%%%%%%%%%%%%%%%%%%%%%%%%%%%%%%%%%%%%%%
\begin{document}
%Header-Make sure you update this information!!!!
\noindent
%%%%%%%%%%%%%%%%%%%%%%%%%%%%%%%%%%%%%%%%%%%%%%%%%%%%%%%%%%%%%%%%%%%%%%%%%%%%%%%%%%%%%%%%%%%%%%%%%%%%%%%%%%%%%%%%%%%%%%%%%%%%%%%%%%%%%%%%
\large\textbf{Tao Li} \hfill \textbf{Homework02}   \\
Email: 2019000153lt@ruc.edu.cn \hfill ID: 2019000153 \\
\normalsize Course: Stochastic Processes \hfill Term: Spring 2020\\
% Instructor: Dr. Sriram \hfill Due Date: $22^{nd}$ November, 2019 \\
\noindent\rule{7in}{2.8pt}
%%%%%%%%%%%%%%%%%%%%%%%%%%%%%%%%%%%%%%%%%%%%%%%%%%%%%%%%%%%%%%%%%%%%%%%%%%%%%%%%%%%%%%%%%%%%%%%%%%%%%%%%%%%%%%%%%%%%%%%%%%%%%%%%%%%%%%%%
% Problem 1.14
%%%%%%%%%%%%%%%%%%%%%%%%%%%%%%%%%%%%%%%%%%%%%%%%%%%%%%%%%%%%%%%%%%%%%%%%%%%%%%%%%%%%%%%%%%%%%%%%%%%%%%%%%%%%%%%%%%%%%%%%%%%%%%%%%%%%%%%%
\begin{problem}{1.14}
    Do the following Markov chains converge to equilibrium?
    \begin{center}
        $
         \begin{matrix}
            (a) & \bm{1} & \bm{2} & \bm{3} & \bm{4} \\
            \bm{1} & 0 & 0 & 1 & 0 \\
            \bm{2} & 0 & 0 & .5 & .5 \\
            \bm{3} & .3 & .7 & 0 & 0\\
            \bm{4} & 1 & 0 & 0 & 0
        \end{matrix}
        \qquad
        \begin{matrix}
            (a) & \bm{1} & \bm{2} & \bm{3} & \bm{4} \\
            \bm{1} & 0 & 1 & .0 & 0 \\
            \bm{2} & 0 & 0 & 0 & 1 \\
            \bm{3} & 1 & 0 & 0 & 0\\
            \bm{4} & 1/3 & 0 & 2/3 & 0
        \end{matrix}
        $
    \end{center}

    \begin{center}
        $       
       \begin{matrix}
           (c) & \bm{1} & \bm{2} & \bm{3} & \bm{4} & \bm{5} & \bm{6}\\
           \bm{1} & 1 & .5 & .5 & 0 & 0 & 0\\
           \bm{2} & 0 & 0 & 0 & 1 & 0 & 0\\
           \bm{3} & 0 & 0 & 0 & .4 & 0 & .6\\
           \bm{4} & 1 & 0 & 0 & 0 & 0 & 0\\
           \bm{5} & 0 & 1 & 0 & 0 & 0 & 0\\
           \bm{6} & .2 & 0 & 0 & 0 & .8 & 0
           \end{matrix}
           $
   \end{center}
\end{problem}
\begin{solution}   
...
\end{solution}

\noindent\rule{7in}{2.8pt}
%%%%%%%%%%%%%%%%%%%%%%%%%%%%%%%%%%%%%%%%%%%%%%%%%%%%%%%%%%%%%%%%%%%%%%%%%%%%%%%%%%%%%%%%%%%%%%%%%%%%%%%%%%%%%%%%%%%%%%%%%%%%%%%%%%%%%%%%
% Problem 1.37
%%%%%%%%%%%%%%%%%%%%%%%%%%%%%%%%%%%%%%%%%%%%%%%%%%%%%%%%%%%%%%%%%%%%%%%%%%%%%%%%%%%%%%%%%%%%%%%%%%%%%%%%%%%%%%%%%%%%%%%%%%%%%%%%%%%%%%%%
\begin{problem}{1.37}
    An individual has three umbrellas, some at her office, and some at home. 
    If she is leaving home in the morning (or leaving work at night) and it is raining, she will take an umbrella, 
    if one is there.Otherwise, she gets wet. Assume that independent of the past, it rains on each trip with probability 0.2.
    To formulate a Markov chain,let $X_n$ be the number of umbrellas at her current location. 
    (a) Find the transition probability for this Markov chain.(b) Calculate the limiting fraction of time she gets wet.
\end{problem}
\begin{solution}
...
\end{solution}

\noindent\rule{7in}{2.8pt}
%%%%%%%%%%%%%%%%%%%%%%%%%%%%%%%%%%%%%%%%%%%%%%%%%%%%%%%%%%%%%%%%%%%%%%%%%%%%%%%%%%%%%%%%%%%%%%%%%%%%%%%%%%%%%%%%%%%%%%%%%%%%%%%%%%%%%%%%
% Problem 1.41
%%%%%%%%%%%%%%%%%%%%%%%%%%%%%%%%%%%%%%%%%%%%%%%%%%%%%%%%%%%%%%%%%%%%%%%%%%%%%%%%%%%%%%%%%%%%%%%%%%%%%%%%%%%%%%%%%%%%%%%%%%%%%%%%%%%%%%%%
\begin{problem}{1.41}
    \textit{Reflecting random walk on the line} .Consider the points 1, 2, 3, 4 to be marked on a straight line. 
    Let $X_n$ be a Markov chain that moves to the right with probability 2/3 and to the left with probability 1/3,
    but subject this time to the rule that if $X_n$ tries to go to the left from 1 or to the right from 4 it stays put.
    Find (a) the transition probability for the chain, and (b) the limiting amount of time the chain spends at each site. 
\end{problem}
\begin{solution}
...
\end{solution}

\noindent\rule{7in}{2.8pt}
%%%%%%%%%%%%%%%%%%%%%%%%%%%%%%%%%%%%%%%%%%%%%%%%%%%%%%%%%%%%%%%%%%%%%%%%%%%%%%%%%%%%%%%%%%%%%%%%%%%%%%%%%%%%%%%%%%%%%%%%%%%%%%%%%%%%%%%%
% Problem 1.44
%%%%%%%%%%%%%%%%%%%%%%%%%%%%%%%%%%%%%%%%%%%%%%%%%%%%%%%%%%%%%%%%%%%%%%%%%%%%%%%%%%%%%%%%%%%%%%%%%%%%%%%%%%%%%%%%%%%%%%%%%%%%%%%%%%%%%%%%
\begin{problem}{1.44}
    \textit{Landscape dynamics} .To make a crude model of a forest we might introduce states 0 = grass, 1 = bushes, 
    2 = small trees, 3 = large trees, and write down a transition matrix like the following:
    \begin{center}
        $       
       \begin{matrix}
             & \bm{0} & \bm{1} & \bm{2} & \bm{3} \\
           \bm{0} & 1/2 & 1/2 & 0 & 0 \\
           \bm{1} & 1/24 & 7/8 & 1/12 & 0 \\
           \bm{2} & 1/36 & 0 & 8/9 & 1/12 \\
           \bm{3} & 1/8 & 0 & 0 & 7/8 
           \end{matrix}
           $
   \end{center}
   The idea behind this matrix is that if left undisturbed a grassy area 
   will see bushes grow, then small trees, which of course grow into large trees. 
   However, disturbances such as tree falls or fires can reset the system to state 0. 
   Find the limiting fraction of land in each of the states.
\end{problem}
\begin{solution}
...
\end{solution}

\noindent\rule{7in}{2.8pt}
%%%%%%%%%%%%%%%%%%%%%%%%%%%%%%%%%%%%%%%%%%%%%%%%%%%%%%%%%%%%%%%%%%%%%%%%%%%%%%%%%%%%%%%%%%%%%%%%%%%%%%%%%%%%%%%%%%%%%%%%%%%%%%%%%%%%%%%%
% Problem 1.45
%%%%%%%%%%%%%%%%%%%%%%%%%%%%%%%%%%%%%%%%%%%%%%%%%%%%%%%%%%%%%%%%%%%%%%%%%%%%%%%%%%%%%%%%%%%%%%%%%%%%%%%%%%%%%%%%%%%%%%%%%%%%%%%%%%%%%%%%
\begin{problem}{1.45}
    Consider a general chain with state space $S$ = {1, 2} and write the transition probability as
    \begin{center}
        $       
       \begin{matrix}
             & \bm{1} & \bm{2} \\
           \bm{1} & 1-a & a \\
           \bm{2} & b & 1-b 
           \end{matrix}
           $
   \end{center}
   Use the Markov property to show that
   $$P\left(X_{n+1}=1\right)-\frac{b}{a+b}=(1-a-b)\left\{P\left(X_{n}=1\right)-\frac{b}{a+b}\right\}$$
   and then conclude
   $$P\left(X_{n}=1\right)=\frac{b}{a+b}+(1-a-b)^n\left\{P\left(X_{0}=1\right)-\frac{b}{a+b}\right\}$$
   This shows that if $0 < a + b < 2$,then $P(X_n = 1)$ converges exponentially fast to its limiting value $b/(a + b)$.
\end{problem}
\begin{solution}
...
\end{solution}

\noindent\rule{7in}{2.8pt}
%%%%%%%%%%%%%%%%%%%%%%%%%%%%%%%%%%%%%%%%%%%%%%%%%%%%%%%%%%%%%%%%%%%%%%%%%%%%%%%%%%%%%%%%%%%%%%%%%%%%%%%%%%%%%%%%%%%%%%%%%%%%%%%%%%%%%%%%
% Problem 1.46
%%%%%%%%%%%%%%%%%%%%%%%%%%%%%%%%%%%%%%%%%%%%%%%%%%%%%%%%%%%%%%%%%%%%%%%%%%%%%%%%%%%%%%%%%%%%%%%%%%%%%%%%%%%%%%%%%%%%%%%%%%%%%%%%%%%%%%%%
\begin{problem}{1.46}
    \textit{Bernoulli–Laplace model of diffusion}. Consider two urns each of which contains $m$ balls;
     $b$ of these $2m$ balls are black, and the remaining $2m − b$ are white. 
     We say that the system is in state $i$ if the first urn contains $i$ black balls and $m − i$ white balls 
     while the second contains $b − i$ black balls and $m − b + i$ white balls. 
     Each trial consists of choosing a ball at random from each urn and exchanging the two. 
     Let $X_n$ be the state of the system after $n$ exchanges have been made. $X_n$ is a Markov chain.
     (a) Compute its transition probability. (b) Verify that the stationary distribution is given by
    $$
        \pi(i)=\left. \binom{b}{i}\binom{2m-b}{m-i} \middle/ \binom{2m}{m} \right.
    $$
    (c) Can you give a simple intuitive explanation why the formula in (b) gives the right answer?
\end{problem}
\begin{solution}
...
\end{solution}

\noindent\rule{7in}{2.8pt}
%%%%%%%%%%%%%%%%%%%%%%%%%%%%%%%%%%%%%%%%%%%%%%%%%%%%%%%%%%%%%%%%%%%%%%%%%%%%%%%%%%%%%%%%%%%%%%%%%%%%%%%%%%%%%%%%%%%%%%%%%%%%%%%%%%%%%%%%
% Problem 1.47
%%%%%%%%%%%%%%%%%%%%%%%%%%%%%%%%%%%%%%%%%%%%%%%%%%%%%%%%%%%%%%%%%%%%%%%%%%%%%%%%%%%%%%%%%%%%%%%%%%%%%%%%%%%%%%%%%%%%%%%%%%%%%%%%%%%%%%%%
\begin{problem}{1.47}
    \textit{Library chain}.On each request the $i$th of $n$ possible books is the one chosen with probability $p_i$. 
    To make it quicker to find the book the next time, the librarian moves the book to the left end of the shelf.
    Define the state at any time to be the sequence of books we see as we examine the shelf from left to right. 
    Since all the books are distinct this list is a permutation of the set $\{1, 2, . . . n\}$, i.e., each number is listed exactly once.
    Show that
    $$
\pi(i_1, \ldots, i_n)=p_{i_1} \cdot \frac{p_{i_2}}{1-p_{i_1}} \cdot \frac{p_{i_3}}{1-p_{i_1}-p_{i_2}} \cdots \frac{p_{i_n}}{1-p_{i_1}-\cdots p_{i_{n-1}}}
$$
is a stationary distribution.
\end{problem}
\begin{solution}
...
\end{solution}

\noindent\rule{7in}{2.8pt}
%%%%%%%%%%%%%%%%%%%%%%%%%%%%%%%%%%%%%%%%%%%%%%%%%%%%%%%%%%%%%%%%%%%%%%%%%
\end{document}
