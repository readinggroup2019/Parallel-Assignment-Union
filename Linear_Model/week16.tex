\documentclass[a4paper, 11pt]{article}
\usepackage{comment} % enables the use of multi-line comments (\ifx \fi) 
\usepackage{lipsum} %This package just generates Lorem Ipsum filler text. 
\usepackage{fullpage} % changes the margin
\usepackage[a4paper, total={7in, 10in}]{geometry}
\usepackage[fleqn]{amsmath}
\usepackage{amssymb,amsthm}  % assumes amsmath package installed
\newtheorem{theorem}{Theorem}
\newtheorem{corollary}{Corollary}
\usepackage{graphicx}
\usepackage{tikz}
\usetikzlibrary{arrows}
\usepackage{verbatim}

\newcommand{\E}{\mathbb{E}}
\newcommand{\R}{\mathbb{R}}
\renewcommand{\P}{\mathbb{P}}
\newcommand{\normal}{\mathcal{N}}
\newcommand{\Var}{\mathrm{Var}}
\newcommand{\Cov}{\mathrm{Cov}}

\usepackage[numbered]{mcode}
\usepackage{float}
\usepackage{tikz}
    \usetikzlibrary{shapes,arrows}
    \usetikzlibrary{arrows,calc,positioning}

    \tikzset{
        block/.style = {draw, rectangle,
            minimum height=1cm,
            minimum width=1.5cm},
        input/.style = {coordinate,node distance=1cm},
        output/.style = {coordinate,node distance=4cm},
        arrow/.style={draw, -latex,node distance=2cm},
        pinstyle/.style = {pin edge={latex-, black,node distance=2cm}},
        sum/.style = {draw, circle, node distance=1cm},
    }
\usepackage{xcolor}
\usepackage{mdframed}
\usepackage[shortlabels]{enumitem}
\usepackage{indentfirst}
\usepackage{hyperref}
    
\renewcommand{\thesubsection}{\thesection.\alph{subsection}}

\newenvironment{problem}[2][Problem]
    { \begin{mdframed}[backgroundcolor=gray!20] \textbf{#1 #2} \\}
    {  \end{mdframed}}

% Define solution environment
\newenvironment{solution}
    {\textit{Solution:}}
    {}

\renewcommand{\qed}{\quad\qedsymbol}
%%%%%%%%%%%%%%%%%%%%%%%%%%%%%%%%%%%%%%%%%%%%%%%%%%%%%%%%%%%%%%%%%%%%%%%%%%%%%%%%%%%%%%%%%%%%%%%%%%%%%%%%%%%%%%%%%%%%%%%%%%%%%%%%%%%%%%%%
\begin{document}
%Header-Make sure you update this information!!!!
\noindent
%%%%%%%%%%%%%%%%%%%%%%%%%%%%%%%%%%%%%%%%%%%%%%%%%%%%%%%%%%%%%%%%%%%%%%%%%%%%%%%%%%%%%%%%%%%%%%%%%%%%%%%%%%%%%%%%%%%%%%%%%%%%%%%%%%%%%%%%
\large\textbf{XXXX} \hfill \textbf{Homework 16}   \\
Email: *****@ruc.edu.cn \hfill ID: 201900015* \\
\normalsize Course: Linear Model   \hfill Term: Spring 2020\\
\noindent\rule{7in}{2.8pt}


%%%%%%%%%%%%%%%%%%%%%%%%%%%%%%%%%%%%%%%%%%%%%%%%%%%%%%%%%%%%%%%%%%%%%%%%%%%%%%%%%%%%%%%%%%%%%%%%%%%%%%%%%%%%%%%%%%%%%%%%%%%%%%%%%%%%%%%%
% Problem1
%%%%%%%%%%%%%%%%%%%%%%%%%%%%%%%%%%%%%%%%%%%%%%%%%%%%%%%%%%%%%%%%%%%%%%%%%%%%%%%%%%%%%%%%%%%%%%%%%%%%%%%%%%%%%%%%%%%%%%%%%%%%%%%%%%%%%%%%
\begin{problem}{10.4.1}
Data $y_{1}, \ldots, y_{n}$ are assumed to follow a binary logistic model in which $y_{j}$ takes value 1 with probability $\pi_{j}=\exp \left(x_{j}^{\mathrm{T}} \beta\right) /\left\{1+\exp \left(x_{j}^{\mathrm{T}} \beta\right)\right\}$ and value 0 otherwise, for $j=1, \ldots, n$.

(a) Show that the deviance for a model with fitted probabilities $\widehat{\pi}_{j}$ can be written as
\[
D=-2\left\{y^{\mathrm{T}} X \widehat{\beta}+\sum_{j=1}^{n} \log \left(1-\widehat{\pi}_{j}\right)\right\}
\]
and that the likelihood equation is $X^{\mathrm{T}} y=X^{\mathrm{T}} \widehat{\pi} .$ Hence show that the deviance is a function of the $\widehat{\pi}_{j}$ alone.

(b) If $\pi_{1}=\cdots=\pi_{n}=\pi,$ then show that $\widehat{\pi}=\bar{y},$ and verify that
\[
D=-2 n\{\bar{y} \log \bar{y}+(1-\bar{y}) \log (1-\bar{y})\}
\]
Comment on the implications for using $D$ to measure the discrepancy between the data and fitted model.

(c) $\operatorname{In}(\mathrm{b}),$ show that Pearson's statistic (10.21) is identically equal to $n .$ Comment.

\end{problem}
\begin{solution}
	
	\textit{Note: This is also the homework in last week.}
	
\end{solution}

\noindent\rule{7in}{2.8pt}

%%%%%%%%%%%%%%%%%%%%%%%%%%%%%%%%%%%%%%%%%%%%%%%%%%%%%%
% Problem2
%%%%%%%%%%%%%%%%%%%%%%%%%%%%%%%%%%%%%%%%%%%%%%%%%%%%%%%%%%%%%%%%%%%%%%%%%%%%%%%%%%%%%%%%%%%%%%%%%%%%%%%%%%%%%%%%%%%%%%%%%%%%%%%%%%%%%%%%
\begin{problem}{10.4.2}
	(a) Show that the parametric link function
	$$g(\pi ; \gamma)=\log \left[\gamma^{-1}\left\{(1-\pi)^{-\gamma}-1\right\}\right], \quad \gamma \neq 0$$
	gives the logit and complementary log-log links when $\gamma=1$ and when $\gamma \rightarrow 0$.
	Give a similar function containing the logit and log-log link functions.
	(b) Show that the link function
	$$g(\pi ; \gamma)=\log \left[\gamma^{-1}\left\{(1-\pi)^{-\gamma}-1\right\}\right], \quad \gamma \neq 0$$
	is symmetric for all γ and gives the logit and identity functions when $\gamma \rightarrow 0$ and when $\gamma=1$.
\end{problem}
\begin{solution}


\end{solution}

\noindent\rule{7in}{2.8pt}
%%%%%%%%%%%%%%%%%%%%%%%%%%%%%%%%%%%%%%%%%%%%%%%%%%%%%%
% Problem3
%%%%%%%%%%%%%%%%%%%%%%%%%%%%%%%%%%%%%%%%%%%%%%%%%%%%%%%%%%%%%%%%%%%%%%%%%%%%%%%%%%%%%%%%%%%%%%%%%%%%%%%%%%%%%%%%%%%%%%%%%%%%%%%%%%%%%%%%
\begin{problem}{10.4.3}
	If $X$ is a Poisson variable with mean $\mu=\exp \left(x^{\mathrm{T}} \beta\right)$ and $Y$ is a binary variable indicating
	the event $X > 0$, find the link function between $\mathrm{E}(Y)$ and $x^{\mathrm{T}} \beta$.

\end{problem}
\begin{solution}
	
\end{solution}

\noindent\rule{7in}{2.8pt}
%%%%%%%%%%%%%%%%%%%%%%%%%%%%%%%%%%%%%%%%%%%%%%%%%%%%%%
% Problem3
%%%%%%%%%%%%%%%%%%%%%%%%%%%%%%%%%%%%%%%%%%%%%%%%%%%%%%%%%%%%%%%%%%%%%%%%%%%%%%%%%%%%%%%%%%%%%%%%%%%%%%%%%%%%%%%%%%%%%%%%%%%%%%%%%%%%%%%%
\begin{problem}{Extra}
Reproduce the right panel of Figure 10.6 in Example 10.17 and think about what will happen if the range of the X-axis goes beyond 12. Also, generate a figure where the X-axis is and the Y-axis is the ratio .

\end{problem}
\begin{solution}

\end{solution}

\noindent\rule{7in}{2.8pt}




\end{document}
