\documentclass[a4paper, 11pt]{article}
\usepackage{comment} % enables the use of multi-line comments (\ifx \fi)
\usepackage{lipsum} %This package just generates Lorem Ipsum filler text.
\usepackage{fullpage} % changes the margin
\usepackage[a4paper, total={7in, 10in}]{geometry}
\usepackage[fleqn]{amsmath}
\usepackage{amssymb,amsthm}  % assumes amsmath package installed
\newtheorem{theorem}{Theorem}
\newtheorem{corollary}{Corollary}
\usepackage{graphicx}
\usepackage{tikz}
\usetikzlibrary{arrows}
\usepackage{verbatim}
\usepackage{pdfpages}
\usepackage[numbered]{mcode}
\usepackage{float}
\usepackage{tikz}
    \usetikzlibrary{shapes,arrows}
    \usetikzlibrary{arrows,calc,positioning}

    \tikzset{
        block/.style = {draw, rectangle,
            minimum height=1cm,
            minimum width=1.5cm},
        input/.style = {coordinate,node distance=1cm},
        output/.style = {coordinate,node distance=4cm},
        arrow/.style={draw, -latex,node distance=2cm},
        pinstyle/.style = {pin edge={latex-, black,node distance=2cm}},
        sum/.style = {draw, circle, node distance=1cm},
    }
\usepackage{xcolor}
\usepackage{mdframed}
\usepackage[shortlabels]{enumitem}
\usepackage{indentfirst}
\usepackage{hyperref}

\renewcommand{\thesubsection}{\thesection.\alph{subsection}}

\newenvironment{problem}[2][Problem]
    { \begin{mdframed}[backgroundcolor=gray!20] \textbf{#1 #2} \\}
    {  \end{mdframed}}

% Define solution environment
\newenvironment{solution}
    {\textit{Solution:}}
    {}

\renewcommand{\qed}{\quad\qedsymbol}
%%%%%%%%%%%%%%%%%%%%%%%%%%%%%%%%%%%%%%%%%%%%%%%%%%%%%%%%%%%%%%%%%%%%%%%%%%%%%%%%%%%%%%%%%%%%%%%%%%%%%%%%%%%%%%%%%%%%%%%%%%%%%%%%%%%%%%%%
\begin{document}
%Header-Make sure you update this information!!!!
\noindent
%%%%%%%%%%%%%%%%%%%%%%%%%%%%%%%%%%%%%%%%%%%%%%%%%%%%%%%%%%%%%%%%%%%%%%%%%%%%%%%%%%%%%%%%%%%%%%%%%%%%%%%%%%%%%%%%%%%%%%%%%%%%%%%%%%%%%%%%
\large\textbf{xxx} \hfill \textbf{Homework - 14}   \\
Email: xxx \hfill ID: xxx \\
\normalsize Course: Linear Model   \hfill Term: Spring 2020\\
\noindent\rule{7in}{2.8pt}
%%%%%%%%%%%%%%%%%%%%%%%%%%%%%%%%%%%%%%%%%%%%%%%%%%%%%%%%%%%%%%%%%%%%%%%%%%%%%%%%%%%%%%%%%%%%%%%%%%%%%%%%%%%%%%%%%%%%%%%%%%%%%%%%%%%%%%%%
% Problem
%%%%%%%%%%%%%%%%%%%%%%%%%%%%%%%%%%%%%%%%%%%%%%%%%%%%%%%%%%%%%%%%%%%%%%%%%%%%%%%%%%%%%%%%%%%%%%%%%%%%%%%%%%%%%%%%%%%%%%%%%%%%%%%%%%%%%%%%
\begin{problem}{10.2.1}
Show that the scaled deviance contribution for a binomial response with probability density 
\\
$\left(\begin{array}{l}m \\ r\end{array}\right) \pi^{r}(1-\pi)^{m-r}, 0<\pi<1, r=0, \ldots, m,$ and $\eta=\pi$ is
    $$
    2\{r \log \{r /(m \widehat{\pi})\}+(m-r) \log [(m-r) /\{m(1-\widehat{\pi})\}]\}.
    $$
\end{problem}
\begin{solution}

\end{solution}

\noindent\rule{7in}{2.8pt}

%%%%%%%%%%%%%%%%%%%%%%%%%%%%%%%%%%%%%%%%%%%%%%%%%%%%%%%%%%%%%%%%%%%%%%%%%%%%%%%%%%%%%%%%%%%%%%%%%%%%%%%%%%%%%%%%%%%%%%%%%%%%%%%%%%%%%%%%
% Problem
%%%%%%%%%%%%%%%%%%%%%%%%%%%%%%%%%%%%%%%%%%%%%%%%%%%%%%%%%%%%%%%%%%%%%%%%%%%%%%%%%%%%%%%%%%%%%%%%%%%%%%%%%%%%%%%%%%%%%%%%%%%%%%%%%%%%%%%%
\begin{problem}{10.2.2}
Show that the scaled deviance contribution for a Poisson response with density $\eta^{y} e^{-\eta} / y !$ $\eta>0, y=0,1, \ldots,$ is $2\{y \log (y / \widehat{\eta})-y+\widehat{\eta}\}.$
\end{problem}
\begin{solution}

\end{solution}

\noindent\rule{7in}{2.8pt}


%%%%%%%%%%%%%%%%%%%%%%%%%%%%%%%%%%%%%%%%%%%%%%%%%%%%%%%%%%%%%%%%%%%%%%%%%%%%%%%%%%%%%%%%%%%%%%%%%%%%%%%%%%%%%%%%%%%%%%%%%%%%%%%%%%%%%%%%

\begin{problem}{10.2.3}
    Consider a linear model with non-normal errors, in which the $j$ th response is $y_{j}=\eta_{j}+$ $\tau \varepsilon_{j},$ where $\eta_{j}=x_{j}^{\mathrm{T}} \beta,$ and $\varepsilon_{j}$ has density $\exp u(\varepsilon),$ for $j=1, \ldots, n.$\\
    (a) Show that the log likelihood contribution from $y_{j}$ may be written as $u\left(z_{j}\right)-\log \tau$ and hence express in terms of $u(\cdot)$ and $\tau$ the quantities needed to obtain the maximum likelihood estimates of $\beta$ by iterative weighted least squares.
    \\(b) Let $x_{j}^{\mathrm{T}}=\left(1, z_{j}^{\mathrm{T}}\right),$ so that the covariate matrix equals $X=\left(1_{n}, Z\right)$ with $1_{n}^{\mathrm{T}} Z=0 .$ Show that the expected information matrix may be written
    $$
    \tau^{-2}\left(\begin{array}{cc}
    Z^{\mathrm{T}} Z & 0 \\
    0 & n A
    \end{array}\right)
    $$
    where $A$ is a $2 \times 2$ matrix that does not depend on the parameters. Show further that $A$ is diagonal if the density of $\varepsilon_{j}$ is symmetric about zero.\\
    (c) Give the matrix $A$ for the $t$ density (3.11) and for the Gumbel density (10.10)
\end{problem}
  \begin{solution}
  ...
  \end{solution}
  
  \noindent\rule{7in}{2.8pt}

% Problem
%%%%%%%%%%%%%%%%%%%%%%%%%%%%%%%%%%%%%%%%%%%%%%%%%%%%%%%%%%%%%%%%%%%%%%%%%%%%%%%%%%%%%%%%%%%%%%%%%%%%%%%%%%%%%%%%%%%%%%%%%%%%%%%%%%%%%%%%
\begin{problem}{10.2.5}
    Show that for a normal linear model in which $\phi$ is replaced by $\widehat{\phi}=S S(\widehat{\beta}) /(n-p),$ the standardized deviance and Pearson residuals both equal the usual standardized residual $r_{j},$ and hence verify that (10.13) reduces to (8.30)

\end{problem}
\begin{solution}

\end{solution}

\noindent\rule{7in}{2.8pt}

%%%%%%%%%%%%%%%%%%%%%%%%%%%%%%%%%%%%%%%%%%%%%%%%%%%%%%%%%%%%%%%%%%%%%%%%%%%%%%%%%%%%%%%%%%%%%%%%%%%%%%%%%%%%%%%%%%%%%%%%%%%%%%%%%%%%%%%%

%%%%%%%%%%%%%%%%%%%%


% Problem
%%%%%%%%%%%%%%%%%%%%%%%%%%%%%%%%%%%%%%%%%%%%%%%%%%%%%%%%%%%%%%%%%%%%%%%%%%%%%%%%%%%%%%%%%%%%%%%%%%%%%%%%%%%%%%%%%%%%%%%%%%%%%%%%%%%%%%%%
\begin{problem}{10.2.7}
    In a nonlinear normal regression model, suppose that $\eta=\beta_{0}\left(1-e^{-\beta_{1} x}\right) .$ Let $S S\left(\beta_{1}\right)$ be the residual sum of squares when $\beta_{1}$ is known, that is, when the single covariate $1-e^{-\beta_{1} x}$ is fitted. Show that the profile log likelihood for $\beta_{1}$ can be written as
    \[
    \ell_{\mathrm{p}}\left(\beta_{1}\right)=\max _{\beta_{0}, \sigma^{2}} \ell\left(\beta_{0}, \beta_{1}, \sigma^{2}\right) \equiv-\frac{n}{2} \log s^{2}\left(\beta_{1}\right),
    \]
    and give the form of a $(1-2 \alpha)$ confidence interval for $\beta_{1}$ based on $\ell_{\mathrm{p}}.$

\end{problem}
\begin{solution}

\end{solution}

\noindent\rule{7in}{2.8pt}

%%%%%%%%%%%%%%%%%%%%%%%%%%%%%%%%%%%%%%%%%%%%%%%%%%%%%%%%%%%%%%%%%%%%%%%%%
\end{document}
