\documentclass[a4paper, 11pt]{article}
\usepackage{comment} % enables the use of multi-line comments (\ifx \fi) 
\usepackage{lipsum} %This package just generates Lorem Ipsum filler text. 
\usepackage{fullpage} % changes the margin
\usepackage[a4paper, total={7in, 10in}]{geometry}
\usepackage[fleqn]{amsmath}
\usepackage{amssymb,amsthm}  % assumes amsmath package installed
\newtheorem{theorem}{Theorem}
\newtheorem{corollary}{Corollary}
\usepackage{graphicx}
\usepackage{tikz}
\usetikzlibrary{arrows}
\usepackage{verbatim}
\usepackage[numbered]{mcode}
\usepackage{float}
\usepackage{tikz}
    \usetikzlibrary{shapes,arrows}
    \usetikzlibrary{arrows,calc,positioning}

    \tikzset{
        block/.style = {draw, rectangle,
            minimum height=1cm,
            minimum width=1.5cm},
        input/.style = {coordinate,node distance=1cm},
        output/.style = {coordinate,node distance=4cm},
        arrow/.style={draw, -latex,node distance=2cm},
        pinstyle/.style = {pin edge={latex-, black,node distance=2cm}},
        sum/.style = {draw, circle, node distance=1cm},
    }
\usepackage{xcolor}
\usepackage{mdframed}
\usepackage[shortlabels]{enumitem}
\usepackage{indentfirst}
\usepackage{hyperref}
    
\renewcommand{\thesubsection}{\thesection.\alph{subsection}}

\newenvironment{problem}[2][Problem]
    { \begin{mdframed}[backgroundcolor=gray!20] \textbf{#1 #2} \\}
    {  \end{mdframed}}

% Define solution environment
\newenvironment{solution}
    {\textit{Solution:}}
    {}

\renewcommand{\qed}{\quad\qedsymbol}
%%%%%%%%%%%%%%%%%%%%%%%%%%%%%%%%%%%%%%%%%%%%%%%%%%%%%%%%%%%%%%%%%%%%%%%%%%%%%%%%%%%%%%%%%%%%%%%%%%%%%%%%%%%%%%%%%%%%%%%%%%%%%%%%%%%%%%%%
\begin{document}
%Header-Make sure you update this information!!!!
\noindent
%%%%%%%%%%%%%%%%%%%%%%%%%%%%%%%%%%%%%%%%%%%%%%%%%%%%%%%%%%%%%%%%%%%%%%%%%%%%%%%%%%%%%%%%%%%%%%%%%%%%%%%%%%%%%%%%%%%%%%%%%%%%%%%%%%%%%%%%
\large\textbf{xxx} \hfill \textbf{Homework - 9}   \\
Email: xxx \hfill ID: xxx \\
\normalsize Course: Statistical Models II \hfill Term: Spring 2020\\
%Instructor: Shiyuan He, Wei Ma \hfill Due Date: $18^{th}$ Apr., 2020 \\
\noindent\rule{7in}{2.8pt}
%%%%%%%%%%%%%%%%%%%%%%%%%%%%%%%%%%%%%%%%%%%%%%%%%%%%%%%%%%%%%%%%%%%%%%%%%%%%%%%%%%%%%%%%%%%%%%%%%%%%%%%%%%%%%%%%%%%%%%%%%%%%%%%%%%%%%%%%%%%%%%%%%%%%%%%%%%%%%%%%%%%%%%%%%%%%%%%%%%%%%%%%%%%%%%%%%%%%%%%%%%%%%%%
% Problem 1
%%%%%%%%%%%%%%%%%%%%%%%%%%%%%%%%%%%%%%%%%%%%%%%%%%%%%%%%%%%%%%%%%%%%%%%%%%%%%%%%%%%%%%%%%%%%%%%%%%%%%%%%%%%%%%%%%%%%%%%%%%%%%%%%%%%%%%%%
\begin{problem}{8.13}
A study of student performance may follow a balanced two-way nested model
$$
y_{i j k}=\mu+\alpha_{i}+\beta_{i j}+e_{i j k}, i=1, \ldots, a ; j=1, \ldots, b ; k=1, \ldots, c
$$
where $i$ may represent schools, $j$ teachers within a school, and $k$ individual students. The standard sums of squares decomposition follows:
$$
\begin{aligned}
S S A &=\sum_{i} b c\left(y_{i . .}-\bar{y}_{. .}\right)^{2} \\
S S B(A) &=\sum_{i} \sum_{j} c\left(y_{i j .}-\bar{y}_{i . .}\right)^{2}, \text { and } \\
S S E &=\sum_{i} \sum_{j} \sum_{k}\left(y_{i j k}-\bar{y}_{i j .}\right)^{2}
\end{aligned}
$$
Use the tools outlined in Section 8.2 for the following problems.

a. Derive (show the steps, fill in the degrees of freedom, $d f$ ) the following expected mean squares where all three effects are random, that is, $\alpha_{i} i i d N\left(0, \sigma_{a}^{2}\right), \beta_{i j} i i d N\left(0, \sigma_{b}^{2}\right),$ and $e_{i j k} i i d N\left(0, \sigma^{2}\right),$ and all three $\alpha_{i}$
$\bar{\beta}_{i j},$ and $e_{i j k}$ mutually independent:
$$
\begin{aligned}
E(S S A / d f) &=\sigma^{2}+c \sigma_{b}^{2}+b c \sigma_{a}^{2} \\
E(S S B(A) / d f) &=\sigma^{2}+c \sigma_{b}^{2} \\
E(S S E / d f) &=\sigma^{2}
\end{aligned}
$$

b. Derive the distribution of an appropriate test statistic for testing the hypothesis $H: \sigma_{a}^{2}=0$

c. If $\alpha_{i}$ were fixed, and the other two effects random, derive the expected mean squares for the same three sums of squares.
\end{problem}
\begin{solution}
...
\end{solution} 

\noindent\rule{7in}{2.8pt}

%%%%%%%%%%%%%%%%%%%%%%%%%%%%%%%%%%%%%%%%%%%%%%%%%%%%%%%%%%%%%%%%%%%%%%%%%
% Problem 2
%%%%%%%%%%%%%%%%%%%%%%%%%%%%%%%%%%%%%%%%%%%%%%%%%%%%%%%%%%%%%%%%%%%%%%%%%%%%%%%%%%%%%%%%%%%%%%%%%%%%%%%%%%%%%%%%%%%%%%%%%%%%%%%%%%%%%%%%

\begin{problem}{8.17}
Fill in the missing algebra from Equation(8.26) to finding the BLUP for forecasting $y_{*}$ in Equation (8.27).
\end{problem}
\begin{solution}
...
\end{solution} 

\noindent\rule{7in}{2.8pt}
%%%%%%%%%%%%%%%%%%%%%%%%%%%%%%%%%%%%%%%%%%%%%%%%%%%%%%%%%%%%%%%%%%%%%%%%%
% Problem 3
%%%%%%%%%%%%%%%%%%%%%%%%%%%%%%%%%%%%%%%%%%%%%%%%%%%%%%%%%%%%%%%%%%%%%%%%%%%%%%%%%%%%%%%%%%%%%%%%%%%%%%%%%%%%%%%%%%%%%%%%%%%%%%%%%%%%%%%%

\begin{problem}{8.18}
Show that variance of the BLUP in Equation (8.27) is 
$$\omega^T(\Omega^{-1}-\Omega^{-1}X(X^T\Omega^{-1}X)^gX^T\Omega^{-1})\omega+x_{*}^T(X^T\Omega^{-1}X)^gx_{*}$$
\end{problem}
\begin{solution}
...
\end{solution} 

\noindent\rule{7in}{2.8pt}
%%%%%%%%%%%%%%%%%%%%%%%%%%%%%%%%%%%%%%%%%%%
%%%%%%%%%%%%%%%%%%%%%%%%%%%%%%%%%%%%%%%%%%%
\end{document}