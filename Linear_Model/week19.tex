\documentclass[a4paper, 11pt]{article}
\usepackage{comment} % enables the use of multi-line comments (\ifx \fi)
\usepackage{lipsum} %This package just generates Lorem Ipsum filler text.
\usepackage{fullpage} % changes the margin
\usepackage[a4paper, total={7in, 10in}]{geometry}
\usepackage[fleqn]{amsmath}
\usepackage{amssymb,amsthm}  % assumes amsmath package installed
\newtheorem{theorem}{Theorem}
\newtheorem{corollary}{Corollary}
\usepackage{graphicx}
\usepackage{tikz}
\usetikzlibrary{arrows}
\usepackage{verbatim}

\newcommand{\E}{\mathbb{E}}
\newcommand{\R}{\mathbb{R}}
\renewcommand{\P}{\mathbb{P}}
\newcommand{\normal}{\mathcal{N}}
\newcommand{\Var}{\mathrm{Var}}
\newcommand{\Cov}{\mathrm{Cov}}

\usepackage[numbered]{mcode}
\usepackage{float}
\usepackage{tikz}
    \usetikzlibrary{shapes,arrows}
    \usetikzlibrary{arrows,calc,positioning}

    \tikzset{
        block/.style = {draw, rectangle,
            minimum height=1cm,
            minimum width=1.5cm},
        input/.style = {coordinate,node distance=1cm},
        output/.style = {coordinate,node distance=4cm},
        arrow/.style={draw, -latex,node distance=2cm},
        pinstyle/.style = {pin edge={latex-, black,node distance=2cm}},
        sum/.style = {draw, circle, node distance=1cm},
    }
\usepackage{xcolor}
\usepackage{listings}
\lstset{
    %backgroundcolor=\color{red!50!green!50!blue!50},%代码块背景色为浅灰色
    rulesepcolor= \color{gray}, %代码块边框颜色
    breaklines=true,  %代码过长则换行
    numbers=none, %行号在左侧显示
    numberstyle= \small,%行号字体
    %keywordstyle= \color{blue},%关键字颜色
    commentstyle=\color{gray}, %注释颜色
    frame=shadowbox%用方框框住代码块
    }
\usepackage{mdframed}
\usepackage[shortlabels]{enumitem}
\usepackage{indentfirst}
\usepackage{hyperref}

\renewcommand{\thesubsection}{\thesection.\alph{subsection}}

\newenvironment{problem}[2][Problem]
    { \begin{mdframed}[backgroundcolor=gray!20] \textbf{#1 #2} \\}
    {  \end{mdframed}}

% Define solution environment
\newenvironment{solution}
    {\textit{Solution:}}
    {}

\renewcommand{\qed}{\quad\qedsymbol}
%%%%%%%%%%%%%%%%%%%%%%%%%%%%%%%%%%%%%%%%%%%%%%%%%%%%%%%%%%%%%%%%%%%%%%%%%%%%%%%%%%%%%%%%%%%%%%%%%%%%%%%%%%%%%%%%%%%%%%%%%%%%%%%%%%%%%%%%
\begin{document}
%Header-Make sure you update this information!!!!
\noindent
%%%%%%%%%%%%%%%%%%%%%%%%%%%%%%%%%%%%%%%%%%%%%%%%%%%%%%%%%%%%%%%%%%%%%%%%%%%%%%%%%%%%%%%%%%%%%%%%%%%%%%%%%%%%%%%%%%%%%%%%%%%%%%%%%%%%%%%%
\large\textbf{Yonghua Su} \hfill \textbf{Homework - 16}   \\
Email: 2019000154@ruc.edu.cn \hfill ID: 2019000154 \\
\normalsize Course: Linear Model   \hfill Term: Spring 2020\\
\noindent\rule{7in}{2.8pt}


%%%%%%%%%%%%%%%%%%%%%%%%%%%%%%%%%%%%%%%%%%%%%%%%%%%%%%%%%%%%%%%%%%%%%%%%%%%%%%%%%%%%%%%%%%%%%%%%%%%%%%%%%%%%%%%%%%%%%%%%%%%%%%%%%%%%%%%%
% Problem1
%%%%%%%%%%%%%%%%%%%%%%%%%%%%%%%%%%%%%%%%%%%%%%%%%%%%%%%%%%%%%%%%%%%%%%%%%%%%%%%%%%%%%%%%%%%%%%%%%%%%%%%%%%%%%%%%%%%%%%%%%%%%%%%%%%%%%%%%
\begin{problem}{10.8.1}
  Show that if $Y$ is continuous with cumulative hazard function $H(y),$ then $H(Y)$ has the unit exponential distribution. Hence establish that $\mathrm{E}\{H(Y) \mid Y>c\}=1+H(c),$ and explain the reasoning behind (10.55)
\end{problem}
\begin{solution}

\end{solution}

\noindent\rule{7in}{2.8pt}

%%%%%%%%%%%%%%%%%%%%%%%%%%%%%%%%%%%%%%%%%%%%%%%%%%%%%%
% Problem2
%%%%%%%%%%%%%%%%%%%%%%%%%%%%%%%%%%%%%%%%%%%%%%%%%%%%%%%%%%%%%%%%%%%%%%%%%%%%%%%%%%%%%%%%%%%%%%%%%%%%%%%%%%%%%%%%%%%%%%%%%%%%%%%%%%%%%%%%
\begin{problem}{10.8.2}
  Let $Y$ be a positive continuous random variable with survivor and hazard functions $\mathcal{F}(y)$ and $h(y) .$ Let $\psi(x)$ and $\chi(x)$ be arbitrary continuous positive functions of the covariate $x$ with $\psi(0)=\chi(0)=1 .$ In a proportional hazards model, the effect of a non-zero covariate is that the hazard function becomes $h(y) \psi(x),$ whereas in an accelerated life model, the survivor function becomes $\mathcal{F}\{y \chi(x)\} .$ Show that the survivor function for the proportional hazards model is $\mathcal{F}(y)^{\psi(x)},$ and deduce that this model is also an accelerated life model if and only if
  $$
\log \psi(x)+G(\tau)=G\{\tau+\log \chi(x)\}
$$
where $G(\tau)=\log \left\{-\log \mathcal{F}\left(e^{\tau}\right)\right\} .$ Show that if this holds for all $\tau$ and some non-unit $\chi(x)$ we must have $G(\tau)=\kappa \tau+\alpha,$ for constants $\kappa$ and $\alpha,$ and find an expression for $\chi(x)$ in terms of $\psi(x) .$ Hence or otherwise show that the classes of proportional hazards and accelerated life models coincide if and only if $Y$ has a Weibull distribution.
\end{problem}
\begin{solution}

\end{solution}

\noindent\rule{7in}{2.8pt}
%%%%%%%%%%%%%%%%%%%%%%%%%%%%%%%%%%%%%%%%%%%%%%%%%%%%%%
% Problem3
%%%%%%%%%%%%%%%%%%%%%%%%%%%%%%%%%%%%%%%%%%%%%%%%%%%%%%%%%%%%%%%%%%%%%%%%%%%%%%%%%%%%%%%%%%%%%%%%%%%%%%%%%%%%%%%%%%%%%%%%%%%%%%%%%%%%%%%%
\begin{problem}{10.8.3}
  In the usual notation for a linear regression model, $X^{\mathrm{T}}(y-X \widehat{\beta})=0 .$ By writing the partial likelihood corresponding to (10.62) as $\sum_{j=1}^{n} d_{j}\left\{x_{j}^{\mathrm{T}} \beta-\log A_{j}(\beta)\right\},$ show that
$$
\sum_{j=1}^{n} x_{j}\left\{d_{j}-\exp \left(x_{j}^{\mathrm{T}} \widehat{\beta}\right) \widehat{H}_{0}\left(y_{j}\right)\right\}=0
$$
Which type of residual for a proportional hazards model is analogous to the raw residual in a linear model?

\end{problem}
\begin{solution}

\end{solution}

\noindent\rule{7in}{2.8pt}
%%%%%%%%%%%%%%%%%%%%%%%%%%%%%%%%%%%%%%%%%%%%%%%%%%%%%%
% Problem3
%%%%%%%%%%%%%%%%%%%%%%%%%%%%%%%%%%%%%%%%%%%%%%%%%%%%%%%%%%%%%%%%%%%%%%%%%%%%%%%%%%%%%%%%%%%%%%%%%%%%%%%%%%%%%%%%%%%%%%%%%%%%%%%%%%%%%%%%
\begin{problem}{10.8.5}
  Write down the partial likelihood contributions from failure times $y=1,2,$ for the data in Table $10.22,$ using the model $\xi=\exp \left\{\beta_{0}+\beta_{1} I(\text { Group }=1)+\beta_{2} x\right\}$
\end{problem}
\begin{solution}

\end{solution}

\noindent\rule{7in}{2.8pt}


\end{document}
