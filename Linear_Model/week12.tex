\documentclass[a4paper, 11pt]{article}
\usepackage{comment} % enables the use of multi-line comments (\ifx \fi) 
\usepackage{lipsum} %This package just generates Lorem Ipsum filler text. 
\usepackage{fullpage} % changes the margin
\usepackage[a4paper, total={7in, 10in}]{geometry}
\usepackage[fleqn]{amsmath}
\usepackage{amssymb,amsthm}  % assumes amsmath package installed
\newtheorem{theorem}{Theorem}
\newtheorem{corollary}{Corollary}
\usepackage{graphicx}
\usepackage{tikz}
\usetikzlibrary{arrows}
\usepackage{verbatim}
\usepackage[numbered]{mcode}
\usepackage{float}
\usepackage{tikz}
    \usetikzlibrary{shapes,arrows}
    \usetikzlibrary{arrows,calc,positioning}

    \tikzset{
        block/.style = {draw, rectangle,
            minimum height=1cm,
            minimum width=1.5cm},
        input/.style = {coordinate,node distance=1cm},
        output/.style = {coordinate,node distance=4cm},
        arrow/.style={draw, -latex,node distance=2cm},
        pinstyle/.style = {pin edge={latex-, black,node distance=2cm}},
        sum/.style = {draw, circle, node distance=1cm},
    }
\usepackage{xcolor}
\usepackage{mdframed}
\usepackage[shortlabels]{enumitem}
\usepackage{indentfirst}
\usepackage{hyperref}
\renewcommand{\thesubsection}{\thesection.\alph{subsection}}

\newenvironment{problem}[2][Problem]
    { \begin{mdframed}[backgroundcolor=gray!20] \textbf{#1 #2} \\}
    {  \end{mdframed}}

% Define solution environment
\newenvironment{solution}
    {\textit{Solution:}}
    {}

\renewcommand{\qed}{\quad\qedsymbol}
%%%%%%%%%%%%%%%%%%%%%%%%%%%%%%%%%%%%%%%%%%%%%%%%%%%%%%%%%%%%%%%%%%%%%%%%%%%%%%%%%%%%%%%%%%%%%%%%%%%%%%%%%%%%%%%%%%%%%%%%%%%%%%%%%%%%%%%%
\begin{document}
%Header-Make sure you update this information!!!!
\noindent
%%%%%%%%%%%%%%%%%%%%%%%%%%%%%%%%%%%%%%%%%%%%%%%%%%%%%%%%%%%%%%%%%%%%%%%%%%%%%%%%%%%%%%%%%%%%%%%%%%%%%%%%%%%%%%%%%%%%%%%%%%%%%%%%%%%%%%%%
\large\textbf{XXX} \hfill \textbf{Homework 12}   \\
Email: XXX@ruc.edu.cn \hfill ID: 20190001-- \\
\normalsize Course: Statistical Models II \hfill Term: Spring 2020\\
%Instructor: Shiyuan He, Wei Ma \hfill Due Date: $18^{th}$ Apr., 2020 \\
\noindent\rule{7in}{2.8pt}
%%%%%%%%%%%%%%%%%%%%%%%%%%%%%%%%%%%%%%%%%%%%%%%%%%%%%%%%%%%%%%%%%%%%%%%%%%%%%%%%%%%%%%%%%%%%%%%%%%%%%%%%%%%%%%%%%%%%%%%%%%%%%%%%%%%%%%%%%%%%%%%%%%%%%%%%%%%%%%%%%%%%%%%%%%%%%%%%%%%%%%%%%%%%%%%%%%%%%%%%%%%%%%%
% Problem 1
%%%%%%%%%%%%%%%%%%%%%%%%%%%%%%%%%%%%%%%%%%%%%%%%%%%%%%%%%%%%%%%%%%%%%%%%%%%%%%%%%%%%%%%%%%%%%%%%%%%%%%%%%%%%%%%%%%%%%%%%%%%%%%%%%%%%%%%%
\begin{problem}{9.1.1}
(a) Show that under the two-sample model, the difference of the sample averages, $\bar{y}_{2}-\bar{y}_{1}$, has variance $\left(n_{1}+n_{2}\right) \sigma^{2} /\left(n_{1} n_{2}\right)$. Show that subject to $n_1 + n_2 = n$, this is minimized
when $n_1$ and $n_2$ are as nearly equal as possible.

(b) Suppose that $n$ units are split into $k$ blocks of size $m + 1$, and that one unit in each block is chosen at random to be treated, while the remaining m are controls. Suppose that the responses in the $j$-th block are $y_{j1}$ and $y_{j2},\cdots, y_{j(m+1)}$, and let $d_j$ represent the
difference between the treated individual and the average of the controls. Show that the
average of these differences has variance $(m+1) \sigma^{2} /(k m)$, and show that for fixed $n$ this is minimized when $m = 1$.
\end{problem}
\begin{solution}
\end{solution} 

\noindent\rule{7in}{2.8pt}

%%%%%%%%%%%%%%%%%%%%%%%%%%%%%%%%%%%%%%%%%%%%%%%%%%%%%%%%%%%%%%%%%%%%%%%%%%%%%%%%%%%%%%%%%%%%%%%%%%%%%%%%%%%%%%%%%%%%%%%%%%%%%%%%%%%%%%%%%%%%%%%%%%%%%%%%%%%%%%%%%%%%%%%%%%%%%%%%%%%%%%%%%%%%%%%%%%%%%%%%%%%%%%%%%%%%%%%%%%%%%%%%%%%%%%%%%%%%%%%%%%%%%%%%%%%%%%%%%%%%%%%%%%%%%%%%%%%%%%%%
% Problem 2
%%%%%%%%%%%%%%%%%%%%%%%%%%%%%%%%%%%%%%%%%%%%%%%%%%%%%%%%%%%%%%%%%%%%%%%%%%%%%%%%%%%%%%%%%%%%%%%%%%%%%%%%%%%%%%%%%%%%%%%%%%%%%%%%%%%%%%%%
\begin{problem}{9.1.2}
Suppose a paired comparison experiment is performed, in which the $j$-th pair satisfies the normal linear model

$$y_{0,}=\mu_{j}-\delta+\varepsilon_{0 j}, \quad y_{1 j}=\mu_{j}+\delta+\epsilon_{1 j}, \quad j=1, \ldots, \ldots$$

but that data analysis is performed using the two-sample model. Show that the variance estimator can be written as

$$S^{2}=\frac{1}{2(m-1)} \sum_{j, t}\left(\mu_{j}-\bar{\mu}+\varepsilon_{t j}-\bar{\varepsilon}_{\ldots}\right)^{2}$$

Deduce that this has expected value $\sigma^{2}+(m-1)^{-1} \sum_{i}\left(\mu_{j}-\bar{\mu}_{\cdot}\right)^{2}$ conditional on the $\mu_j$, and hence show that if the µj are normally distributed with variance $\tau_2$, then $\mathbb{E}(S_2) =\sigma^{2}+\tau^{2}$.

Show that if the two-sample model is used in this situation, the length of a $95\%$ confidence interval for $2\delta$ is roughly $2\left(\sigma^{2}+\tau^{2}\right)^{1 / 2} t_{2(m-1)}(0.025)$, whereas under the paired comparisons model the length is about $2 \sigma t_{m-1}(0.025)$. 

For what values of $\tau^{2} / \sigma^{2}$ are the two-sample intervals shorter when 

(a) $m = 3$, 

(b) $m = 11$? 

Discuss your results.
\end{problem}
\begin{solution}
\end{solution} 

\noindent\rule{7in}{2.8pt}

%%%%%%%%%%%%%%%%%%%%%%%%%%%%%%%%%%%%%%%%%%%%%%%%%%%%%%%%%%%%%%%%%%%%%%%%%%%%%%%%%%%%%%%%%%%%%%%%%%%%%%%%%%%%%%%%%%%%%%%%%%%%%%%%%%%%%%%%%%%%%%%%%%%%%%%%%%%%%%%%%%%%%%%%%%%%%%%%%%%%%%%%%%%%%%%%%%%%%%%%%%%%%%%%%%%%%%%%%%%%%%%%%%%%%%%%%%%%%%%%%%%%%%%%%%%%%%%%%%%%%%%%%%%%%%%%%%%%%%%%
% Problem 3
%%%%%%%%%%%%%%%%%%%%%%%%%%%%%%%%%%%%%%%%%%%%%%%%%%%%%%%%%%%%%%%%%%%%%%%%%%%%%%%%%%%%%%%%%%%%%%%%%%%%%%%%%%%%%%%%%%%%%%%%%%%%%%%%%%%%%%%%
\begin{problem}{9.1.3}
	Check (9.3), find $\operatorname{var}\left(T_{j}\right)$ and $\operatorname{cov}\left(T_{i}, T_{k}\right)$ and hence verify the given formulae for the mean and variance of $\bar{Y}_{1}-\bar{Y}_{0}$.
\end{problem}
\begin{solution}
\end{solution} 

\noindent\rule{7in}{2.8pt}

%%%%%%%%%%%%%%%%%%%%%%%%%%%%%%%%%%%%%%%%%%%%%%%%%%%%%%%%%%%%%%%%%%%%%%%%%%%%%%%%%%%%%%%%%%%%%%%%%%%%%%%%%%%%%%%%%%%%%%%%%%%%%%%%%%%%%%%%%%%%%%%%%%%%%%%%%%%%%%%%%%%%%%%%%%%%%%%%%%%%%%%%%%%%%%%%%%%%%%%%%%%%%%%%%%%%%%%%%%%%%%%%%%%%%%%%%%%%%%%%%%%%%%%%%%%%%%%%%%%%%%%%%%%%%%%%%%%%%%%%
% Problem 4
%%%%%%%%%%%%%%%%%%%%%%%%%%%%%%%%%%%%%%%%%%%%%%%%%%%%%%%%%%%%%%%%%%%%%%%%%%%%%%%%%%%%%%%%%%%%%%%%%%%%%%%%%%%%%%%%%%%%%%%%%%%%%%%%%%%%%%%%
\begin{problem}{9.1.4}
	In Example 9.1, show that $Z$ is a monotonic function of $\bar{D}$.
\end{problem}
\begin{solution}
\end{solution} 

\noindent\rule{7in}{2.8pt}

%%%%%%%%%%%%%%%%%%%%%%%%%%%%%%%%%%%%%%%%%%%%%%%%%%%%%%%%%%%%%%%%%%%%%%%%%
\end{document}