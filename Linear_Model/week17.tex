\documentclass[a4paper, 11pt]{article}
\usepackage{comment} % enables the use of multi-line comments (\ifx \fi)
\usepackage{lipsum} %This package just generates Lorem Ipsum filler text.
\usepackage{fullpage} % changes the margin
\usepackage[a4paper, total={7in, 10in}]{geometry}
\usepackage[fleqn]{amsmath}
\usepackage{amssymb,amsthm}  % assumes amsmath package installed
\newtheorem{theorem}{Theorem}
\newtheorem{corollary}{Corollary}
\usepackage{graphicx}
\usepackage{tikz}
\usetikzlibrary{arrows}
\usepackage{verbatim}

\newcommand{\E}{\mathbb{E}}
\newcommand{\R}{\mathbb{R}}
\renewcommand{\P}{\mathbb{P}}
\newcommand{\normal}{\mathcal{N}}
\newcommand{\Var}{\mathrm{Var}}
\newcommand{\Cov}{\mathrm{Cov}}

\usepackage[numbered]{mcode}
\usepackage{float}
\usepackage{tikz}
    \usetikzlibrary{shapes,arrows}
    \usetikzlibrary{arrows,calc,positioning}

    \tikzset{
        block/.style = {draw, rectangle,
            minimum height=1cm,
            minimum width=1.5cm},
        input/.style = {coordinate,node distance=1cm},
        output/.style = {coordinate,node distance=4cm},
        arrow/.style={draw, -latex,node distance=2cm},
        pinstyle/.style = {pin edge={latex-, black,node distance=2cm}},
        sum/.style = {draw, circle, node distance=1cm},
    }
\usepackage{xcolor}
\usepackage{listings}
\lstset{
    %backgroundcolor=\color{red!50!green!50!blue!50},%代码块背景色为浅灰色
    rulesepcolor= \color{gray}, %代码块边框颜色
    breaklines=true,  %代码过长则换行
    numbers=none, %行号在左侧显示
    numberstyle= \small,%行号字体
    %keywordstyle= \color{blue},%关键字颜色
    commentstyle=\color{gray}, %注释颜色
    frame=shadowbox%用方框框住代码块
    }
\usepackage{mdframed}
\usepackage[shortlabels]{enumitem}
\usepackage{indentfirst}
\usepackage{hyperref}

\renewcommand{\thesubsection}{\thesection.\alph{subsection}}

\newenvironment{problem}[2][Problem]
    { \begin{mdframed}[backgroundcolor=gray!20] \textbf{#1 #2} \\}
    {  \end{mdframed}}

% Define solution environment
\newenvironment{solution}
    {\textit{Solution:}}
    {}

\renewcommand{\qed}{\quad\qedsymbol}
%%%%%%%%%%%%%%%%%%%%%%%%%%%%%%%%%%%%%%%%%%%%%%%%%%%%%%%%%%%%%%%%%%%%%%%%%%%%%%%%%%%%%%%%%%%%%%%%%%%%%%%%%%%%%%%%%%%%%%%%%%%%%%%%%%%%%%%%
\begin{document}
%Header-Make sure you update this information!!!!
\noindent
%%%%%%%%%%%%%%%%%%%%%%%%%%%%%%%%%%%%%%%%%%%%%%%%%%%%%%%%%%%%%%%%%%%%%%%%%%%%%%%%%%%%%%%%%%%%%%%%%%%%%%%%%%%%%%%%%%%%%%%%%%%%%%%%%%%%%%%%
\large\textbf{XXXX} \hfill \textbf{Homework 16}   \\
Email: *****@ruc.edu.cn \hfill ID: 201900015* \\
\normalsize Course: Linear Model   \hfill Term: Spring 2020\\
\noindent\rule{7in}{2.8pt}


%%%%%%%%%%%%%%%%%%%%%%%%%%%%%%%%%%%%%%%%%%%%%%%%%%%%%%%%%%%%%%%%%%%%%%%%%%%%%%%%%%%%%%%%%%%%%%%%%%%%%%%%%%%%%%%%%%%%%%%%%%%%%%%%%%%%%%%%
% Problem1
%%%%%%%%%%%%%%%%%%%%%%%%%%%%%%%%%%%%%%%%%%%%%%%%%%%%%%%%%%%%%%%%%%%%%%%%%%%%%%%%%%%%%%%%%%%%%%%%%%%%%%%%%%%%%%%%%%%%%%%%%%%%%%%%%%%%%%%%
\begin{problem}{10.5.1}
  Consider the $2 \times n$ table of independent Poisson variables
  $$
  \begin{array}{lllll}
  Y_{11} & \cdots & Y_{1 j} & \cdots & Y_{1 n} \\
  Y_{21} & \cdots & Y_{2 j} & \cdots & Y_{2 n}
  \end{array}
  $$
  where
  $$
  \eta_{1 j}=\log \mathrm{E}\left(Y_{1 j}\right)=x_{1 j}^{\mathrm{T}} \beta, \quad \eta_{2 j}=\log \mathrm{E}\left(Y_{2 j}\right)=x_{2 j}^{\mathrm{T}} \beta
  $$
  Show that the conditional density of $Y_{1 j}$ given that $Y_{1 j}+Y_{2 j}=m_{j}$ is binomial with denominator $m_{j}$ and probability $\pi_{j}$ satisfying $\log \left\{\pi_{j} /\left(1-\pi_{j}\right)\right\}=x_{j}^{\mathrm{T}} \beta,$ where $x_{j}=x_{1 j}-$ $x_{2 j} .$
  This implies that a contingency table in which a single, binary, classification is regarded as the response can be analyzed using logistic regression. What advantages are there to doing so in terms of model-fitting and the examination of residuals?
\end{problem}
\begin{solution}


\end{solution}

\noindent\rule{7in}{2.8pt}

%%%%%%%%%%%%%%%%%%%%%%%%%%%%%%%%%%%%%%%%%%%%%%%%%%%%%%
% Problem2
%%%%%%%%%%%%%%%%%%%%%%%%%%%%%%%%%%%%%%%%%%%%%%%%%%%%%%%%%%%%%%%%%%%%%%%%%%%%%%%%%%%%%%%%%%%%%%%%%%%%%%%%%%%%%%%%%%%%%%%%%%%%%%%%%%%%%%%%
\begin{problem}{10.5.2}
In light of the preceding exercise and the discussion on page 501 , reconsider the models fitted in Example $10.21 .$ Say why Table 10.13 contains much larger standard errors for the logistic than for the log-linear model.
\end{problem}
\begin{solution}


\end{solution}

\noindent\rule{7in}{2.8pt}
%%%%%%%%%%%%%%%%%%%%%%%%%%%%%%%%%%%%%%%%%%%%%%%%%%%%%%
% Problem3
%%%%%%%%%%%%%%%%%%%%%%%%%%%%%%%%%%%%%%%%%%%%%%%%%%%%%%%%%%%%%%%%%%%%%%%%%%%%%%%%%%%%%%%%%%%%%%%%%%%%%%%%%%%%%%%%%%%%%%%%%%%%%%%%%%%%%%%%
\begin{problem}{10.5.3}
  For a $2 \times 2$ contingency table with probabilities
  $$
  \begin{array}{ll}
  \pi_{00} & \pi_{01} \\
  \pi_{10} & \pi_{11}
  \end{array}
  $$
  the maximal log-linear model may be written as
  $$
  \begin{array}{ll}
  \eta_{00}=\alpha+\beta+\gamma+(\beta \gamma), & \eta_{01}=\alpha+\beta-\gamma-(\beta \gamma) \\
  \eta_{10}=\alpha-\beta+\gamma-(\beta \gamma), & \eta_{11}=\alpha-\beta-\gamma+(\beta \gamma)
  \end{array}
  $$
  where $\eta_{j k}=\log \mathrm{E}\left(Y_{j k}\right)=\log \left(m \pi_{j k}\right)$ and $m=\sum_{j, k} y_{j k} .$ Show that the 'interaction' term
  $(\beta \gamma)$ may be written $(\beta \gamma)=\frac{1}{4} \log \Delta,$ where $\Delta$ is the odds ratio $\left(\pi_{00} \pi_{11}\right) /\left(\pi_{01} \pi_{10}\right),$ so that
  $(\beta \gamma)=0$ is equivalent to $\Delta=1$.

\end{problem}
\begin{solution}

\end{solution}

\noindent\rule{7in}{2.8pt}
%%%%%%%%%%%%%%%%%%%%%%%%%%%%%%%%%%%%%%%%%%%%%%%%%%%%%%
% Problem3
%%%%%%%%%%%%%%%%%%%%%%%%%%%%%%%%%%%%%%%%%%%%%%%%%%%%%%%%%%%%%%%%%%%%%%%%%%%%%%%%%%%%%%%%%%%%%%%%%%%%%%%%%%%%%%%%%%%%%%%%%%%%%%%%%%%%%%%%
\begin{problem}{Practicals 10.4}
  In  1961 and 1962 an experiment was conducted in Sweden to assess the effect of a speed limit on the accident rate on motorways. The experiment was conducted on 92 days in each year, matched so that day $j$ in 1961 was comparable to day $j$ in $1962 .$ On some days the speed limit was in effect and enforced, and not on other days. The number of accidents was recorded daily.
\begin{enumerate}[(a)]
  \item Let $Y_{i j}$ be the number of accidents on day $j$ in $196 i$, and let $I_{i j}$ indicate whether the speed limit was in effect that day. A simple model is that $Y_{1 j}$ has a Poisson distribution with mean $\lambda_{j} \exp \left(\beta I_{1 j}\right),$ while the corresponding mean for 1962 is $\lambda_{j} \exp \left(\alpha+\beta I_{2 j}\right)$
  Show that there is a cut in the likelihood, and deduce that inference for $\alpha$ and $\beta$ may be performed using the binomial conditional distribution of $Y_{1 j}$ given $Y_{1 j}+Y_{2 j},$ for $j=1, \ldots, 92 ;$ give this distribution.
  \item The data are in limits. The following code fits and displays the binomial model:
    \begin{lstlisting}[language=R]
      data(limits)
      limits.bin <- glm(cbind(y1,y2)~delta,binomial,data=limits)
      summary(limits.bin)
      plot.glm.diag(limits.bin)
    \end{lstlisting}

To fit the Poisson model:
    \begin{lstlisting}[language=R]
      attach(limits)
      acc <- c(y1,y2)
      d <- c(i61,i62)
      detach("limits")
      year <- rep(c(1,2),c(92,92))
      day <- factor(rep(1:92,2))
      limits.poi <- glm(acc~day+year+day,poisson)
      summary(limits.poi)
    \end{lstlisting}

Compare the output from the two fits, and verify that the estimates of the year and
speed limit effects and their standard errors are equal.

(Sections 10.4–10.5; Svensson, 1981)
\end{enumerate}
\end{problem}
\begin{solution}

\end{solution}

\noindent\rule{7in}{2.8pt}




\end{document}
