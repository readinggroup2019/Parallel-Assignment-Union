\documentclass[a4paper, 11pt]{article}
\usepackage{comment} % enables the use of multi-line comments (\ifx \fi)
\usepackage{lipsum} %This package just generates Lorem Ipsum filler text.
\usepackage{fullpage} % changes the margin
\usepackage[a4paper, total={7in, 10in}]{geometry}
\usepackage[fleqn]{amsmath}
\usepackage{amssymb,amsthm}  % assumes amsmath package installed
\newtheorem{theorem}{Theorem}
\newtheorem{corollary}{Corollary}
\usepackage{graphicx}
\usepackage{tikz}
\usetikzlibrary{arrows}
\usepackage{verbatim}
\usepackage[numbered]{mcode}
\usepackage{float}
\usepackage{tikz}
    \usetikzlibrary{shapes,arrows}
    \usetikzlibrary{arrows,calc,positioning}

    \tikzset{
        block/.style = {draw, rectangle,
            minimum height=1cm,
            minimum width=1.5cm},
        input/.style = {coordinate,node distance=1cm},
        output/.style = {coordinate,node distance=4cm},
        arrow/.style={draw, -latex,node distance=2cm},
        pinstyle/.style = {pin edge={latex-, black,node distance=2cm}},
        sum/.style = {draw, circle, node distance=1cm},
    }
\usepackage{xcolor}
\usepackage{mdframed}
\usepackage[shortlabels]{enumitem}
\usepackage{indentfirst}
\usepackage{hyperref}

\renewcommand{\thesubsection}{\thesection.\alph{subsection}}

\newenvironment{problem}[2][Problem]
    { \begin{mdframed}[backgroundcolor=gray!20] \textbf{#1 #2} \\}
    {  \end{mdframed}}

% Define solution environment
\newenvironment{solution}
    {\textit{Solution:}}
    {}

\renewcommand{\qed}{\quad\qedsymbol}
%%%%%%%%%%%%%%%%%%%%%%%%%%%%%%%%%%%%%%%%%%%%%%%%%%%%%%%%%%%%%%%%%%%%%%%%%%%%%%%%%%%%%%%%%%%%%%%%%%%%%%%%%%%%%%%%%%%%%%%%%%%%%%%%%%%%%%%%
\begin{document}
%Header-Make sure you update this information!!!!
\noindent
%%%%%%%%%%%%%%%%%%%%%%%%%%%%%%%%%%%%%%%%%%%%%%%%%%%%%%%%%%%%%%%%%%%%%%%%%%%%%%%%%%%%%%%%%%%%%%%%%%%%%%%%%%%%%%%%%%%%%%%%%%%%%%%%%%%%%%%%
\large\textbf{Tao Li} \hfill \textbf{Homework02}   \\
Email: 2019000153lt@ruc.edu.cn \hfill ID: 2019000153 \\
\normalsize Course: Linear Model   \hfill Term: Spring 2020\\
\noindent\rule{7in}{2.8pt}
%%%%%%%%%%%%%%%%%%%%%%%%%%%%%%%%%%%%%%%%%%%%%%%%%%%%%%%%%%%%%%%%%%%%%%%%%%%%%%%%%%%%%%%%%%%%%%%%%%%%%%%%%%%%%%%%%%%%%%%%%%%%%%%%%%%%%%%%
% Problem 2.24
%%%%%%%%%%%%%%%%%%%%%%%%%%%%%%%%%%%%%%%%%%%%%%%%%%%%%%%%%%%%%%%%%%%%%%%%%%%%%%%%%%%%%%%%%%%%%%%%%%%%%%%%%%%%%%%%%%%%%%%%%%%%%%%%%%%%%%%%
\begin{problem}{2.24}
  Using the tools outlined in the two exercises above, for a design matrix $\mathbf{X}$ we can construct a sequence of Householder 
  matrices $\mathbf{U}_1,...,\mathbf{U}_p$ following (2.13) and (2.14) such that 
  $$\mathbf{U}_p...\mathbf{U}_2\mathbf{U}_1\mathbf{X}=\left[\begin{array}{c}{\mathbf{R}} \\{\mathbf{0}}\end{array}\right]$$
  where $\mathbf{R}$ is square and upper triangular.
  \begin{enumerate}[a.]
    \item Show that $\mathbf{U}_p...\mathbf{U}_2\mathbf{U}_1\mathbf{X}$ is an orthogonal matrix.
    \item Show that $\mathbf{R^TR = X^TX}$.(Note that $\mathbf{R}$ constructed this way may differ from Gram–Schmidt by signs.)
  \end{enumerate}
\end{problem}
\begin{solution}
...
\end{solution}

\noindent\rule{7in}{2.8pt}
%%%%%%%%%%%%%%%%%%%%%%%%%%%%%%%%%%%%%%%%%%%%%%%%%%%%%%%%%%%%%%%%%%%%%%%%%%%%%%%%%%%%%%%%%%%%%%%%%%%%%%%%%%%%%%%%%%%%%%%%%%%%%%%%%%%%%%%%
% Problem 3.2
%%%%%%%%%%%%%%%%%%%%%%%%%%%%%%%%%%%%%%%%%%%%%%%%%%%%%%%%%%%%%%%%%%%%%%%%%%%%%%%%%%%%%%%%%%%%%%%%%%%%%%%%%%%%%%%%%%%%%%%%%%%%%%%%%%%%%%%%
\begin{problem}{3.2}
  Prove the converse to Result 3.2: if the least squares estimator $\lambda^T \mathbf{\hat{b}}$ is the same for 
  all solutions $\mathbf{\hat{b}}$ to the normal equations, then $\lambda^T \mathbf{b}$ is estimable.
\end{problem}
\begin{solution}
...
\end{solution}

\noindent\rule{7in}{2.8pt}
%%%%%%%%%%%%%%%%%%%%%%%%%%%%%%%%%%%%%%%%%%%%%%%%%%%%%%%%%%%%%%%%%%%%%%%%%%%%%%%%%%%%%%%%%%%%%%%%%%%%%%%%%%%%%%%%%%%%%%%%%%%%%%%%%%%%%%%%
% Problem 3.6
%%%%%%%%%%%%%%%%%%%%%%%%%%%%%%%%%%%%%%%%%%%%%%%%%%%%%%%%%%%%%%%%%%%%%%%%%%%%%%%%%%%%%%%%%%%%%%%%%%%%%%%%%%%%%%%%%%%%%%%%%%%%%%%%%%%%%%%%
\begin{problem}{3.6}
  Recall the two-way crossed problem in Section 3.5. 
  \begin{enumerate}[a.]
    \item Show that $\mathbf{G}^T$ is a generalized inverse for $\mathbf{X^TX}$,and then compute both $\mathbf{\left(X^TX\right)G^T}$ and $\mathbf{I-G\left(X^TX\right)}$.
    \item Find the solution $\mathbf{GX^Ty}$ to the normal equations. Also compute the solution $\mathbf{G^TX^Ty}$.
    \item Find the least squares estimators for the following estimable functions:
    \begin{itemize}
      \item $\mu+\alpha_{i}+\beta_{j}$
      \item $\alpha_{i}-\alpha_{k}$
      \item $\beta_{j}-\beta_{k}$
      \item $\sum d_{i} \alpha_{i}$ with $\sum d_{i}=0$
      \item $\sum f_{j} \beta_{j}$ with $\sum f_{j}=0$
    \end{itemize} 
  \end{enumerate}
\end{problem}
\begin{solution}
...
\end{solution}

\noindent\rule{7in}{2.8pt}
%%%%%%%%%%%%%%%%%%%%%%%%%%%%%%%%%%%%%%%%%%%%%%%%%%%%%%%%%%%%%%%%%%%%%%%%%%%%%%%%%%%%%%%%%%%%%%%%%%%%%%%%%%%%%%%%%%%%%%%%%%%%%%%%%%%%%%%%
% Problem 3.11(e)
%%%%%%%%%%%%%%%%%%%%%%%%%%%%%%%%%%%%%%%%%%%%%%%%%%%%%%%%%%%%%%%%%%%%%%%%%%%%%%%%%%%%%%%%%%%%%%%%%%%%%%%%%%%%%%%%%%%%%%%%%%%%%%%%%%%%%%%%
\begin{problem}{3.11(e)}
  Consider the following two-way crossed ANOVA model with interaction and replication:
  $$
  y_{ijk}=\mu+\alpha_{i}+\beta_{j}+\gamma_{ij}+e_{ijk}
  $$
  for $i = 1,...,a$,$j = 1,...,b$, and $k = 1,...,n_{ij} \geq 1$.First simplify matters with $a = 2, b = 3$, and $n_{ij} = n$.
  \begin{enumerate}[a.]
    \item Write this as a linear model by writing $\mathbf{Xb}$.
    \item Find $rank(\mathbf{X}) = r$.
    \item Find a set of $r$ linearly independent estimable functions.
    \item Give a set of $p - r$ jointly nonestimable functions.
  \end{enumerate}
\end{problem}
\begin{solution}
...
\end{solution}
\noindent\rule{7in}{2.8pt}
\end{document}
