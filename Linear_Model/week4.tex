\documentclass[a4paper, 11pt]{article}
\usepackage{comment} % enables the use of multi-line comments (\ifx \fi) 
\usepackage{lipsum} %This package just generates Lorem Ipsum filler text. 
\usepackage{fullpage} % changes the margin
\usepackage[a4paper, total={7in, 10in}]{geometry}
\usepackage[fleqn]{amsmath}
\usepackage{amssymb,amsthm}  % assumes amsmath package installed
\newtheorem{theorem}{Theorem}
\newtheorem{corollary}{Corollary}
\usepackage{graphicx}
\usepackage{tikz}
\usetikzlibrary{arrows}
\usepackage{verbatim}
\renewcommand{\E}{\mathbb{E}}
\renewcommand{\P}{\mathbb{P}}
\renewcommand{\Var}{\mathrm{Var}}
\renewcommand{\Cov}{\mathrm{Cov}}
\usepackage[numbered]{mcode}
\usepackage{float}
\usepackage{tikz}
    \usetikzlibrary{shapes,arrows}
    \usetikzlibrary{arrows,calc,positioning}

    \tikzset{
        block/.style = {draw, rectangle,
            minimum height=1cm,
            minimum width=1.5cm},
        input/.style = {coordinate,node distance=1cm},
        output/.style = {coordinate,node distance=4cm},
        arrow/.style={draw, -latex,node distance=2cm},
        pinstyle/.style = {pin edge={latex-, black,node distance=2cm}},
        sum/.style = {draw, circle, node distance=1cm},
    }
\usepackage{xcolor}
\usepackage{mdframed}
\usepackage[shortlabels]{enumitem}
\usepackage{indentfirst}
\usepackage{hyperref}
    
\renewcommand{\thesubsection}{\thesection.\alph{subsection}}

\newenvironment{problem}[2][Problem]
    { \begin{mdframed}[backgroundcolor=gray!20] \textbf{#1 #2} \\}
    {  \end{mdframed}}

% Define solution environment
\newenvironment{solution}
    {\textit{Solution:}}
    {}

\renewcommand{\qed}{\quad\qedsymbol}
%%%%%%%%%%%%%%%%%%%%%%%%%%%%%%%%%%%%%%%%%%%%%%%%%%%%%%%%%%%%%%%%%%%%%%%%%%%%%%%%%%%%%%%%%%%%%%%%%%%%%%%%%%%%%%%%%%%%%%%%%%%%%%%%%%%%%%%%
\begin{document}
%Header-Make sure you update this information!!!!
\noindent
%%%%%%%%%%%%%%%%%%%%%%%%%%%%%%%%%%%%%%%%%%%%%%%%%%%%%%%%%%%%%%%%%%%%%%%%%%%%%%%%%%%%%%%%%%%%%%%%%%%%%%%%%%%%%%%%%%%%%%%%%%%%%%%%%%%%%%%%
\large\textbf{...} \hfill \textbf{Homework - 4\#}   \\
Email: ... \hfill ID: ... \\
\normalsize ... \hfill ...\\
Instructor: ...\hfill Due Date:2020-3-15 \\
\noindent\rule{7in}{2.8pt}

%%%%%%%%%%%%%%%%%%%%%%%%%%%%%%%%%%%%%%%%%%%%%%%%%%%%%%%%%%%%%%%%%%%%%%%%%%%%%%%%%%%%%%%%%%%%%%%%%%%%%%%%%%%%%%%%%%%%%%%%%%%%%%%%%%%%%%%%
% Problem 1
%%%%%%%%%%%%%%%%%%%%%%%%%%%%%%%%%%%%%%%%%%%%%%%%%%%%%%%%%%%%%%%%%%%%%%%%%%%%%%%%%%%%%%%%%%%%%%%%%%%%%%%%%%%%%%%%%%%%%%%%%%%%%%%%%%%%%%%%
\begin{problem}{5.2}
Let $\mathbf{Z} \sim N_{p}\left(\mathbf{0}, \mathbf{I}_{p}\right),$ and let $\mathbf{A}$ be a $p \times p$ matrix such that $\mathbf{A A}^{T}=\mathbf{V}$.

a. Show that if $\mathbf{Q}$ is an orthogonal matrix, then $\mathbf{Q Z} \sim N_{p}\left(\mathbf{0}, \mathbf{I}_{p}\right)$

b. Show that $(\mathbf{A Q})(\mathbf{A Q})^{T}=\mathbf{V}$

c. Show that $\mathbf{X}=\mu+\mathbf{A Q Z} \sim N_{p}(\mu, \mathbf{V})$
\end{problem}
\begin{solution}
...
\end{solution} 

\noindent\rule{7in}{2.8pt}

%%%%%%%%%%%%%%%%%%%%%%%%%%%%%%%%%%%%%%%%%%%%%%%%%%%%%%%%%%%%%%%%%%%%%%%%%
% Problem 2
%%%%%%%%%%%%%%%%%%%%%%%%%%%%%%%%%%%%%%%%%%%%%%%%%%%%%%%%%%%%%%%%%%%%%%%%%%%%%%%%%%%%%%%%%%%%%%%%%%%%%%%%%%%%%%%%%%%%%%%%%%%%%%%%%%%%%%%%

\begin{problem}{5.8}
(Variation on Result 5.11) Show that if $U \sim \chi_{p}^{2}(\phi),$ then $\operatorname{Pr}(U>c)$ is increasing in $p$ for fixed $\phi \geq 0, c>0$
\end{problem}
\begin{solution}
...
\end{solution} 

\noindent\rule{7in}{2.8pt}
%%%%%%%%%%%%%%%%%%%%%%%%%%%%%%%%%%%%%%%%%%%%%%%%%%%%%%%%%%%%%%%%%%%%%%%%%
% Problem 3
%%%%%%%%%%%%%%%%%%%%%%%%%%%%%%%%%%%%%%%%%%%%%%%%%%%%%%%%%%%%%%%%%%%%%%%%%%%%%%%%%%%%%%%%%%%%%%%%%%%%%%%%%%%%%%%%%%%%%%%%%%%%%%%%%%%%%%%%

\begin{problem}{5.14}
Using the result of Exercise $5.12,$ show that

a. $\operatorname{Var}\left(\mathbf{X}^{T} \mathbf{A X}\right)=2 \operatorname{tr}(\mathbf{A V})^{2}+4 \mu^{T} \mathbf{A V A} \mu$

b. (Easier) If $\mathbf{X} \sim N_{p}(\mathbf{0}, \mathbf{V}),$ then $\operatorname{Var}\left(\mathbf{X}^{T} \mathbf{A} \mathbf{X}\right)=2 \operatorname{tr}(\mathbf{A} \mathbf{V})^{2}$
\end{problem}
\begin{solution}
...
\end{solution} 

\noindent\rule{7in}{2.8pt}
%%%%%%%%%%%%%%%%%%%%%%%%%%%%%%%%%%%%%%%%%%%%%%%%%%%%%%%%%%%%%%%%%%%%%%%%%
% Problem 4
%%%%%%%%%%%%%%%%%%%%%%%%%%%%%%%%%%%%%%%%%%%%%%%%%%%%%%%%%%%%%%%%%%%%%%%%%%%%%%%%%%%%%%%%%%%%%%%%%%%%%%%%%%%%%%%%%%%%%%%%%%%%%%%%%%%%%%%%

\begin{problem}{5.15}
Prove the converse to Result 5.14: Let $\mathbf{X} \sim N_{p}(\mu, \mathbf{V})$ and $\mathbf{A}$ is symmetric; then if $\mathbf{X}^{T} \mathbf{A} \mathbf{X} \sim \chi_{s}^{2}(\phi)$ for some $\phi,$ then $\mathbf{A V}$ is idempotent with rank $s$ .
\end{problem}
\begin{solution}
...
\end{solution} 

\noindent\rule{7in}{2.8pt}
%%%%%%%
% Problem 5
%%%%%%%%%%%%%%%%%%%%%%%%%%%%%%%%%%%%%%%%%%%%%%%%%%%%%%%%%%%%%%%%%%%%%%%%%%%%%%%%%%%%%%%%%%%%%%%%%%%%%%%%%%%%%%%%%%%%%%%%%%%%%%%%%%%%%%%%

\begin{problem}{5.24}
In the normal linear model, that is, $\mathbf{y} \sim N_{N}\left(\mathbf{X b}, \sigma^{2} \mathbf{I}_{N}\right),$ find the conditional distribution of $\mathbf{a}^{T} \mathbf{y}$ given $\mathbf{X}^{T} \mathbf{y} .$ For simplicity, assume that $\mathbf{X}$ has full-column rank.
\end{problem}
\begin{solution}
...
\end{solution} 

\noindent\rule{7in}{2.8pt}
\end{document}