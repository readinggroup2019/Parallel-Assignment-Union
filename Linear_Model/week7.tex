\documentclass[a4paper]{article}

%% Language and font encodings
\usepackage[english]{babel}
\usepackage[utf8x]{inputenc}
\usepackage[T1]{fontenc}
\usepackage{wrapfig}
\usepackage{subcaption}
\usepackage{graphics}
\usepackage{booktabs}
\usepackage{multirow}
\usepackage[table]{xcolor}
\usepackage{amsmath}
\usepackage{amsthm}
\usepackage{amsfonts}
\usepackage{tweaklist}
\usepackage{blindtext}
%% Useful packages
\usepackage{amsmath}
\usepackage{graphicx}
\usepackage[colorinlistoftodos]{todonotes}
\usepackage[colorlinks=true, allcolors=blue]{hyperref}
\usepackage{xeCJK}
%\usepackage{emumerates}
%% Sets page size and margins
\usepackage[a4paper,top=2cm,bottom=2cm,left=1cm,right=3cm,marginparwidth=2cm]{geometry}

\begin{document}
\setlength{\leftskip}{20pt}
\title{ Assignments2}
\author{Maoyu Zhang,2019000157}
%%%%% ------------------------ %%%%% 

 \maketitle

% \begin{abstract}
% \end{abstract}
% \tableofcontents

%%%%--------------------------------------------------
% \clearpage



\begin{enumerate} %starts the numbering

\item[*] 
In one way ANOVA model, compute $(X^TX+C^TC)^{-1}X^T$ 

 {\bf Solution:}
 
 
 \item[6.11] 
 Consider the simple linear regression problem with $ x_i = i$, and $N = 5$. Find the power of the F-test for testing whether the slope is zero when testing at level $\alpha = 0.05$ and the slope takes values 0.1, 0.2, and 0.3.

 {\bf Solution:}
 


 \item[6.14] 
(a)Prove the usual Cauchy–Schwarz inequality:
\begin{center}
    $\left(\mathbf{u}^{T} \mathbf{w}\right)^{2} \leq\|\mathbf{u}\| \times\|\mathbf{w}\|$
\end{center}
by finding the scalar $\alpha$ that minimizes $\|\mathbf{u}-\alpha \mathbf{w}\|^2 $and examining the case where the minimum is zero.

(b)
Obtain the generalized Cauchy–Schwarz inequality (6.25) by applying
the same steps to
\begin{center}
    $\left(\mathbf{u}-\alpha \mathbf{A}^{-1} \mathbf{w}\right)^{T} \mathbf{A}\left(\mathbf{u}-\alpha \mathbf{A}^{-1} \mathbf{w}\right)$
\end{center}

 {\bf Solution:}
 
 
 
 \item[6.18] 
 For the one-way ANOVA model with $n_i = n = 5$ and $a = 3$, compute the expected length (or squared length) of simultaneous confidence intervals computed using Bonferroni, Scheffe, and Tukey’s methods for all pairwise differences$\alpha_i-\alpha_k$.
 
 {\bf Solution:}
 

 
\end{enumerate}
% ends the numbering
% -----------------------------------Appendix----------------------------------------
%\newpage
% -----------------------------------REFERENCE----------------------------------------
 \bibliographystyle{alpha}
% \bibliography{sample}
\end{document}
