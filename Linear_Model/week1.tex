\documentclass[a4paper, 11pt]{article}
\usepackage{comment} % enables the use of multi-line comments (\ifx \fi)
\usepackage{lipsum} %This package just generates Lorem Ipsum filler text.
\usepackage{fullpage} % changes the margin
\usepackage[a4paper, total={7in, 10in}]{geometry}
\usepackage[fleqn]{amsmath}
\usepackage{amssymb,amsthm}  % assumes amsmath package installed
\newtheorem{theorem}{Theorem}
\newtheorem{corollary}{Corollary}
\usepackage{graphicx}
\usepackage{tikz}
\usetikzlibrary{arrows}
\usepackage{verbatim}
\usepackage[numbered]{mcode}
\usepackage{float}
\usepackage{tikz}
    \usetikzlibrary{shapes,arrows}
    \usetikzlibrary{arrows,calc,positioning}

    \tikzset{
        block/.style = {draw, rectangle,
            minimum height=1cm,
            minimum width=1.5cm},
        input/.style = {coordinate,node distance=1cm},
        output/.style = {coordinate,node distance=4cm},
        arrow/.style={draw, -latex,node distance=2cm},
        pinstyle/.style = {pin edge={latex-, black,node distance=2cm}},
        sum/.style = {draw, circle, node distance=1cm},
    }
\usepackage{xcolor}
\usepackage{mdframed}
\usepackage[shortlabels]{enumitem}
\usepackage{indentfirst}
\usepackage{hyperref}

\renewcommand{\thesubsection}{\thesection.\alph{subsection}}

\newenvironment{problem}[2][Problem]
    { \begin{mdframed}[backgroundcolor=gray!20] \textbf{#1 #2} \\}
    {  \end{mdframed}}

% Define solution environment
\newenvironment{solution}
    {\textit{Solution:}}
    {}

\renewcommand{\qed}{\quad\qedsymbol}
%%%%%%%%%%%%%%%%%%%%%%%%%%%%%%%%%%%%%%%%%%%%%%%%%%%%%%%%%%%%%%%%%%%%%%%%%%%%%%%%%%%%%%%%%%%%%%%%%%%%%%%%%%%%%%%%%%%%%%%%%%%%%%%%%%%%%%%%
\begin{document}
%Header-Make sure you update this information!!!!
\noindent
%%%%%%%%%%%%%%%%%%%%%%%%%%%%%%%%%%%%%%%%%%%%%%%%%%%%%%%%%%%%%%%%%%%%%%%%%%%%%%%%%%%%%%%%%%%%%%%%%%%%%%%%%%%%%%%%%%%%%%%%%%%%%%%%%%%%%%%%
\large\textbf{Jiawei Shan} \hfill \textbf{Homework - \#}   \\
Email: jwshan@ruc.edu.cn \hfill ID: 2019000151 \\
\normalsize Course: Linear Model   \hfill Term: Spring 2020\\

\noindent\rule{7in}{2.8pt}
%%%%%%%%%%%%%%%%%%%%%%%%%%%%%%%%%%%%%%%%%%%%%%%%%%%%%%%%%%%%%%%%%%%%%%%%%%%%%%%%%%%%%%%%%%%%%%%%%%%%%%%%%%%%%%%%%%%%%%%%%%%%%%%%%%%%%%%%
% Problem 34
%%%%%%%%%%%%%%%%%%%%%%%%%%%%%%%%%%%%%%%%%%%%%%%%%%%%%%%%%%%%%%%%%%%%%%%%%%%%%%%%%%%%%%%%%%%%%%%%%%%%%%%%%%%%%%%%%%%%%%%%%%%%%%%%%%%%%%%%
\begin{problem}{34}
  Let $\mathbf{A}=\left[\begin{array}{cccc}{8} & {4} & {2} & {2} \\ {4} & {4} & {0} & {0} \\ {2} & {0} & {2} & {0} \\ {2} & {0} & {0} & {2}\end{array}\right] \quad$ and $c=\left[\begin{array}{c}{1} \\ {-2} \\ {-1} \\ {4}\end{array}\right]$
  \begin{enumerate}[a.]
    \item Show that $\mathbf{c}$ is in $\mathcal{C}(\mathbf{A}) .$ (Find a vector $\mathbf{x}$ such that $\mathbf{A x}=\mathbf{c} .$ )
    \item Find two different generalized inverses for $\mathbf{A}$.
    \item For one of your generalized inverses in (b), compute $AA^{g}$
    \item Is $\mathbf{A} \mathbf{A}^{g}$ from (c) idempotent? Symmetric?
    \item Show that $\mathbf{A} \mathbf{A}^{g} \mathbf{c}=\mathbf{c}$
    \item For your two generalized inverses in (b), compute $\mathbf{A}^{g} \mathbf{c}$ and show that each vector solves $\mathbf{A x}=\mathbf{c}$
  \end{enumerate}
\end{problem}
\begin{solution}
...
\end{solution}
\noindent\rule{7in}{2.8pt}


%%%%%%%%%%%%%%%%%%%%%%%%%%%%%%%%%%%%%%%%%%%%%%%%%%%%%%%%%%%%%%%%%%%%%%%%%%%%%%%%%%%%%%%%%%%%%%%%%%%%%%%%%%%%%%%%%%%%%%%%%%%%%%%%%%%%%%%%
% Problem 49
%%%%%%%%%%%%%%%%%%%%%%%%%%%%%%%%%%%%%%%%%%%%%%%%%%%%%%%%%%%%%%%%%%%%%%%%%%%%%%%%%%%%%%%%%%%%%%%%%%%%%%%%%%%%%%%%%%%%%%%%%%%%%%%%%%%%%%%%
\begin{problem}{49}
 Show that $\mathbf{A}\mathbf{A}^{T}$ and $\mathbf{A}^{T}\mathbf{A}$  have the same nonzero eigenvalues.
\end{problem}
\begin{solution}
...
\end{solution}
\noindent\rule{7in}{2.8pt}


%%%%%%%%%%%%%%%%%%%%%%%%%%%%%%%%%%%%%%%%%%%%%%%%%%%%%%%%%%%%%%%%%%%%%%%%%%%%%%%%%%%%%%%%%%%%%%%%%%%%%%%%%%%%%%%%%%%%%%%%%%%%%%%%%%%%%%%%
% Problem 73
%%%%%%%%%%%%%%%%%%%%%%%%%%%%%%%%%%%%%%%%%%%%%%%%%%%%%%%%%%%%%%%%%%%%%%%%%%%%%%%%%%%%%%%%%%%%%%%%%%%%%%%%%%%%%%%%%%%%%%%%%%%%%%%%%%%%%%%%
\begin{problem}{73}
  Let $\mathbf{V}$ be a symmetric $p \times p$ matrix with eigenvalues $\lambda_{1}, \cdots, \lambda_{p} ;$ show the following:
  \begin{enumerate}[a.]
    \item $\left|\mathbf{I}_{p}+t \mathbf{V}\right|=\prod_{i=1}^{p}\left(1+t \lambda_{i}\right)$
    \item The derivative of $\left|\mathbf{I}_{p}+t \mathbf{V}\right|$ with respect to $t$ is equal to trace$(\mathbf{V})$
  \end{enumerate}
\end{problem}
\begin{solution}
...
\end{solution}
\noindent\rule{7in}{2.8pt}


%%%%%%%%%%%%%%%%%%%%%%%%%%%%%%%%%%%%%%%%%%%%%%%%%%%%%%%%%%%%%%%%%%%%%%%%%%%%%%%%%%%%%%%%%%%%%%%%%%%%%%%%%%%%%%%%%%%%%%%%%%%%%%%%%%%%%%%%
% Problem 75
%%%%%%%%%%%%%%%%%%%%%%%%%%%%%%%%%%%%%%%%%%%%%%%%%%%%%%%%%%%%%%%%%%%%%%%%%%%%%%%%%%%%%%%%%%%%%%%%%%%%%%%%%%%%%%%%%%%%%%%%%%%%%%%%%%%%%%%%
\begin{problem}{75}
  (Binomial inverse theorem) Verify the following for $\mathbf{A},$ B nonsingular:
  $$
  (\mathbf{A}+\mathbf{U B V})^{-1}=\mathbf{A}^{-1}-\mathbf{A}^{-1} \mathbf{U B}^{-1}\left(\mathbf{B}^{-1}+\mathbf{V A}^{-1} \mathbf{U}\right)^{-1} \mathbf{B V A}^{-1}
  $$
\end{problem}
\begin{solution}
...
\end{solution}
\noindent\rule{7in}{2.8pt}
% \begin{figure}[H]
%     \centering
%     \includegraphics[scale=0.25]{q2.png}
%     \caption{Plot showing $\hat{\phi}$ as a function of time.}
%     \label{fig_q2l}
% \end{figure}

% \lstinputlisting{HW6Q2.m}
%%%%%%%%%%%%%%%%%%%%%%%%%%%%%%%%%%%%%%%%%%%%%%%%%%%%%%%%%%%%%%%%%%%%%%%%%
\end{document}
