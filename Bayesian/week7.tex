\documentclass[a4paper, 11pt]{article}
\usepackage{comment} % enables the use of multi-line comments (\ifx \fi)
\usepackage{lipsum} %This package just generates Lorem Ipsum filler text.
\usepackage{fullpage} % changes the margin
\usepackage[a4paper, total={7in, 10in}]{geometry}
\usepackage[fleqn]{amsmath}
\usepackage{amssymb,amsthm}  % assumes amsmath package installed
\newtheorem{theorem}{Theorem}
\newtheorem{corollary}{Corollary}
\usepackage{graphicx}
\usepackage{tikz}
\usetikzlibrary{arrows}
\usepackage{verbatim}
\usepackage[numbered]{mcode}
\usepackage{float}
\usepackage{tikz}
    \usetikzlibrary{shapes,arrows}
    \usetikzlibrary{arrows,calc,positioning}

    \tikzset{
        block/.style = {draw, rectangle,
            minimum height=1cm,
            minimum width=1.5cm},
        input/.style = {coordinate,node distance=1cm},
        output/.style = {coordinate,node distance=4cm},
        arrow/.style={draw, -latex,node distance=2cm},
        pinstyle/.style = {pin edge={latex-, black,node distance=2cm}},
        sum/.style = {draw, circle, node distance=1cm},
    }
\usepackage{xcolor}
\usepackage{mdframed}
\usepackage[shortlabels]{enumitem}
\usepackage{indentfirst}
\usepackage{hyperref}

\renewcommand{\thesubsection}{\thesection.\alph{subsection}}

\newenvironment{problem}[2][Problem]
    { \begin{mdframed}[backgroundcolor=gray!20] \textbf{#1 #2} \\}
    {  \end{mdframed}}

% Define solution environment
\newenvironment{solution}
    {\textit{Solution:}}
    {}

\renewcommand{\qed}{\quad\qedsymbol}
%%%%%%%%%%%%%%%%%%%%%%%%%%%%%%%%%%%%%%%%%%%%%%%%%%%%%%%%%%%%%%%%%%%%%%%%%%%%%%%%%%%%%%%%%%%%%%%%%%%%%%%%%%%%%%%%%%%%%%%%%%%%%%%%%%%%%%%%
\begin{document}
%Header-Make sure you update this information!!!!
\noindent
%%%%%%%%%%%%%%%%%%%%%%%%%%%%%%%%%%%%%%%%%%%%%%%%%%%%%%%%%%%%%%%%%%%%%%%%%%%%%%%%%%%%%%%%%%%%%%%%%%%%%%%%%%%%%%%%%%%%%%%%%%%%%%%%%%%%%%%%
\large\textbf{Yonghua Su} \hfill \textbf{Homework - 7}   \\
Email: 2019000154@ruc.edu.cn \hfill ID: 2019000154 \\
\normalsize Course: Bayesian \hfill Term: Spring 2020\\
Instructor: Dr. Luo  \hfill Due Date: $8^{st}$ Apr., 2020 \\
\noindent\rule{7in}{2.8pt}


%%%%%%%%%%%%%%%%%%%%%%%%%%%%%%%%%%%%%%%%%%%
%%%%%%%%%%%%%%%%%%%%%%%%%%%%%%%%%%%%%%%%%%%

\begin{problem}{10.3}
  Posterior computations for the binomial model: suppose $y_1 ∼ \text{Bin}(n_1, p_1)$ is the number of successfully treated patients under an experimental new drug, and $y_2 ∼ \text{Bin}(n_2,p_2)$
  is the number of successfully treated patients under the standard treatment. Assume that $y_1$ and $y_2$ are independent and assume independent beta prior densities for the two probabilities of success. 
  Let $n_1 = 10, y_1 = 6$, and $n_2 = 20, y_2 = 10$. Repeat the following for several different beta prior specifications.

  (a) Use simulation to find a 95\% posterior interval for $p_1 −p_2$ and the posterior probability that $p_1 > p_2$.

  (b) Numerically integrate to estimate the posterior probability that $p_1 > p_2$.

\end{problem}
\begin{solution}
...
\end{solution}

\noindent\rule{7in}{2.8pt}
%%%%%%%%%%%%%%%%%%%%%%%%%%%%%%%%%%%%%%%%%%%
%%%%%%%%%%%%%%%%%%%%%%%%%%%%%%%%%%%%%%%%%%%

\begin{problem}{10.4}
  Rejection sampling:

  (a) Prove that rejection sampling gives draws from $p(\theta|y)$.

  (b) Why is the boundedness condition on $p(\theta|y)/q(\theta)$ necessary for rejection sampling?
  
\end{problem}
\begin{solution}
...
\end{solution}

\noindent\rule{7in}{2.8pt}

%%%%%%%%%%%%%%%%%%%%%%%%%%%%%%%%%%%%%%%%%%%
%%%%%%%%%%%%%%%%%%%%%%%%%%%%%%%%%%%%%%%%%%%

\begin{problem}{11.1}
  Metropolis-Hastings algorithm: 
  Show that the stationary distribution for the Metropolis- Hastings algorithm is, in fact, the target distribution, $p(\theta|y)$.
\end{problem}
\begin{solution}
...
\end{solution}

\noindent\rule{7in}{2.8pt}
%%%%%%%%%%%%%%%%%%%%%%%%%%%%%%%%%%%%%%%%%%%
%%%%%%%%%%%%%%%%%%%%%%%%%%%%%%%%%%%%%%%%%%%


\begin{problem}{11.3}
  Gibbs sampling: Table 11.4 contains quality control measurements from 6 machines in a factory. 
  Quality control measurements are expensive and time-consuming, so only 5 measurements were done for each machine. 
  In addition to the existing machines, we are interested in the quality of another machine (the seventh machine). 
  Implement a separate, a pooled and hierarchical Gaussian model with common variance described in Section 11.6. 
  Run the simulations long enough for approximate convergence. Using each of three models—separate, pooled, 
  and hierarchical—report: 
  (i) the posterior distribu- tion of the mean of the quality measurements of the sixth machine, 
  (ii) the predictive distribution for another quality measurement of the sixth machine, and 
  (iii) the posterior distribution of the mean of the quality measurements of the seventh machine.

  \begin{tabular}{cc} 
    Machine & Measurements \\
    \hline 1 & 83,92,92,46,67 \\
    2 & 117,109,114,104,87 \\
    3 & 101,93,92,86,67 \\
    4 & 105,119,116,102,116 \\
    5 & 79,97,103,79,92 \\
    6 & 57,92,104,77,100
    \end{tabular} 
    
  Table 11.4: Quality control measurements from 6 machines in a factory.
\end{problem}
\begin{solution}

\end{solution}

\noindent\rule{7in}{2.8pt}
%%%%%%%%%%%%%%%%%%%%%%%%%%%%%%%%%%%%%%%%%%%
%%%%%%%%%%%%%%%%%%%%%%%%%%%%%%%%%%%%%%%%%%%
\begin{problem}{11.6}
  Effective sample size:

  (a) Derive the asymptotic formula (11.5) for the variance of the average of correlated simulations.
\end{problem}
\begin{solution}
...
\end{solution}

\noindent\rule{7in}{2.8pt}

%%%%%%%%%%%%%%%%%%%%%%%%%%%%%%%%%%%%%%%%%%%
%%%%%%%%%%%%%%%%%%%%%%%%%%%%%%%%%%%%%%%%%%%
\end{document}
