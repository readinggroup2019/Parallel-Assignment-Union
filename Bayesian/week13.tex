\documentclass[a4paper, 11pt]{article}
\usepackage{comment} % enables the use of multi-line comments (\ifx \fi) 
\usepackage{lipsum} %This package just generates Lorem Ipsum filler text. 
\usepackage{fullpage} % changes the margin
\usepackage[a4paper, total={7in, 10in}]{geometry}
\usepackage[fleqn]{amsmath}
\usepackage{amssymb,amsthm}  % assumes amsmath package installed
\newtheorem{theorem}{Theorem}
\newtheorem{corollary}{Corollary}
\usepackage{graphicx}
\usepackage{tikz}
\usetikzlibrary{arrows}
\usepackage{verbatim}
\usepackage{float}
\usepackage{tikz}
    \usetikzlibrary{shapes,arrows}
    \usetikzlibrary{arrows,calc,positioning}

    \tikzset{
        block/.style = {draw, rectangle,
            minimum height=1cm,
            minimum width=1.5cm},
        input/.style = {coordinate,node distance=1cm},
        output/.style = {coordinate,node distance=4cm},
        arrow/.style={draw, -latex,node distance=2cm},
        pinstyle/.style = {pin edge={latex-, black,node distance=2cm}},
        sum/.style = {draw, circle, node distance=1cm},
    }
\usepackage{xcolor}
\usepackage{mdframed}
\usepackage[shortlabels]{enumitem}
\usepackage{indentfirst}
\usepackage{hyperref}
\usepackage{mathrsfs}
\renewcommand{\thesubsection}{\thesection.\alph{subsection}}

\newenvironment{problem}[2][Problem]
    { \begin{mdframed}[backgroundcolor=gray!20] \textbf{#1 #2} \\}
    {  \end{mdframed}}

% Define solution environment
\newenvironment{solution}
    {\textit{Solution:}}
    {}

\renewcommand{\qed}{\quad\qedsymbol}
%%%%%%%%%%%%%%%%%%%%%%%%%%%%%%%%%%%%%%%%%%%%%%%%%%%%%%%%%%%%%%%%%%%%%%%%%%%%%%%%%%%%%%%%%%%%%%%%%%%%%%%%%%%%%%%%%%%%%%%%%%%%%%%%%%%%%%%%
\begin{document}
%Header-Make sure you update this information!!!!
\noindent
%%%%%%%%%%%%%%%%%%%%%%%%%%%%%%%%%%%%%%%%%%%%%%%%%%%%%%%%%%%%%%%%%%%%%%%%%%%%%%%%%%%%%%%%%%%%%%%%%%%%%%%%%%%%%%%%%%%%%%%%%%%%%%%%%%%%%%%%
\large\textbf{Maoyu Zhang} \hfill \textbf{Homework - 13}   \\
Email:2019000157@ruc.edu.cn  \hfill ID: 2019000157 \\
\normalsize Course: Bayesian \hfill Term: Spring 2020\\
Instructor: Dr. He \hfill Due Date: $20^{th}$ May, 2020 \\
\noindent\rule{7in}{2.8pt}
%%%%%%%%%%%%%%%%%%%%%%%%%%%%%%%%%%%%%%%%%%%
%%%%%%%%%%%%%%%%%%%%%%%%%%%%%%%%%%%%%%%%%%%
%%%%%%%%%%%%%%%%%%%%%%%%%%%%%%%%%%%%%%%%%%%
%%%%%%%%%%%%%%%%%%%%%%%%%%%%%%%%%%%%%%%%%%%

\begin{problem}{7.4}
For the transition kernel,
$$
X^{(t+1)} | x^{(t)} \sim \mathcal{N}\left(\rho x^{(t)}, \tau^{2}\right)
$$
gives sufficient conditions on $\rho$ and $\tau$ for the stationary distribution $\pi$ to exist.
Show that, in this case, 1r is a normal distribution and that (7.4) occurs.
\end{problem}
\begin{solution}

\end{solution}

\noindent\rule{7in}{2.8pt}
%%%%%%%%%%%%%%%%%%%%%%%%%%%%%%%%%%%%%%%%%%%
%%%%%%%%%%%%%%%%%%%%%%%%%%%%%%%%%%%%%%%%%%%
\begin{problem}{7.5}
(Doukhan et al. 1994) The algorithm presented in this problem is used in Chap-
ter 12 as a benchmark for slow convergence.

(a) Prove the following result:

Lemma $7.24 .$ Consider a probability density $g$ on [0,1] and a function $0<$ $\rho<1$ such that
\[
\int_{0}^{1} \frac{g(x)}{1-\rho(x)} d x<\infty
\]
The Markov chain with transition kernel
\[
K\left(x, x^{\prime}\right)=\rho(x) \delta_{x}\left(x^{\prime}\right)+(1-\rho(x)) g\left(x^{\prime}\right)
\]
where $\delta_{x}$ is the Dirac mass at $x,$ has stationary distribution
\[
f(x) \propto g(x) /(1-\rho(x))
\]

(b)Show that an algorithm for generating the Markov chain associated with
Lemma 7.24 is given by

Algorithm A.30 -Repeat or Simulate-
1. Take $X^{(t+1)}=x^{(t)}$ with probability $\rho\left(x^{(t)}\right)$
2. Else, generate $X^{(t+1)} \sim g(y)$

(c) Highlight the similarity with the Accept-Reject algorithm and discuss in which sense they are complementary.



\end{problem}
\begin{solution}

\end{solution}

\noindent\rule{7in}{2.8pt}
%%%%%%%%%%%%%%%%%%%%%%%%%%%%%%%%%%%%%%%%%%%
%%%%%%%%%%%%%%%%%%%%%%%%%%%%%%%%%%%%%%%%%%%
\begin{problem}{7.10}

The inequality (7.8) can also be established using Orey's inequality (See Problem 6.42 for a slightly different formulation) For two transitions $P$ and $Q$
\[
\left\|P^{n}-Q^{n}\right\|_{T V} \leq 2 P\left(X_{n} \neq Y_{n}\right), \quad X_{n} \sim P^{n}, \quad Y_{n} \sim Q^{n}
\]
Deduce that when $P$ is associated with the stationary distribution $f$ and when $X_{n}$ is generated by $[A .25],$ under the condition (7.7)
\[
\left\|P^{n}-f\right\|_{T V} \leq\left(1-\frac{1}{M}\right)^{n}
\]
Hint: Use a coupling argument based on
\[
X^{n}=\left\{\begin{array}{ll}
Y^{n} & \text { with probability } 1 / M \\
Z^{n} \sim \frac{g(z)-f(z) / M}{1-1 / M} & \text { otherwise }
\end{array}\right.
\]
\end{problem}
\begin{solution}

\end{solution}

\noindent\rule{7in}{2.8pt}
%%%%%%%%%%%%%%%%%%%%%%%%%%%%%%%%%%%%%%%%%%%
%%%%%%%%%%%%%%%%%%%%%%%%%%%%%%%%%%%%%%%%%%%


\end{document}
