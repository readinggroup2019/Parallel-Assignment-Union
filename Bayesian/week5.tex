\documentclass[a4paper, 11pt]{article}
\usepackage{comment} % enables the use of multi-line comments (\ifx \fi)
\usepackage{lipsum} %This package just generates Lorem Ipsum filler text.
\usepackage{fullpage} % changes the margin
\usepackage[a4paper, total={7in, 10in}]{geometry}
\usepackage[fleqn]{amsmath}
\usepackage{amssymb,amsthm}  % assumes amsmath package installed
\newtheorem{theorem}{Theorem}
\newtheorem{corollary}{Corollary}
\usepackage{graphicx}
\usepackage{tikz}
\usetikzlibrary{arrows}
\usepackage{verbatim}
\newcommand{\E}{\mathbb{E}}
\renewcommand{\P}{\mathbb{P}}
\newcommand{\Var}{\mathrm{Var}}
\newcommand{\Cov}{\mathrm{Cov}}
\usepackage[numbered]{mcode}
\usepackage{float}
\usepackage{tikz}
    \usetikzlibrary{shapes,arrows}
    \usetikzlibrary{arrows,calc,positioning}

    \tikzset{
        block/.style = {draw, rectangle,
            minimum height=1cm,
            minimum width=1.5cm},
        input/.style = {coordinate,node distance=1cm},
        output/.style = {coordinate,node distance=4cm},
        arrow/.style={draw, -latex,node distance=2cm},
        pinstyle/.style = {pin edge={latex-, black,node distance=2cm}},
        sum/.style = {draw, circle, node distance=1cm},
    }
\usepackage{xcolor}
\usepackage{mdframed}
\usepackage[shortlabels]{enumitem}
\usepackage{indentfirst}
\usepackage{hyperref}

\renewcommand{\thesubsection}{\thesection.\alph{subsection}}

\newenvironment{problem}[2][Problem]
    { \begin{mdframed}[backgroundcolor=gray!20] \textbf{#1 #2} \\}
    {  \end{mdframed}}

% Define solution environment
\newenvironment{solution}
    {\textit{Solution:}}
    {}

\renewcommand{\qed}{\quad\qedsymbol}
%%%%%%%%%%%%%%%%%%%%%%%%%%%%%%%%%%%%%%%%%%%%%%%%%%%%%%%%%%%%%%%%%%%%%%%%%%%%%%%%%%%%%%%%%%%%%%%%%%%%%%%%%%%%%%%%%%%%%%%%%%%%%%%%%%%%%%%%
\begin{document}
%Header-Make sure you update this information!!!!
\noindent

%%%%%%%%%%%%%%%%%%%%%%%%%%%%%%%%%%%%%%%%%%%%%%%%%%%%%%%%%%%%%%%%%%%%%%%%%%%%%%%%%%%%%%%%%%%%%%%%%%%%%%%%%%%%%%%%%%%%%%%%%%%%%%%%%%%%%%%%
\large\textbf{Tao Li} \hfill \textbf{Homework05}   \\
Email:201900153lt@ruc.edu.cn  \hfill ID: 2019000153\\
\normalsize Course: Linear Model   \hfill Term: Spring 2020\\
\noindent\rule{7in}{2.8pt}

%%%%%%%%%%%%%%%%%%%%%%%%%%%%%%%%%%%%%%%%%%%%%%%%%%%%%%%%%%%%%%%%%%%%%%%%%%%%%%%%%%%%%%%%%%%%%%%%%%%%%%%%%%%%%%%%%%%%%%%%%%%%%%%%%%%%%%%%
% Problem 1
%%%%%%%%%%%%%%%%%%%%%%%%%%%%%%%%%%%%%%%%%%%%%%%%%%%%%%%%%%%%%%%%%%%%%%%%%%%%%%%%%%%%%%%%%%%%%%%%%%%%%%%%%%%%%%%%%%%%%%%%%%%%%%%%%%%%%%%%
\begin{problem}{1.1}
Conditional probability: suppose that if $\theta = 1$, then $y$ has a normal distribution with mean 1 and standard deviation $\sigma$, 
and if $\theta = 2$, then $y$ has a normal distribution with mean 2 and standard deviation $\sigma$. 
Also, suppose $\operatorname{Pr}(\theta = 1) = 0.5$ and $\operatorname{Pr}(\theta = 2) = 0.5$.

(a) For $\sigma = 2$, write the formula for the marginal probability density for $y$ and sketch it. 

(b) What is $\operatorname{Pr}(\theta = 1|y = 1)$, again supposing $\sigma = 2$?

(c) Describe how the posterior density of $\theta$ changes in shape as $\sigma$ is increased and as it is decreased.
\end{problem}
\begin{solution}
...
\end{solution}

\noindent\rule{7in}{2.8pt}

%%%%%%%%%%%%%%%%%%%%%%%%%%%%%%%%%%%%%%%%%%%%%%%%%%%%%%%%%%%%%%%%%%%%%%%%%
% Problem 2
%%%%%%%%%%%%%%%%%%%%%%%%%%%%%%%%%%%%%%%%%%%%%%%%%%%%%%%%%%%%%%%%%%%%%%%%%%%%%%%%%%%%%%%%%%%%%%%%%%%%%%%%%%%%%%%%%%%%%%%%%%%%%%%%%%%%%%%%

\begin{problem}{1.2}
Conditional means and variances: show that (1.8) and (1.9) hold if $u$ is a vector.
\end{problem}
\begin{solution}
...
\end{solution}

\noindent\rule{7in}{2.8pt}
%%%%%%%%%%%%%%%%%%%%%%%%%%%%%%%%%%%%%%%%%%%%%%%%%%%%%%%%%%%%%%%%%%%%%%%%%
% Problem 3
%%%%%%%%%%%%%%%%%%%%%%%%%%%%%%%%%%%%%%%%%%%%%%%%%%%%%%%%%%%%%%%%%%%%%%%%%%%%%%%%%%%%%%%%%%%%%%%%%%%%%%%%%%%%%%%%%%%%%%%%%%%%%%%%%%%%%%%%

\begin{problem}{1.6}
Conditional probability: approximately 1/125 of all births are fraternal twins and 1/300 of births are identical twins. 
Elvis Presley had a twin brother (who died at birth). What is the probability that Elvis was an identical twin? 
(You may approximate the probability of a boy or girl birth as $\frac{1}{2}$.
\end{problem}
\begin{solution}
...
\end{solution}

\noindent\rule{7in}{2.8pt}
%%%%%%%%%%%%%%%%%%%%%%%%%%%%%%%%%%%%%%%%%%%%%%%%%%%%%%%%%%%%%%%%%%%%%%%%%
% Problem 4
%%%%%%%%%%%%%%%%%%%%%%%%%%%%%%%%%%%%%%%%%%%%%%%%%%%%%%%%%%%%%%%%%%%%%%%%%%%%%%%%%%%%%%%%%%%%%%%%%%%%%%%%%%%%%%%%%%%%%%%%%%%%%%%%%%%%%%%%

\begin{problem}{1.7}
    Conditional probability: the following problem is loosely based on the television game show \textit{Let’s Make a Deal}. 
    At the end of the show, a contestant is asked to choose one of three large boxes, where one box contains 
    a fabulous prize and the other two boxes contain lesser prizes. After the contestant chooses a box, Monty Hall, 
    the host of the show, opens one of the two boxes containing smaller prizes. 
    (In order to keep the conclusion suspenseful, Monty does not open the box selected by the contestant.)
     Monty offers the contestant the opportunity to switch from the chosen box to the remaining unopened box. 
     Should the contestant switch or stay with the original choice? Calculate the probability that 
     the contestant wins under each strategy. This is an exercise in being clear about the information that 
     should be conditioned on when constructing a probability judgment. 
     See Selvin (1975) and Morgan et al. (1991) for further discussion of this problem.
\end{problem}
\begin{solution}
...
\end{solution}

\noindent\rule{7in}{2.8pt}
%%%%%%%
% Problem 5
%%%%%%%%%%%%%%%%%%%%%%%%%%%%%%%%%%%%%%%%%%%%%%%%%%%%%%%%%%%%%%%%%%%%%%%%%%%%%%%%%%%%%%%%%%%%%%%%%%%%%%%%%%%%%%%%%%%%%%%%%%%%%%%%%%%%%%%%

\begin{problem}{1.8}
Subjective probability: discuss the following statement. ‘The probability of event $E$ is considered “subjective” 
if two rational persons $A$ and $B$ can assign unequal probabilities to $E$, $P_A(E)$ and $P_B(E)$.
These probabilities can also be interpreted as “conditional”: $P_A(E) = P(E|I_A)$ and $P_B(E) = P(E|I_B)$, 
where $I_A$ and $I_B$ represent the knowledge available to persons $A$ and $B$, respectively.’
 Apply this idea to the following examples.

(a) The probability that a ‘6’ appears when a fair die is rolled, where $A$ observes the outcome of the die roll 
and $B$ does not.

(b) The probability that Brazil wins the next World Cup, where $A$ is ignorant of soccer and $B$ is a knowledgeable sports fan.
\end{problem}
\begin{solution}
...
\end{solution}

\noindent\rule{7in}{2.8pt}
%%%%%%%%%%%%%%%%%%%%%%%%%%%%%%%%%%%%%%%%%%%%%%%%%%%%%%%%%%%%%%%%%%%%%%%%%
% Problem 6
%%%%%%%%%%%%%%%%%%%%%%%%%%%%%%%%%%%%%%%%%%%%%%%%%%%%%%%%%%%%%%%%%%%%%%%%%%%%%%%%%%%%%%%%%%%%%%%%%%%%%%%%%%%%%%%%%%%%%%%%%%%%%%%%%%%%%%%%

\begin{problem}{1.9}
Simulation of a queuing problem: a clinic has three doctors. Patients come into the clinic at random, 
starting at 9 a.m., according to a Poisson process with time parameter 10 minutes: that is, 
the time after opening at which the first patient appears follows an exponential distribution with expectation 
10 minutes and then, after each patient arrives, the waiting time until the next patient is independently 
exponentially distributed, also with expectation 10 minutes. When a patient arrives, he or she waits until a doctor 
is available. The amount of time spent by each doctor with each patient is a random variable, uniformly distributed 
between 5 and 20 minutes. The office stops admitting new patients at 4 p.m. and closes when the last patient is 
through with the doctor.

(a) Simulate this process once. How many patients came to the office? How many had to wait for a doctor? What was their average wait? When did the office close?

(b) Simulate the process 100 times and estimate the median and $50\%$ interval for each of the summaries in (a). 
\end{problem}
\begin{solution}
...
\end{solution}

\noindent\rule{7in}{2.8pt}
\end{document}