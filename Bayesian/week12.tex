\documentclass[a4paper, 11pt]{article}
\usepackage{comment} % enables the use of multi-line comments (\ifx \fi) 
\usepackage{lipsum} %This package just generates Lorem Ipsum filler text. 
\usepackage{fullpage} % changes the margin
\usepackage[a4paper, total={7in, 10in}]{geometry}
\usepackage[fleqn]{amsmath}
\usepackage{amssymb,amsthm}  % assumes amsmath package installed
\newtheorem{theorem}{Theorem}
\newtheorem{corollary}{Corollary}
\usepackage{graphicx}
\usepackage{tikz}
\usetikzlibrary{arrows}
\usepackage{verbatim}
\usepackage{float}
\usepackage{tikz}
    \usetikzlibrary{shapes,arrows}
    \usetikzlibrary{arrows,calc,positioning}

    \tikzset{
        block/.style = {draw, rectangle,
            minimum height=1cm,
            minimum width=1.5cm},
        input/.style = {coordinate,node distance=1cm},
        output/.style = {coordinate,node distance=4cm},
        arrow/.style={draw, -latex,node distance=2cm},
        pinstyle/.style = {pin edge={latex-, black,node distance=2cm}},
        sum/.style = {draw, circle, node distance=1cm},
    }
\usepackage{xcolor}
\usepackage{mdframed}
\usepackage[shortlabels]{enumitem}
\usepackage{indentfirst}
\usepackage{hyperref}
\usepackage{mathrsfs}
\renewcommand{\thesubsection}{\thesection.\alph{subsection}}

\newenvironment{problem}[2][Problem]
    { \begin{mdframed}[backgroundcolor=gray!20] \textbf{#1 #2} \\}
    {  \end{mdframed}}

% Define solution environment
\newenvironment{solution}
    {\textit{Solution:}}
    {}

\renewcommand{\qed}{\quad\qedsymbol}
%%%%%%%%%%%%%%%%%%%%%%%%%%%%%%%%%%%%%%%%%%%%%%%%%%%%%%%%%%%%%%%%%%%%%%%%%%%%%%%%%%%%%%%%%%%%%%%%%%%%%%%%%%%%%%%%%%%%%%%%%%%%%%%%%%%%%%%%
\begin{document}
%Header-Make sure you update this information!!!!
\noindent
%%%%%%%%%%%%%%%%%%%%%%%%%%%%%%%%%%%%%%%%%%%%%%%%%%%%%%%%%%%%%%%%%%%%%%%%%%%%%%%%%%%%%%%%%%%%%%%%%%%%%%%%%%%%%%%%%%%%%%%%%%%%%%%%%%%%%%%%
\large\textbf{x} \hfill \textbf{Homework - 12}   \\
Email:201900015x@ruc.edu.cn  \hfill ID: 201900015x \\
\normalsize Course: Bayesian \hfill Term: Spring 2020\\
Instructor: Dr. He \hfill Due Date: $20^{th}$ May, 2020 \\
\noindent\rule{7in}{2.8pt}
%%%%%%%%%%%%%%%%%%%%%%%%%%%%%%%%%%%%%%%%%%%
%%%%%%%%%%%%%%%%%%%%%%%%%%%%%%%%%%%%%%%%%%%
%%%%%%%%%%%%%%%%%%%%%%%%%%%%%%%%%%%%%%%%%%%
%%%%%%%%%%%%%%%%%%%%%%%%%%%%%%%%%%%%%%%%%%%

\begin{problem}{6.14}
Show that the multiplicative random walk:
$$X_{t+1}=X_t\epsilon_t$$
is not irreducible when $\epsilon_{t} \sim \mathcal{E} x p(1)$ and $x_{0} \in \mathbb{R}$.(Hint: Show that it produces
two irreducible components.) 
\end{problem}
\begin{solution}
    
\end{solution}

\noindent\rule{7in}{2.8pt}
%%%%%%%%%%%%%%%%%%%%%%%%%%%%%%%%%%%%%%%%%%%
%%%%%%%%%%%%%%%%%%%%%%%%%%%%%%%%%%%%%%%%%%%
\begin{problem}{6.17}
Show that the split chain defined on $\mathcal{X} \times\{0,1\}$ by the following transition kernel: 
\begin{align*}
    P\left(\check{X}_{n+1} \in A \times\{0\} |\left(x_{n}, 0\right)\right) 
    &=\mathbb{I}_{C}\left(x_{n}\right)\left\{\frac{P\left(X_{n}, A \cap C\right)-\epsilon \nu(A \cap C)}{1-\epsilon}(1-\epsilon)\right. \\
    &\left.+\frac{P\left(X_{n}, A \cap C^{c}\right)-\epsilon \nu\left(A \cap C^{c}\right)}{1-\epsilon}\right\} \\
    &+\mathbb{I}_{C^{c}}\left(X_{n}\right)\left\{P\left(X_{n}, A \cap C\right)(1-\epsilon)+P\left(X_{n}, A \cap C^{c}\right)\right\} \\
    \\
    P\left(\check{X}_{n+1} \in A \times\{1\} |\left(x_{n}, 0\right)\right)
    &=\mathbb{I}_{C}\left(X_{n}\right) \frac{P\left(X_{n}, A \cap C\right)-\epsilon \nu(A \cap C)}{1-\epsilon} \epsilon+\mathbb{I}_{C^{c}}\left(X_{n}\right) P\left(X_{n}, A \cap C\right) \epsilon \\
    \\
    P\left(\bar{X}_{n+1} \in A \times\{0\} |\left(x_{n}, 1\right)\right)&=\nu(A \cap C)(1-\epsilon)+\nu\left(A \cap C^{c}\right) \\
    \\
    P\left(\tilde{X}_{n+1} \in A \times\{1\} |\left(x_{n}, 1\right)\right)&=\nu(A \cap C) \epsilon
\end{align*}
satisfies
$$\begin{array}{c}
    P\left(\check{X}_{n+1} \in A \times\{1\} | \check{x}_{n}\right)=\varepsilon \nu(A \cap C) \\
    P\left(\check{X}_{n+1} \in A \times\{0\} | \check{x}_{n}\right)=\nu\left(A \cap C^{c}\right)+(1-\varepsilon) \nu(A \cap C)
    \end{array}$$
for every $\check{x}_{n} \in C \times\{1\} .$ Deduce that $C \times\{1\}$ is an atom of the split chain $\left(\check{X}_{n}\right)$.
\end{problem}
\begin{solution}

\end{solution}

\noindent\rule{7in}{2.8pt}
%%%%%%%%%%%%%%%%%%%%%%%%%%%%%%%%%%%%%%%%%%%
%%%%%%%%%%%%%%%%%%%%%%%%%%%%%%%%%%%%%%%%%%%
\begin{problem}{6.29(a)}
Establish (i) and (ii) of Theorem 6.28. 

(a)Use
$$K^{n}(x, A) \geq K^{r}(x, \alpha) K^{s}(\alpha, \alpha) K^{t}(\alpha, A)$$
for $r+s+t=n$ and $r$ and $s$ such that
$$K^r(x,\alpha)>0,K^s(\alpha,A)>0$$
to derive from the Chapman-Kolmogorov equations that $\mathbb{E}_{x}\left[\eta_{A}\right]=\infty$ when $\mathbb{E}_{\alpha}\left[\eta_{\alpha}\right]=\infty$
\end{problem}
\begin{solution}

\end{solution}

\noindent\rule{7in}{2.8pt}
%%%%%%%%%%%%%%%%%%%%%%%%%%%%%%%%%%%%%%%%%%%
%%%%%%%%%%%%%%%%%%%%%%%%%%%%%%%%%%%%%%%%%%%
\begin{problem}{6.30}
Referring to Definition 6.32, show that if $P\left(\eta_{A}=\infty\right) \neq 0$ then $\mathbb{E}_{x}\left[\eta_{A}\right]=\infty$,
but that $P\left(\eta_{A}=\infty\right)=0$ does not imply $\mathbb{E}_{x}\left[\eta_{A}\right]<\infty$.
\end{problem}
\begin{solution}

\end{solution}
    
\noindent\rule{7in}{2.8pt}
%%%%%%%%%%%%%%%%%%%%%%%%%%%%%%%%%%%%%%%%%%%
%%%%%%%%%%%%%%%%%%%%%%%%%%%%%%%%%%%%%%%%%%%
\end{document}
